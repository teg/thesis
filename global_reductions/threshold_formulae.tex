\subsection{Threshold Formulae}\label{subsection:ThresholdFormulae}

Recently, Je\v r\'abek showed that cut-free $\SKS$ proofs can be constructed in quasipolynomial time from $\SKS$ proofs with cut \cite{Jera::On-the-C:kx}. This is a very surprising result because received wisdom suggests that cut elimination requires exponential-time normalisation, as is the case in Gentzen proof systems. Je\v r\'abek obtained his result by relying on a construction over threshold functions by Atserias, Galesi and Pudl\'ak, in the monotone sequent calculus \cite{AtseGalePudl:02:Monotone:yu}. We note that the monotone sequent calculus specifies a weaker logic than propositional logic because negation is not freely applicable.

The technique that Je\v r\'abek adopts is indirect because normalisation is performed over proofs in the sequent calculus, which are, in turn, related to deep-inference ones by polynomial simulations, originally studied in \cite{Brun:06:Deep-Inf:qy}.

In \emph{Quasipolynomial Normalisation in Deep Inference via Atomic Flows and Threshold Formulae}, we demonstrated again Je\v r\'abek's result, still by adopting, essentially, the Atserias-Galesi-Pudl\'ak technique, and we improved on that as follows:
\begin{enumerate}
%---------------------------------------
\item we significantly simplified the technicalities associated with the use of threshold functions, in particular the formulae and derivations that we adopted were simpler than those in \cite{AtseGalePudl:02:Monotone:yu};
%---------------------------------------
\item our cut-elimination procedure was direct, \emph{i.e.}, it is internal to system $\SKS$.
%---------------------------------------
\end{enumerate}

In this section I generalise those results in the following two ways:
\begin{enumerate}
%---------------------------------------
\item they are extended from cut elimination to streamlining;
%---------------------------------------
\item we observe, in Remark~\vref{remark:FromGammasToThresholds}, a criterion on the kind of formulae we need to make the procedure work, which does not necessarily restrict us to threshold formulae.
%---------------------------------------
\end{enumerate}

As Atserias, Galesi and Pudl\'ak argue, there is no apparent reason for this normalisation problem not to be polynomial. The difficulty in obtaining polynomiality resides in finding a suitable class of derivations as described in Remark~\vref{remark:FromGammasToThresholds}.

We present here the main construction of this section, \emph{i.e.}, a class of derivations $\Gammasf$ that adhere to the condition of Definition~\vref{definition:MultipleIsolatedSubflowsRemoval}. The complexity of the $\Gammasf$ derivations dominates the complexity of the streamlined proof, and is due to the complexity of certain threshold formulae, on which the $\Gammasf$ derivations are based. The $\Gammasf$ derivations are constructed in Definition~\vref{definition:ThresholdDerivations}; this directly leads to Theorem~\vref{theorem:ThresholdDerivations}, which states a crucial property of the $\Gammasf$ derivations and which is the main result of this section.

Threshold formulae realise boolean threshold functions, which are defined as boolean functions that are true if and only if at least $k$ of $n$ inputs are true (see \cite{Wege:87:The-Comp:vn} for a thorough reference on threshold functions).

There are several ways of encoding threshold functions into formulae, and the problem is to find, among them, an encoding that allows us to obtain Theorem~\vref{theorem:ThresholdDerivations}. Efficiently obtaining the property stated in Theorem~\ref{theorem:ThresholdDerivations} crucially depends also on the proof system we adopt.

\Tom{Used to be: In the following, $\lfloor x\rfloor$ denotes the maximum integer $n$ such that $n\le x$.}

\Tom{Added last sentence after Fran\c{c}ois' suggestion:}

In the following, $\underline  n$ (resp., $\overline  n$) denotes the maximum (resp., minimum) integer $x$ such that $x\le n/2$ (resp., $x\ge n/2$). The reason for this notation will become clear in Definition~\vref{definition:ThresholdFormulae}. We will need to split the $n$ atoms $a_1$, $\dots$, $a_n$ into the $\underline  n$ atoms $a_1$, $\dots$, $a_{\underline  n}$ and the $\overline  n$ atoms $a_{\underline  n+1}$, $\dots$, $a_n$. It is important to notice that, for any $n$, $\underline n+\overline n=n$.

The following class of threshold formulae, which we found to work for system $\SKS$, is a simplification of the one adopted in \cite{AtseGalePudl:02:Monotone:yu}.

\newcommand{\avec}[2]{(a_{#1},\dots,a_{#2})}
\Tom{Redefined the macro outputting ${\boldsymbol a}_l^n$ to output $(a_l,\dots,a_n)$. It is no longer a vector of $n-l+1$ atoms, but it is the inputs to a function of arity $n-l+1$.}

%---------------

\Tom{Added explanation:}

\renewcommand{\th}[2]{\mathop{\thetaup_{#1}^{#2}}}

We now define a class of operators $\th kn$, which takes $n$ atoms as arguments and returns a formula that is true if and only if at least $k$ of the inputs are true.

\Tom{Changed away from using $p$ and $q$.}

\Tom{No longer require the atoms to be distinct, as it the restriction might just as well be put where it is used (it will then also be clear straight away why it is needed).}

\Tom{Changed the definition to make it clear that the the names of the atoms are local variables (in programming parlance), so we have a wellfounded definition. The old definition was in terms of global variables (so it could not rely on an implicit renaming). This is the old definition:
Consider $n>0$, distinct atoms $a_1$, \dots, $a_n$, and let $p=\lfloor n/2\rfloor$ and $q=n-p$; for $k\ge0$, we define the \emph{threshold formulae\/} $\th kn\avec1n$ as follows:
\begin{itemize}
%---------------------------------------
\item for any $n>0$ let $\th0n\avec1n\equiv\ttt$;
%---------------------------------------
\item for any $n>0$ and $k>n$ let $\th kn\avec1n\equiv\fff$;
%---------------------------------------
\item $\th11(a_1)\equiv a_1$;
%---------------------------------------
\item for any $n>1$ and $0<k\le n$ let
\[\th kn\avec1n\equiv\bigvee_{\begin{subarray}{l}i+j=k      \\
                                                0\le i\le p\\
                                                0\le j\le q
                             \end{subarray}}
\vlsbr(\th ip\avec1p.\th jq\avec{p+1}n)\;.\]
%---------------------------------------
\end{itemize}
}

%-------------------------------------------------------------------------------
\begin{definition}\label{definition:ThresholdFormulae}
For every $n>0$ and $k\ge0$, we define the operator $\th kn$ inductively as follows:
\[
\th kn\avec1n=
\begin{cases}
    \ttt
  &
    \text{if $k=0$}
  \\
    \fff
  &
    \text{if $k>n$}
  \\
    a_1
  &
    \text{if $n=k=1$}
  \\
    \bigvee_{
      \begin{subarray}{l}
	i+j=k
      \\
	0\le i\le \underline  n
      \\
	0\le j\le \overline  n
      \end{subarray}
    }
    \vlsbr(\th i{\underline  n}\avec1{\underline  n}.\th j{\overline  n}\avec{\underline  n+1}n)
  &
    \text{otherwise.}
\end{cases}
\]
For any $n$ atoms $a_1$, $\dots$, $a_n$, we call $\th kn\avec1n$ the \emph{threshold formula at level $k$} (\emph{with respect to $a_1$, $\dots$, $a_n$\/})\index{threshold formula}.
\end{definition}

See, in Figure~\vref{figure:ThresholdFormulae}, some examples of threshold formulae.

%-------------------------------------------------------------------------------
\begin{figure}
\vlsmallbrackets
\begin{eqnarray*}
%---------------------------------------
\th02(a,b)&\equiv&\ttt
\quad,\\
\noalign{\medskip}
%---------------------------------------
\th12(a,b)&\equiv&\vls[({\vlnos\th11(a)}.{\vlnos\th01(b)}).
                       ({\vlnos\th01(a)}.{\vlnos\th11(b)})]
           \equiv     [(a.\ttt).(\ttt.b)]\\
          &=     &\vls [a      .      b ]
\quad,\\
\noalign{\medskip}
%---------------------------------------
\th22(a,b)&\equiv&\vls({\vlnos\th11(a)}.{\vlnos\th11(b)})\\
          &\equiv&\vls(a.b)
\quad,\\
\noalign{\medskip}
%---------------------------------------
\th03(a,b,c)&\equiv&\ttt
\quad,\\
\noalign{\medskip}
%---------------------------------------
\th13(a,b,c)&\equiv&\vls[({\vlnos\th11(a)}.{\vlnos\th02(b,c)}).
                         ({\vlnos\th01(a)}.{\vlnos\th12(b,c)})]
             \equiv     [(a.\ttt).(\ttt.[(b.\ttt).(\ttt.c)])]\\
            &=     &\vls[a.b.c]
\quad,\\
\noalign{\medskip}
%---------------------------------------
\th23(a,b,c)&\equiv&\vls[({\vlnos\th11(a)}.{\vlnos\th12(b,c)}).
                    ({\vlnos\th01(a)}.{\vlnos\th22(b,c)})]\\
            &=     &\vls[(a.[b.c]).(b.c)]
\quad,\\
\noalign{\medskip}
%---------------------------------------
\th33(a,b,c)&\equiv&\vls({\vlnos\th11(a)}.{\vlnos\th22(b,c)})
             \equiv     [(a.(b.c))]\\
            &=     &\vls(a.b.c)
\quad,\\
\noalign{\medskip}
%---------------------------------------
\th05(a,b,c,d,e)&\equiv&\ttt
\quad,\\
\noalign{\medskip}
%---------------------------------------
\th15(a,b,c,d,e)&\equiv&\vls[({\vlnos\th12(a,b)}.{\vlnos\th03(c,d,e)}).
                             ({\vlnos\th02(a,b)}.{\vlnos\th13(c,d,e)})]\\
                &=     &\vls[a.b.c.d.e]
\quad,\\
\noalign{\medskip}
%---------------------------------------
\th25(a,b,c,d,e)&\equiv&\vls[({\vlnos\th22(a,b)}.{\vlnos\th03(c,d,e)}).
                             ({\vlnos\th12(a,b)}.{\vlnos\th13(c,d,e)}).
                             ({\vlnos\th02(a,b)}.{\vlnos\th23(c,d,e)})]\\
                &=     &\vls[(a.b                                    ).
                             ([a.b]             .[c.d.e]             ).
                                                 (c.[d.e]).(d.e)      ]
\quad,\\
\noalign{\medskip}
%---------------------------------------
\th35(a,b,c,d,e)&\equiv&\vls[({\vlnos\th22(a,b)}.{\vlnos\th13(c,d,e)}).
                             ({\vlnos\th12(a,b)}.{\vlnos\th23(c,d,e)}).
                             ({\vlnos\th02(a,b)}.{\vlnos\th33(c,d,e)})]\\
                &=     &\vls[(a.b               .[c.d.e]             ).
                             ([a.b]             .[(c.[d.e]).(d.e)]   ).
                                                 (c.d.e)              ]
\quad,\\
\noalign{\medskip}
%---------------------------------------
\th45(a,b,c,d,e)&\equiv&\vls[({\vlnos\th22(a,b)}.{\vlnos\th23(c,d,e)}).
                             ({\vlnos\th12(a,b)}.{\vlnos\th33(c,d,e)})]\\
                &=     &\vls[(a.b               .[(c.[d.e]).(d.e)]   ).
                             ([a.b]             .c.d.e               )]
\quad,\\
\noalign{\medskip}
%---------------------------------------
\th55(a,b,c,d,e)&\equiv&\vls({\vlnos\th22(a,b)}.{\vlnos\th33(c,d,e)})\\
                &=     &\vls(a.b.c.d.e)
\quad,\\
\noalign{\medskip}
%---------------------------------------
\th65(a,b,c,d,e)&\equiv&\fff
\quad.
\end{eqnarray*}
\caption{Examples of threshold formulae.}
\label{figure:ThresholdFormulae}
\end{figure}

The formulae for threshold functions adopted in \cite{AtseGalePudl:02:Monotone:yu} correspond, for each choice of $k$ and $n$, to $\bigvee_{i\ge k}\th in\avec1n$. We presume that \cite{AtseGalePudl:02:Monotone:yu} employs these more complicated formulae because the formalism adopted there, the sequent calculus, is less flexible than deep inference, requiring more information in threshold formulae in order to construct suitable derivations.

% \TODO{Remove?}
%
% %-------------------------------------------------------------------------------
% \begin{remark}
% For $n>0$, we have $\th1n\avec1n=\vls[a_1.\vldots.a_n]$ and $\th nn\avec1n=\vls(a_1.\vldots.a_n)$.
% \end{remark}
%
%
% \TODO{Read again:}
% %-------------------------------------------------------------------------------
% \begin{remark}\label{remark:ThersholdSubstitution}
% Given a threshold formula $\th kn\avec1n$ and an atom $a_i$, both $\th kn\avec1n\{a_i/\fff\}$ and $\th kn\avec 1n\{a_i/\ttt\}$ are threshold formula as well. In particular, $\th kn\avec1n\{a_i/\fff\}$ (resp., $\th kn\avec 1n\{a_i/\ttt\}$) is true if and only if at least $k$ (resp., $k-1$) of the atoms $a_1$, $\dots$, $a_{i-1}$, $a_{i+1}$, $\dots$, $a_n$ are true.
% \end{remark}

The size of the threshold formulae dominates the cost of the normalisation procedure, so, we evaluate their size. We leave as an exercise the proof of the following proposition.

%-------------------------------------------------------------------------------
\begin{proposition}\label{proposition:LargestThresholdFormula}
For any $n>0$ and $k\ge0$, $\size{\th kn\avec1n}\le\size{\th{\underline  n+1}n\avec1n}$.
\end{proposition}

%-------------------------------------------------------------------------------
\begin{lemma}\label{lemma:SizeThresholdMax}
The size of\/ $\th{\underline  n+1}n\avec1n$ is $n^{\Ord{\log n}}$.
\end{lemma}

\Tom{Added a bit more explanation to the proof, and changed away from using $p$ and $q$.}

%-------------------------------------------------------------------------------
\begin{proof}
Observe that $\size{\th kn\avec1n}\le\size{\th k{n+1}\avec1{n+1}}$. Consider:
\begin{equation}\label{PropQuasIneq}
\begin{split}
\size{\th{\underline  n+1}n\avec1n}
&=\textstyle\sum_{\begin{subarray}{l}i+j=\underline  n+1    \\
                                     0\le i\le \underline  n\\
                                     0\le j\le {\overline  n}
                  \end{subarray}}
  \left(\size{\th i{\underline  n}\avec1{\underline  n}}+
        \size{\th j{\overline  n}\avec{\underline  n+1}n}\right)             \\
&\le\textstyle\sum_{\begin{subarray}{l}i+j=\underline  n+1\\
                                       0\le i,j\le {\overline  n}
                    \end{subarray}}
  \left(\size{\th i{\overline  n}\avec1{\overline  n}}+
        \size{\th j{\overline  n}\avec1{\overline  n}}\right)                 \\
&\le2({\overline  n}+1)
  \size{\th{\underline {(\overline  n)}+1}{\overline  n}\avec1{\overline  n}}\quad,
\end{split}
\end{equation}
where we use Proposition~\vref{proposition:LargestThresholdFormula}. Let $h=2/\log\frac{3}{2}$, then we show that, for any $n>0$, we have $\size{\th{\underline  n+1}n\avec1n}\le n^{h\log n}$. We reason by induction on $n$; the case $n=1$ trivially holds. For $n>1$, we have that $2(\overline  n+1)\le n^2$, $\overline  n\le n$ and $\overline  n\le\frac{2}{3}n$, so by the inequality~\eqref{PropQuasIneq}, we have
\begin{equation*}
\begin{split}
\size{\th{\underline  n+1}n\avec1n}
&\le2(\overline  n+1)
     {\overline  n}^{h\log\overline  n}
       \\
&\le n^2n^{h\log(\frac{2}{3}n)}=n^{h\log n-h\log\frac{3}{2}+2}=n^{h\log n}
\quad.
\end{split}
\end{equation*}
\end{proof}

%-------------------------------------------------------------------------------
\begin{theorem}\label{theorem:SizeThreshold}
For any $k\ge0$ the size of\/ $\th kn\avec1n$ is $n^{\Ord{\log n}}$.
\end{theorem}

%-------------------------------------------------------------------------------
\begin{proof}
It immediately follows from Proposition~\vref{proposition:LargestThresholdFormula} and Lemma~\vref{lemma:SizeThresholdMax}.
\end{proof}

\Tom{Specify that we are considering $n$ distinct atoms and change away from using $p$ and $q$.}

%-------------------------------------------------------------------------------
\begin{remark}\label{remark:UpsideDownCoweakening}
Given $n>1$ and distinct atoms $a_1$, $\dots$, $a_n$. For $0\le k\le \overline  n$ and $1\le l\le \underline  n$, the following derivation is well defined:
\[\hss
\vlinf{\weu}
      {}
      {\fff}
      {\vls({\th {\underline  n}{\underline  n}\avec1{\underline  n}}\{a_l/\fff\}.\th k{\overline  n}\avec{{\underline  n}+1}n)}
=
\vls(
\vlinf{\weu}
      {}
      {\vls(\ttt)}
      {\vls(a_1.\cdots.a_{l-1}.a_{l+1}.\cdots.a_{\underline  n}.\th k{\overline  n}\avec{{\underline  n}+1}n)}
.\fff)
\;.
\]
Analogously, for $0\le k\le {\underline  n}$ and ${\underline  n}+1\le l\le n$, we can define the following derivation:
\[\hss
\vlinf{\weu}
      {}
      {\fff}
      {\vls(\th k{\underline  n}\avec1{\underline  n}.\th {\overline  n}{\overline  n}\avec{{\underline  n}+1}n\{a_l/\fff\})}
=
\vls(
\vlinf{\weu}
      {}
      {\vls(\ttt)}
      {\vls(\th k{\underline  n}\avec1{\underline  n}.a_{{\underline  n}+1}.\cdots.a_{l-1}.a_{l+1}.\cdots.a_n)}
.\fff)
\;.
\]
Both classes of derivations are used in Definition~\vref{definition:AuxillaryThresholdDerivation}.

The only reason why we require atoms to be distinct is to avoid certain technical problems with substitutions. The same situation occurs in Definitions~\vref{definition:AuxillaryThresholdDerivation} and \vref{definition:ThresholdDerivations}.
\end{remark}

\Tom{Broke one case into two to avoid overfull.}

\newcommand{\Uth}[3]{\mathop{\mathsf\Upsilon_{#1,#2}^{#3}}}
\newcommand{\Dth}[3]{\mathop{\mathsf\Delta_{#1,#2}^{#3}}}
\newcommand{\Gth}[3]{\mathop{\Gammasf_{#1,#2}^{#3}}}
%-------------------------------------------------------------------------------
\begin{definition}\label{definition:AuxillaryThresholdDerivation}
Consider $n>0$, distinct atoms $a_1$, \dots, $a_n$.
\begin{itemize}
%---------------------------------------
%---------------------------------------
\item
For $n>1$ and $1\le l\le n$, we define the derivations $\Uth kln\avec1n$ and $\Dth kln\avec1n$ as follows:
\[\hss
\Uth kln\avec1n=
\begin{cases}
\vlinf{\weu}
      {}
      {\fff}
      {\vls({\vlnos(\th {\underline  n}{\underline  n}\avec1{\underline  n})}\{a_l/\fff\}.\th{k-{\underline  n}}{\overline  n}\avec{{\underline  n}+1}n)}
             &\text{if ${\underline  n}\le k\le n$ and $l\le {\underline  n}$}\\
\noalign{\medskip}
\vlinf{\weu}
      {}
      {\fff}
      {\vls(\th{k-{\overline  n}}{\underline  n}\avec1{\underline  n}.{\vlnos(\th {\overline  n}{\overline  n}\avec{{\underline  n}+1}n)}\{a_l/\fff\})}
             &\text{if ${\overline  n}\le k\le n$ and ${\underline  n}<l$}\\
\noalign{\medskip}
\fff         &\text{otherwise}
              \end{cases}
\]
and
\[
\Dth kln\avec1n=\begin{cases}
\vlinf{\wed}
      {}
      {\th k{\overline  n}\avec{{\underline  n}+1}n}
      {\fff}
             &\text{if $0<k\le {\overline  n}$ and $l\le {\underline  n}$}\\
\noalign{\medskip}
\vlinf{\wed}
      {}
      {\th k{\underline  n}\avec1{\underline  n}}
      {\fff}
             &\text{if $0<k\le {\underline  n}$ and ${\underline  n}<l$}\\
\noalign{\medskip}
\fff         &\text{otherwise}
              \end{cases}\quad.
\]
%---------------------------------------
%---------------------------------------
\item
For $k\ge0$ and $1\le l\le n$, we define the derivations $\vlsmash{\Gth kln\avec1n}$, recursively on $n$, as follows:
\begin{itemize}
%---------------------------------------
\item $\Gth 011(a_1)=\ttt$;
%---------------------------------------
\item for $k>0$, $\Gth k11(a_1)=\fff$;
%---------------------------------------
\item for $k>n$, $\Gth kln\avec1n=\fff$;
%---------------------------------------
\item for $n>1$, $k\le n$ and $l\le\underline  n$, let $\Gth kln\avec1n$ be
\[
\vls[
\bigvee_{\begin{subarray}{l}i+j=k      \\
                            0\le i<{\underline  n}   \\
                            0\le j\le {\overline  n}
         \end{subarray}}(
\Gth il{\underline  n}\avec1{\underline  n}.
\th j{\overline  n}\avec{{\underline  n}+1}n).
\Uth kln\avec1n.\Dth{k+1}ln\avec1n]
\]
%---------------------------------------
\item for $n>1$, $k\le n$ and $\underline  n<l$, let $\Gth kln\avec1n$ be
\[
\vls[
\bigvee_{\begin{subarray}{l}i+j=k      \\
                            0\le i\le {\underline  n}\\
                            0\le j<{\overline  n}
         \end{subarray}}(
\th i{\underline  n}\avec1{\underline  n}.
\Gth j{l-{\underline  n}}{\overline  n}\avec{{\underline  n}+1}n).
\Uth kln\avec1n.\Dth{k+1}ln\avec1n]
\;.
\]
%---------------------------------------
\end{itemize}
%---------------------------------------
%---------------------------------------
\end{itemize}
\end{definition}

\Tom{Removed example of $\Gth kln\avec1n$, as it was not very helpful.}

%-------------------------------------------------------------------------------
\begin{theorem}\label{theorem:AuxillaryThresholdDerivations}
For any $n>0$, $k\ge0$ and\/ $1\le l\le n$, the derivation\/ $\vlsmash{\Gth kln\avec1n}$ has shape
\[
\vlder{}{\{\awd,\awu\}}{\th{k+1}n\avec1n\{a_l/\ttt\}}
                       {\th kn\avec1n\{a_l/\fff\}}
\quad,
\]
and\/ $\size{\Gth kln\avec1n}$ is $n^{\Ord{\log n}}$.
\end{theorem}

\Tom{Broke the derivation horizontally in a different wayto avoid overfull.}

%-------------------------------------------------------------------------------
\begin{proof}
The shape of $\Gth kln\avec1n$ can be verified by inspecting Definition~\vref{definition:AuxillaryThresholdDerivation}. For example, this is the case when $n>1$ and $l\le \underline  n\le k<\overline  n$:
\vlstore
{
}
\[
\vlder{\Gth kln\avec1n}
      {}
      {\th{k+1}n\avec1n\{a_l/\ttt\}}
      {\th kn\avec1n\{a_l/\fff\}}
\;=\;
\vls[
\textstyle\bigvee_{\begin{subarray}{l}i+j=k      \\
                                      0\le i<p   \\
                                      0\le j\le q
                   \end{subarray}}
(\vlder{\Gth ilp\avec1p}
      {}
      {\th{i+1}p\avec1p\{a_l/\ttt\}}
      {\th ip\avec1p\{a_l/\fff\}}
\;\;.\;\;
\th jq\avec{p+1}n)
\;\;.]
\]
\[
\vls[
.\;
\vlinf{\weu}
      {}
      {\fff}
      {\vls({\vlnos(\th pp\avec1p)}\{a_l/\fff\}.\th{k-p}q\avec{p+1}n)}
\;.\;
\vlinf{\wed}
      {}
      {\th{k+1}q\avec{p+1}n}
      {\fff}
]
\quad.
\]
General (co)weak\-en\-ing rule instances can be replaced by their atomic counterparts due to Lemma~\vref{lemma:GenericWeakening}. The size bound on $\Gth kln\avec1n$ follows from Proposition~\vref{proposition:DerivationSubstitution} and Theorem~\vref{theorem:SizeThreshold}.
\end{proof}

\Tom{Rephrased slightly to avoid overfull}

\Tom{Extracted a constructin with superswitches and (co)contractions into its own Lemma.}

\begin{definition}\label{definition:ThresholdDerivations}
Consider $n>0$, distinct atoms $a_1$, \dots, $a_n$. For $k\ge0$, we define the derivation $\vlsmash{\Gammasf^n_k\avec1n}$ to be:
\[\hss
\vlderivation
{
 \vlde{}{\{\cou\}}
 {
  \vlsbr
  (
   \vlder{}{\{\acd,\swi\}}
   {
    \vlsbr
    [
     a_1
    \;\;.\;\;
     \th {k+1}n\avec1n\{a_1/\fff\}
    ]
   }
   {
    \th {k+1}n\avec1n
   }
  \;\;.\;\;\cdots\;\;.\;\;
   \vlder{}{\{\acd,\swi\}}
   {
    \vlsbr
    [
     a_n
    \;\;.\;\;
     \th {k+1}n\avec1n\{a_n/\fff\}
    ]
   }
   {
    \th {k+1}n\avec1n
   }
  )
 }
 {
  \vlde{}{\{\cod\}}
  {
   \th {k+1}n\avec1n
  }
  {
   \vlhy
   {
    \vlsbr
    [
     \vlder{}{\{\acu,\swi\}}
     {
      \th {k+1}n\avec1n
     }
     {
      \vlsbr
      (
       a_1
      \;\;.\;\;
       \vlder{\Gth k1n\avec1n}{}
       {
        \th {k+1}n\avec1n\{a_1/\ttt\}
       }
       {
        \th kn\avec1n\{a_1/\fff\}
       }
      )
     }
    \;\;.\;\;\cdots\;\;.\;\;
     \vlder{}{\{\acu,\swi\}}
     {
      \th {k+1}n\avec1n
     }
     {
      \vlsbr
      (
       a_n
      \;\;.\;\;
       \vlder{\Gth knn\avec1n}{}
       {
        \th {k+1}n\avec1n\{a_n/\ttt\}
       }
       {
        \th kn\avec1n\{a_n/\fff\}
       }
      )
     }
    ]
   }
  }
 }
}\;,
\]
where we use the derivations constructed in the proof of Lemma~\vref{lemma:SuperSwitchContraction}.
\end{definition}

%-------------------------------------------------------------------------------
\begin{theorem}\label{theorem:ThresholdDerivations}
For any $n>0$ and $k\ge0$, the derivation\/ $\vlsmash{\Gammasf^n_k\avec1n}$ has shape
\[
\vlder{}{\SKS\setminus\{\aid,\aiu\}}
{
 \vls([a_1.\th{k+1}n\avec1n\{a_1/\fff\}].\cdots.[a_n.\th{k+1}n\avec1n\{a_n/\fff\}])
}
{
 \vls[(a_1.\th{k}n\avec1n\{a_1/\fff\}).\cdots.(a_n.\th{k}n\avec1n\{a_n/\fff\})]
}
\quad,
\]
and\/ $\size{\Gammasf^n_k\avec1n}$ is $n^{\Ord{\log n}}$.
\end{theorem}
%------------


\newcommand{\frqmis}{{\mathsf{qmis}}}
\newcommand{\QMISR}{\mathsf{QMISR}}
%---------------------------------------
\begin{definition}\label{definition:QuasipolynomialMultipleSubflowRemoval}
For every $n>0$, we define
\begin{itemize}
\item the reduction ${\to_{\frqmis_n}}$ (where $\frqmis$ stands for \emph{quasipolynomial multiple isolated subflows}\index{Quasipolynomial Multiple Isolated Subflows Remover!reduction}); and
\item and the operator the \emph{Quasipolynomial Multiple Isolated Subflows Remover}\index{Quasipolynomial Multiple Isolated Subflows Remover!operator}, $\QMISR_n$,
\end{itemize}
to be special cases of ${\to_{\frmis_n}}$ and $\MISR_n$, respectively, such that, given atoms $\avec 1n$,
\begin{itemize}
\item $N=n$;
\item for $0\le k\le n$ and $1\le i\le n$, $\gamma_{k,i}=(\th kn\avec1n)\{a_i/\fff\}$; and
\item for $1\le k\le n$, $\Gammasf_k=\Gammasf^n_k\avec1n$.
\end{itemize}
\end{definition}
%---------------

%-------------------------------------------------------------------------
\begin{theorem}\label{theorem:SoundQuasipolynomialMultipleSubflowsRemoval}
For every $n>0$, $\to_{\frqmis_n}$ is sound; moreover, if\/ $\Phi\to_{\frqmis_n}\Psi$, then the size of $\Psi$ depends polynomially on the size of $\Phi$ and quasipolynomially on $n$.
\end{theorem}

\begin{proof}
The result follows by Theorem~\vref{theorem:SoundMultipleIsolatedSubflowsRemoval}, Definition~\vref{definition:ThresholdFormulae} and Theorem~\vref{theorem:ThresholdDerivations}.
\end{proof}
%----------

\Tom{Changed from itemize to enumerate.}

%------------------------------------------------------------
\begin{proposition}\label{proposition:QuasipolynomialMultipleIsolatedSubflowRemover}
Given atoms $a_1$, $\dots$, $a_n$ and a derivation $\Phi$ that is in simple form with respect to $a_1$, $\dots$, $a_n$,
\begin{enumerate}
\item $\QMISR_n(\Phi,a_1,\dots,a_n)$ is weakly streamlined with respect to $a_1$, $\dots$, $a_n$;
\item for any atom $b$,
\begin{enumerate}
\item if $\Phi$ is weakly streamlined with respect to $b$, then $\QMISR_n(\Phi,a_1,\dots,a_n)$ is weakly streamlined with respect to $b$, and
\item if $b$ is not the dual of any of $a_1$, $\dots$, $a_n$ and $\Phi$ is in simple form with respect to $b$, then $\QMISR_n(\Phi,a_1,\dots,a_n)$ is in simple form with respect to $b$; and
\end{enumerate}
\item the size of\/ $\QMISR_n(\Phi,a_1,\dots,a_n)$ depends polynomially on the size of\/ $\Phi$, and quasipolynomially on $n$.
\end{enumerate}
\end{proposition}

\begin{proof}
The statements follow by Proposition~\vref{proposition:MultipleIsolatedSubflowRemover} and Theorem~\vref{theorem:ThresholdDerivations}.
\end{proof}
%----------
