\newcommand{\PB}{\mathsf{PB}}

\TODO{Make sure we don't join $a$ with $\bar a$.}

Given a derivation $\Phi$ and an atom $a$, the operator, $\PB$, defined in this section produces a derivation with the same premiss and conclusion as $\Phi$, which is weakly streamlined with respect to both $a$ and $\bar a$. This operator is a strict improvement over $\ISR$, since it does not require the input derivation to be in simple form, and it deals with the dual atoms in parallell. We will see later how a derivation containing $n$ atoms can be weakly streamlined by $n/2$ applications of $\PB$.

The operator is defined in terms of the following flow reduction.

%--------------------------------
\newcommand{\frpb}{{\mathsf{pb}}}
\begin{definition}\label{definition:PathBreaker}
We define the reduction $\to_\frpb$ (where $\frpb$ stands for \emph{path breaker}\index{Path Breaker!reduction}) as follows, for any atomic flows $\phi$ and $\psi$:
\[
\atomicflow
{
(-8, 7)*{\afvjdm{6}{\boldsymbol\epsilon}{}};
( 0, 8)*{\afaid{}{}{}{}{}{}};
( 8, 7)*{\afvjdm{6}{}{\boldsymbol{\epsilon'}}};
(-5, 0)*{\affr{8}{8}};
(-4, 2)*{\aflabelright\phi};
%---
( 5, 0)*{\affr{8}{8}};
( 6, 2)*{\aflabelright{\psi}};
( 8,-7)*{\afvjum{6}{}{\boldsymbol{\iota'}}};
( 0,-8)*{\afaiu{}{}{}{}{}{}};
(-8,-7)*{\afvjum{6}{\boldsymbol\iota}{}};
}
\quad\to_\frpb\quad
\atomicflow
{
%%%%% RED %%%%%
(0,-20)="D";
(0,-10)="Dhalf";
%% contractions
"D"+"D"="A";
%left
"A"+(-14,-15.5)-"D"*{\afvjmcol{23}{Red}};
"A"+(-11,-17)*{\afvjumcol{4}{\boldsymbol\iota}{}{Red}};
%right
"A"+(11,-11.5)-"Dhalf"*{\afvjmcol{11}{Red}};
"A"+(14,-15.5)-"D"*{\afvjmcol{23}{Red}};
"A"+(11,-17)*{\afvjumcol{4}{}{\boldsymbol{\iota'}}{Red}};
% top boxes
(0,0)="A";
"A"+(-11,-14)*{\afcjrmcol{6}{20}{Red}};
"A"+(11,-14)*{\afcjlmcol{6}{20}{Red}};
"A"+( 0,  8)*{\afaidcol{}{}{}{}{}{}{Red}{Red}};
"A"+(-2, -8)*{\afawucol{}{}{}{}{}{Red}};
% join one
"A"+(2,-10)*{\afvjcol{12}{Red}};
% middle boxes
"A"+"D"="A";
"A"+(9.5,-10)*{\afcjlmcol{3}{12}{Red}};
"A"+( 2,-8)*{\afawucol{}{}{}{}{}{Red}};
%%%%% GREEN %%%%%
%% cocontractions
(0,0)="A";
%left
"A"+(-11,17)*{\afvjdmcol{4}{\boldsymbol\epsilon}{}{OliveGreen}};
"A"+"D"+(-14,15.5)*{\afvjmcol{23}{OliveGreen}};
"A"+"Dhalf"+(-11,11.5)*{\afvjmcol{11}{OliveGreen}};
%right
"A"+(11,17)*{\afvjdmcol{4}{}{\boldsymbol{\epsilon'}}{OliveGreen}};
"A"+"D"+(14,15.5)*{\afvjmcol{23}{OliveGreen}};
% middle boxes
"A"+"D"="A";
"A"+(-9.5,10)*{\afcjlmcol{3}{12}{OliveGreen}};
"A"+(-2, 8)*{\afawdcol{}{}{}{}{}{OliveGreen}};
% join two
"A"+(-2,-10)*{\afvjcol{12}{OliveGreen}};
% bottom boxes
"A"+"D"="A";
"A"+(-11,14)*{\afcjlmcol{6}{20}{OliveGreen}};
"A"+(11,14)*{\afcjrmcol{6}{20}{OliveGreen}};
"A"+( 0,-8)*{\afaiucol{}{}{}{}{}{}{OliveGreen}{OliveGreen}};
"A"+( 2, 8)*{\afawdcol{}{}{}{}{}{OliveGreen}};
%%%%% BLACK %%%%%
%% cocontractions
(0,0)="A";
%left
"A"+(-8,5.5)*{\afvjm3};
"A"+(-11,11)*{\affr88};
"A"+(-11,11)*{\copy\contrup};
%right
"A"+(8,5.5)*{\afvjm3};
"A"+"Dhalf"+(11,11.5)*{\afvjm{11}};
"A"+(11,11)*{\affr88};
"A"+(11,11)*{\copy\contrup};
%% contractions
"D"+"D"="A";
%left
"A"+(-11,-11.5)-"Dhalf"*{\afvjm{11}};
"A"+(-8,-5.5)*{\afvjm3};
"A"+(-11,-11)*{\affr88};
"A"+(-11,-11)*{\copy\contrdown};
%right
"A"+(8,-5.5)*{\afvjm3};
"A"+(11,-11)*{\affr88};
"A"+(11,-11)*{\copy\contrdown};
% top boxes
(0,0)="A";
"A"+(-8,5.9)*{\aflabelright{f_1(\boldsymbol\epsilon)}};
"A"+(-5,  0)*{\affr{8}{8}};
"A"+(-6,  2)*{\aflabelright{f_1(\phi)}};
"A"+(-8,-5.3)*{\aflabelright{f_1(\boldsymbol\iota)}};
%
"A"+(8,5.9)*{\aflabelleft{g_1(\boldsymbol{\epsilon'})}};
"A"+( 5,  0)*{\affr{8}{8}};
"A"+( 4,  2)*{\aflabelright{g_1(\psi)}};
"A"+(8,-5.3)*{\aflabelleft{g_1(\boldsymbol{\iota'})}};
% middle boxes
"A"+"D"="A";
"A"+(9.5,10)*{\afcjrm{3}{12}};
"A"+(-9.5,-10)*{\afcjrm{3}{12}};
"A"+(-8,5.9)*{\aflabelright{f_2(\boldsymbol\epsilon)}};
"A"+(-5, 0)*{\affr{8}{8}};
"A"+(-6, 2)*{\aflabelright{f_2(\phi)}};
"A"+(-8,-5.3)*{\aflabelright{f_2(\boldsymbol\iota)}};
%
"A"+(8,5.9)*{\aflabelleft{g_2(\boldsymbol{\epsilon'})}};
"A"+( 5, 0)*{\affr{8}{8}};
"A"+( 4, 2)*{\aflabelright{g_2(\psi)}};
"A"+(8,-5.3)*{\aflabelleft{g_2(\boldsymbol{\iota'})}};
% bottom boxes
"A"+"D"="A";
"A"+(-8,5.9)*{\aflabelright{f_3(\boldsymbol\epsilon)}};
"A"+(-5, 0)*{\affr{8}{8}};
"A"+(-6, 2)*{\aflabelright{f_3(\phi)}};
"A"+(-8,-5.3)*{\aflabelright{f_3(\boldsymbol\iota)}};
%
"A"+(8,5.9)*{\aflabelleft{g_3(\boldsymbol{\epsilon'})}};
"A"+( 5, 0)*{\affr{8}{8}};
"A"+( 4, 2)*{\aflabelright{g_3(\psi)}};
"A"+(8,-5.3)*{\aflabelleft{g_3(\boldsymbol{\iota'})}};
}\quad,
\]
where the evidenced interaction and cut vertices belong to the same connected component.
\end{definition}
%---------------

%----------------------------------------------
\begin{theorem}\label{theorem:SoundPathBreaker}
Reduction $\to_\frpb$ is sound; moreover, if\/ $\Phi\to_\frpb\Psi$, then the size of $\Psi$ depends polynomially on the size of $\Phi$.
\end{theorem}

\begin{proof}
Let $\Phi$ be a derivation with flow $\phi'$, such that $\phi'\to_\frpb\psi'$. We show that there exists a derivation $\Psi$ with flow $\psi'$ and with the same premiss and conclusion as $\Phi$. In the following, we refer to the figure in Definition~\vref{definition:PathBreaker}.

Since the evidenced interaction and cut vertices belong to the same connected component, we can construct, by Theorem~\vref{theorem:aiDecomposedForm} and Convention~\vref{convention:AlternativeAiDecomposedForm}, the following $\ai$-decomposed form of $\Phi$, whose size depends at most cubically on $\Phi$:
\[
\vlder{\Phi'}{}
{
 \vlsbr[\beta\;.\;\vlinf{}{}{\fff}{\vls(a^\phi.\bar a^\psi)}]
}
{
 \vlsbr(\vlinf{}{}{\vls[a^\phi.\bar a^\psi]}{\ttt}\;.\;\alpha)
}\quad,
\]
for some atom $a$ and formulae $\alpha$ and $\beta$.

We combine three copies of $\Phi'$ to obtain the desired derivation $\Psi$ with flow $\psi'$ and the same premiss and conclusion as $\Phi$:

\newbox\DeltaTopK
\setbox\DeltaTopK=
\hbox{$
\vlder{\Phi'}{}
{
 \vlsbr[\beta\;.\;(\vlinf{}{}{\ttt}{a^{f_1(\phi)}}\;.\;\bar a^{g_1(\psi)})]
}
{
 \vlsbr(\vlinf{}{}{\vls[a^{f_1(\phi)}.\bar a^{g_1(\psi)}]}{\ttt}\;.\;\alpha)
}
$}
\newbox\DeltaK
\setbox\DeltaK=
\hbox{$
\vlder{\Phi'}{}
{
 \vlsbr[\beta\;.\;(a^{f_2(\phi)}\;.\;\vlinf{}{}{\ttt}{\bar a^{g_2(\psi)}})]
}
{
 \vlsbr([\vlinf{}{}{a^{f_2(\phi)}}{\fff}\;.\;\bar a^{g_2(\psi)}]\;.\;\alpha)
}
$}
\newbox\DeltaBotK
\setbox\DeltaBotK=
\hbox{$
\vlder{\Phi'}{}
{
 \vlsbr[\beta\;.\;\vlinf{}{}{\fff}{\vls(a^{f_3(\phi)}.\bar a^{g_3(\psi)})}]
}
{
 \vlsbr([a^{f_3(\phi)}\;.\;\vlinf{}{}{\bar a^{g_3(\psi)}}{\fff}]\;.\;\alpha)
}
$}
\[
\Psi\quad=\quad
\vlderivation
{
 \vlin{\cod}{}{\beta}
 {
  \vlin{\swi}{}
  {
   \vls
   [
    \vlinf{\cod}{}{\beta}{\vls[\beta.\beta]}
   \;\;\;\;.\;\;\;\;
    \box\DeltaBotK
   ]
  }
  {
   \vlin{\swi}{}
   {
    \vls
    (
%     \vlinf{\swi}{}
%     {
%      \vls
      [
       \beta
      \;\;\;\;.\;\;\;\;
       \box\DeltaK
      ]
%     }
%     {
%      \vls(\vlsmallbrackets[\beta.\bar a^\psi].\alpha)
%     }
    \;\;\;\;\;.\;\;\;\;\;
     \alpha
    )   
   }
   {
    \vlin{\cod}{}
    {
     \vls
     (
      \box\DeltaTopK
     \;\;\;\;.\;\;\;\;
      \vlinf{\cou}{}{\vls(\alpha.\alpha)}{\alpha}
     )
    }
    {
     \vlhy{\alpha}
    }
   }
  }
 } 
}\qquad.
\]
We know that the size of $\Phi'$ depends at most cubically on the size of $\Phi$, and we know, by Proposition~\vref{proposition:GenericStructural}, that the size of $\Psi$ depends at most quadratically on the size of $\alpha$ and the size of $\beta$. Hence, the size of $\Psi$ depends at most polynomially on the size of $\Phi$.
\end{proof}
%----------

%-------------------------------------
\begin{lemma}\label{lemma:PathBreaker}
Given two atomic flows $\phi$ and $\psi$, such that $\phi\to_\frpb\psi$; in the following we refer to Definition~\ref{definition:PathBreaker}:
\begin{itemize}
\item let $f_1$, $f_2$, $f_3$, $g_1$, $g_2$ and $g_3$ be the evidenced isomorphisms;
\item let $\nu'_\aiu$ be the evidenced cut and let $\nu'_\aid$ be the evidenced interaction vertex in the redex; and
\item let $\nu_\aiu$ be the evidenced cut and let $\nu_\aid$ be the evidenced interaction vertex in the contractum,
\end{itemize}
then, given an interaction (resp., cut) vertex $\nu$ in $\psi$, there is an interaction (resp., cut) vertex $\nu'$ in $\phi$, such that
\begin{itemize}
\item for some $1\le i\le 3$, $\nu=f_i(\nu')$ or $\nu=g_i(\nu')$, or $\nu=\nu_\aid$ and $\nu'=\nu'_\aid$ (resp., $\nu=\nu_\aiu$ and $\nu'=\nu'_\aiu$);
\item if there is a path from $\nu$ to $\bot$ (resp., $\top$) in $\psi$, then there is a path from $\nu'$ to $\bot$ (resp., $\top$) in $\phi$; and
\item if there is a cut (resp., interaction) vertex $\hat\nu$ in $\psi$, such that there is a path from $\nu$ to $\hat\nu$ in $\psi$, then there is a cut (resp., interaction) vertex $\hat\nu'$ in $\phi$, such that, for some $1\le i\le 3$, $\hat\nu=f_i(\nu')$ or $\hat\nu=g_i(\nu')$, or $\hat\nu=\nu_\aiu$ and $\hat\nu'=\nu'_\aiu$ (resp., $\hat\nu=\nu_\aid$ and $\hat\nu'=\nu'_\aid$); and there is a path from $\nu'$ to $\hat\nu'$ in $\phi$.
\end{itemize}
\end{lemma}

\begin{proof}
We consider each case separately:
\begin{itemize}
  \item by definition;
  \item any path from $\nu$ to $\bot$ (resp., $\top$) in $\psi$ must contain an edge $\epsilon$, such that, for some lower (resp., upper) edge $\epsilon'$ of $\phi$ and some $1\le i\le 3$, $f_i(\epsilon')=\epsilon$ or $g_i(\epsilon')=\epsilon$. Hence, there is a path from $\nu'$ to $\bot$ (resp., $\top$) in $\phi$; and
 \item we have to consider two cases:
 \begin{itemize}
  \item for some $1\le i\le 3$, $\nu=f_i(\nu')$ and $\hat\nu=f_i(\hat\nu')$, or $\nu=g_i(\nu')$ and $\hat\nu=g_i(\hat\nu')$, then there is a path from $\nu'$ to $\hat\nu'$ in $\phi$; or
  \item $\nu=g_1(\nu')$ and $\hat\nu=g_2(\hat\nu')$, or $\nu=f_2(\nu')$ and $\hat\nu=f_3(\hat\nu')$ (resp., $\nu=g_2(\nu')$ and $\hat\nu=g_1(\hat\nu')$, or $\nu=f_3(\nu')$ and $\hat\nu=f_2(\hat\nu')$), then there is a path from $\nu'$ to $\nu'_\aiu$ (resp., $\nu'_\aid$) in $\phi$.
 \end{itemize}
\end{itemize}
\end{proof}
%----------

%----------------------------
\begin{definition}\label{definition:DerPathBreaker}
The \emph{Path Breaker}\index{Path Breaker!operator}, $\PB$, is an operator whose arguments are an atom $a$ and a derivation $\Phi$. If $\Phi$ is weakly streamlined with respect to both $a$ and $\bar a$, then $\PB(\Phi,a)=\Phi$; otherwise, consider the following $\ai$-decomposed form of $\Phi$:
\[
\vlder{\Phi'}{}
{
 \vlsbr
 [
  \beta
 \;.\;
  \vlinf{}{}
  {
   \fff
  }
  {
   \vls(a^\psi.\bar a)
  }
 \;.\;\cdots\;.\;
  \vlinf{}{}
  {
   \fff
  }
  {
   \vls(a^\psi.\bar a)
  }
 ]
}
{
 \vlsbr
 (
  \vlinf{}{}
  {
   \vls[a^\psi.\bar a]
  }
  {
   \ttt
  }
 \;.\;\cdots\;.\;
  \vlinf{}{}
  {
   \vls[a^\psi.\bar a]
  }
  {
   \ttt
  }
 \;.\;
  \alpha
 )
}\quad,
\]
with atomic flow

\TODO{Make sure no evidenced interaction/cut can be mapped to from anything else than $a$ or $\bar a$.}

\[
\phi''\;=\;
\atomicflow
{
(-8, 7)*{\afvjdm{6}{\boldsymbol\epsilon}{}};
( 0, 8)*{\afaidm{}{}{}{}{}{}};
( 8, 7)*{\afvjdm{6}{}{\boldsymbol{\epsilon'}}};
(-5, 0)*{\affr{8}{8}};
(-4, 2)*{\aflabelright{\phi'}};
%---
( 5, 0)*{\affr{8}{8}};
( 6, 2)*{\aflabelright{\psi'}};
( 8,-7)*{\afvjum{6}{}{\boldsymbol{\iota'}}};
( 0,-8)*{\afaium{}{}{}{}{}{}};
(-8,-7)*{\afvjum{6}{\boldsymbol\iota}{}};
}
\quad,
\]
such that occurrences of $a$ do not appear in an interaction or cut instance in $\Phi'$. Consider the derivation
\[
\Psi\;=\;
\vlder{\Phi'}{}
{
 \vlsbr
 [
  \beta
 \;\;\;.\;\;\;
  \vlderivation
  {
   \vlin{}{}
   {
    \fff
   }
   {
    \vlde{}{\{\cod\}}
    {
     \vls(a.\bar a)
    }
    {
     \vlhy
     {
      \vls[(a.\bar a).\cdots.(a.\bar a)]
     }
    }
   }
  }
 ]
}
{
 \vlsbr
 (
  \vlderivation
  {
   \vlde{}{\{\cou\}}
   {
    \vls([a.\bar a].\cdots.[a.\bar a])
   }
   {
    \vlin{}{}
    {
     \vls[a.\bar a]
    }
    {
     \vlhy
     {
      \ttt
     }
    }
   }
  }
 \;\;\;.\;\;\;
  \alpha
 )
}\quad,
\]
with atomic flow
\[
\psi''\;=\;
\atomicflow{
(-11, 11.5)*{\afvjdm{15}{\boldsymbol\epsilon}{}};
( 13, 11.5)*{\afvjdm{15}{}{\boldsymbol{\epsilon'}}};
(  1, 18)*{\afaidex{}{}{}{}{}{}{12}{4}};
%-
( -5, 10)*{\affr68};
( -5, 10)*{\copy\contrup};
( -5,  5)*{\afvjm2};
( -8,  0)*{\affr{8}{8}};
( -4,  2)*{\aflabelleft{\phi'}};
( -5, -5)*{\afvjm2};
( -5,-10)*{\affr68};
( -5,-10)*{\copy\contrdown};
%-
(  7, 10)*{\affr68};
(  7, 10)*{\copy\contrup};
(  7,  5)*{\afvjm2};
( 10,  0)*{\affr{8}{8}};
( 14,  2)*{\aflabelleft{\psi'}};
(  7, -5)*{\afvjm2};
(  7,-10)*{\affr68};
(  7,-10)*{\copy\contrdown};
%-
(  1,-18)*{\afaiuex{}{}{}{}{}{}{12}{4}};
(-11,-11.5)*{\afvjum{15}{\boldsymbol\iota}{}};
( 13,-11.5)*{\afvjum{15}{}{\boldsymbol{\iota'}}};
%----
( -6,  0)*{\affr{14}{30}};
(  1, 13)*{\aflabelleft{\phi}};
(9.5,  0)*{\affr{13}{30}};
( 16, 13)*{\aflabelleft{\psi}};
}\quad.
\]
We then define $\PB(\Phi,a)$ to be such that $\Psi\to_\frpb\PB(\Phi,a)$, where $\phi$ and $\psi$ are the flows, by the same names, shown in Definition~\vref{definition:PathBreaker}.
\end{definition}
%---------------

%------------------------------------------------------------
\begin{proposition}\label{proposition:PathBreaker}
Given an atom $a$ and a derivation $\Phi$,
\begin{enumerate}
\item $\PB(\Phi,a)$ is weakly streamlined with respect to both $a$ and $\bar a$;
\item for any atom $b$, if $\Phi$ is weakly streamlined with respect to $b$, then $\PB(\Phi,a)$ is weakly streamlined with respect to $b$; and
\item the size of\/ $\PB(\Phi,a)$ depends polynomially on the size of\/ $\Phi$.
\end{enumerate}
\end{proposition}

\begin{proof}
If $\Phi$ is weakly streamlined with respect to both $a$ and $\bar a$, the result is trivial. Assume $\Phi$ is not weakly streamlined with respect to both $a$ and $\bar a$, and let $\phi$, $\psi$, $\phi'$, $\psi'$, $\phi''$ and $\psi''$ be the atomic flows given in Definition~\ref{definition:DerPathBreaker} and let $\nu_\aid$ (resp., $\nu_\aiu$) be the evidenced interaction (resp., cut) vertex in $\psi''$, then
\begin{enumerate}
\item by Definition~\ref{definition:DerPathBreaker} all the paths from an interaction (resp., cut) vertex whose edges are mapped to from instances of $a$ or $\bar a$ must start from $\nu_\aid$ (resp., $\nu_\aiu$). In Definition~\ref{definition:PathBreaker} we have colored these edges in red (resp., green). Since the red and the green edges never coincide, there are no paths from $\nu_\aid$ to $\nu_\aiu$;
\item if there is a path from an interaction (resp., cut) vertex in the atomic flow of $\PB(\Phi,a)$ whose edges are mapped to from instances of $b$, then, by Lemma~\vref{lemma:PathBreaker}, there is a path from an interaction (resp., cut) vertex in $\phi$ or $\psi$, so also in $\phi'$ or $\psi'$, whose edges are mapped to from instances of $b$. Hence, the statement follows by contradiction; and
\item the statement follows by Theorem~\vref{theorem:SoundPathBreaker}.
\end{enumerate}
\end{proof}
%----------

%-----------------------------------------
\begin{example}\label{example:PathBreaker}
Given a derivation $\Phi$ where the atoms $a_1$ and $a_2$ occur, such that the atomic flow associated with $\Phi$ is
\[
\atomicflow
{
(-2,8)*{\afaid{}{}{}{}{}{}};
(4,6)*{\afvjm4};
(0,0)*{\affr{10}8};
(5,2)*{\aflabelleft{\phi_1}};
(-4,-6)*{\afvjm4};
(2,-8)*{\afaiu{}{}{}{}{}{}};
}\quad
\atomicflow
{
(-2,8)*{\afaid{}{}{}{}{}{}};
(4,6)*{\afvjm4};
(0,0)*{\affr{10}8};
(5,2)*{\aflabelleft{\phi_2}};
(-4,-6)*{\afvjm4};
(2,-8)*{\afaiu{}{}{}{}{}{}};
}\quad
\atomicflow
{
(0,6)*{\afvjm4};
(0,0)*{\affr88};
(4,2)*{\aflabelleft{\psi}};
(0,-6)*{\afvjm4};
}\quad,
\]

\TODO{Alessio said: `\emph{This is a long shot. Perhaps you can argue a bit more and more clearly about the red edges. Not clear what the latter and former subflows are.}'.}

where all the edges in $\phi_1$ are mapped to from $a_1$ and all the edges in $\phi_2$ are mapped to from $a_2$, and there are no edges in $\psi$ that are mapped to from $a_1$ or $a_2$, then the atomic flow associated with $\PB((\Phi,a_1),a_2)$ is the juxtaposition of the following three flows (where indications of the different isomorphisms are left out):

\TODO{adjust labels}

\TODO{state that we ignore isomorphisms}

\TODO{put flows on one page (how?)}

\[
\atomicflow
{
%cocontraction - top
(4,37.5)*{\afvjm{3}};
(4,32)*{\affr{50}8};
(4,32)*{\copy\contrup};
%contraction - bot
(-4,-32)*{\affr{50}8};
(-4,-32)*{\copy\contrdown};
(-4,-37.5)*{\afvjm{3}};
%---------------------
(4,-18)="D";
(0,-9)="Dhalf";
%----------------
%%first
(-20,0)="B";
% cocontractions
"B"-"D"-"D"-(-12,7)="A";
%left
"A"+"Dhalf"+"Dhalf"+(4,-4)*{\afvjm{42}};
"A"+"Dhalf"+(0,-4)*{\afvjm{24}};
"A"+(-4,-4)*{\afvjm6};
% contractions
"B"+"D"+"D"+(-12,7)="A";
%right
"A"+(4,4)*{\afvjm6};
"A"+(0,4)-"Dhalf"*{\afvjm{24}};
"A"+(-4,4)-"Dhalf"-"Dhalf"*{\afvjm{42}};
%---
% top boxes
"B"-"D"="A";
"A"+(-2, 8)*{\afaid{}{}{}{}{}{}};
"A"+( 0,-8)*{\afawu{}{}{}{}{}};
"A"+( 0, 0)*{\affr{10}{8}};
"A"+( 2, 2)*{\aflabelright{\phi_1}};
% join one
"A"+(4,-9)*{\afvjcol{10}{Red}};
% middle boxes
"B"="A";
"A"+(-4, 8)*{\afawd{}{}{}{}{}};
"A"+( 4,-8)*{\afawu{}{}{}{}{}};
"A"+( 0, 0)*{\affr{10}{8}};
"A"+( 2, 2)*{\aflabelright{\phi_1}};
% join two
"A"+(0,-9)*{\afvjcol{10}{Red}};
% bottom boxes
"B"+"D"="A";
"A"+(2,-8)*{\afaiu{}{}{}{}{}{}};
"A"+(0, 8)*{\afawd{}{}{}{}{}};
"A"+(0, 0)*{\affr{10}{8}};
"A"+( 2, 2)*{\aflabelright{\phi_1}};
%----------------
%%second
(0,0)="B";
% cocontractions
"B"-"D"-"D"-(-12,7)="A";
%left
"A"+"Dhalf"+"Dhalf"+(4,-4)*{\afvjm{42}};
"A"+"Dhalf"+(0,-4)*{\afvjm{24}};
"A"+(-4,-4)*{\afvjm6};
% contractions
"B"+"D"+"D"+(-12,7)="A";
%right
"A"+(4,4)*{\afvjm6};
"A"+(0,4)-"Dhalf"*{\afvjm{24}};
"A"+(-4,4)-"Dhalf"-"Dhalf"*{\afvjm{42}};
%---
% top boxes
"B"-"D"="A";
"A"+(-2, 8)*{\afaid{}{}{}{}{}{}};
"A"+( 0,-8)*{\afawu{}{}{}{}{}};
"A"+( 0, 0)*{\affr{10}{8}};
"A"+( 2, 2)*{\aflabelright{\phi_1}};
% join one
"A"+(4,-9)*{\afvjcol{10}{Red}};
% middle boxes
"B"="A";
"A"+(-4, 8)*{\afawd{}{}{}{}{}};
"A"+( 4,-8)*{\afawu{}{}{}{}{}};
"A"+( 0, 0)*{\affr{10}{8}};
"A"+( 2, 2)*{\aflabelright{\phi_1}};
% join two
"A"+(0,-9)*{\afvjcol{10}{Red}};
% bottom boxes
"B"+"D"="A";
"A"+(2,-8)*{\afaiu{}{}{}{}{}{}};
"A"+(0, 8)*{\afawd{}{}{}{}{}};
"A"+(0, 0)*{\affr{10}{8}};
"A"+( 2, 2)*{\aflabelright{\phi_1}};
%----------------
%%third
(20,0)="B";
% cocontractions
"B"-"D"-"D"-(-12,7)="A";
%left
"A"+"Dhalf"+"Dhalf"+(4,-4)*{\afvjm{42}};
"A"+"Dhalf"+(0,-4)*{\afvjm{24}};
"A"+(-4,-4)*{\afvjm6};
% contractions
"B"+"D"+"D"+(-12,7)="A";
%right
"A"+(4,4)*{\afvjm6};
"A"+(0,4)-"Dhalf"*{\afvjm{24}};
"A"+(-4,4)-"Dhalf"-"Dhalf"*{\afvjm{42}};
%---
% top boxes
"B"-"D"="A";
"A"+(-2, 8)*{\afaid{}{}{}{}{}{}};
"A"+( 0,-8)*{\afawu{}{}{}{}{}};
"A"+( 0, 0)*{\affr{10}{8}};
"A"+( 2, 2)*{\aflabelright{\phi_1}};
% join one
"A"+(4,-9)*{\afvjcol{10}{Red}};
% middle boxes
"B"="A";
"A"+(-4, 8)*{\afawd{}{}{}{}{}};
"A"+( 4,-8)*{\afawu{}{}{}{}{}};
"A"+( 0, 0)*{\affr{10}{8}};
"A"+( 2, 2)*{\aflabelright{\phi_1}};
% join two
"A"+(0,-9)*{\afvjcol{10}{Red}};
% bottom boxes
"B"+"D"="A";
"A"+(2,-8)*{\afaiu{}{}{}{}{}{}};
"A"+(0, 8)*{\afawd{}{}{}{}{}};
"A"+(0, 0)*{\affr{10}{8}};
"A"+( 2, 2)*{\aflabelright{\phi_1}};
}
\]
\[
\atomicflow
{
(12,-50)="D";
% cocontractions
(18,-11)="A";
%
"A"+(6,5.5)*{\afvjm3};
"A"+(6,0)*{\affr{34}8};
"A"+(6,0)*{\copy\contrup};
"A"+(2,-29)*{\afvjm{50}};
"A"+(6,-29)*{\afvjm{50}};
"A"+(10,-33)*{\afvjm{58}};
"A"+(14,-54)*{\afvjm{100}};
"A"+(18,-54)*{\afvjm{100}};
"A"+(22,-58)*{\afvjm{108}};
% === BOX ONE ===
(0,-27)="A";
%-
"A"+( -6,24)*{\afaidex{}{}{}{}{}{}31};
%-
"A"+(-12,16)*{\affr{10}8};
"A"+(-12,16)*{\copy\contrup};
"A"+(  0,16)*{\affr{10}8};
"A"+(  0,16)*{\copy\contrup};
%-
"A"+(-16,8)*{\afvj8};
"A"+(-8,8)*{\afcjr88};
"A"+(0,8)*{\afcjrm{16}8};
%
"A"+(-8,8)*{\afcjl88};
"A"+(0,8)*{\afvj8};
"A"+(8,8)*{\afcjrm88};
%
"A"+(0,8)*{\afcjl{16}8};
"A"+(8,8)*{\afcjl88};
"A"+(16,8)*{\afvjm8};
%-
"A"+(-12,0)*{\affr{10}8};
"A"+(0,0)*{\affr{10}8};
"A"+(12,0)*{\affr{10}8};
"A"+(-10,2)*{\aflabelright{\phi_2}};
"A"+(2,2)*{\aflabelright{\phi_2}};
"A"+(14,2)*{\aflabelright{\phi_2}};
%-
"A"+(-8,-8)*{\afcjl88};
"A"+(0,-8)*{\afcjlcol{16}8{Red}};
%-
"A"+(-8,-8)*{\afcjrm88};
"A"+(0,-8)*{\afvj8};
"A"+(8,-8)*{\afcjlcol88{Red}};
%
"A"+(0,-8)*{\afcjrm{16}8};
"A"+(8,-8)*{\afcjr88};
"A"+(16,-8)*{\afvjcol8{Red}};
%
"A"+(  0,-16)*{\affr{10}8};
"A"+(  0,-16)*{\copy\contrdown};
"A"+( 12,-16)*{\affr{10}8};
"A"+( 12,-16)*{\copy\contrdown};
%---
"A"+(0,-24)*{\afawu{}{}{}{}};
"A"+(12,-25)*{\afvjcol{10}{Red}};
% === BOX TWO ===
"A"+"D"="A";
%-
"A"+(-12,24)*{\afawd{}{}{}{}};
%-
"A"+(-12,16)*{\affr{10}8};
"A"+(-12,16)*{\copy\contrup};
"A"+(  0,16)*{\affr{10}8};
"A"+(  0,16)*{\copy\contrup};
%-
"A"+(-16,8)*{\afvj8};
"A"+(-8,8)*{\afcjrcol88{Red}};
"A"+(0,8)*{\afcjrm{16}8};
%
"A"+(-8,8)*{\afcjl88};
"A"+(0,8)*{\afvjcol8{Red}};
"A"+(8,8)*{\afcjrm88};
%
"A"+(0,8)*{\afcjl{16}8};
"A"+(8,8)*{\afcjlcol88{Red}};
%-
"A"+(-12,0)*{\affr{10}8};
"A"+(0,0)*{\affr{10}8};
"A"+(12,0)*{\affr{10}8};
"A"+(-10,2)*{\aflabelright{\phi_2}};
"A"+(2,2)*{\aflabelright{\phi_2}};
"A"+(14,2)*{\aflabelright{\phi_2}};
%-
"A"+(-8,-8)*{\afcjlcol88{Red}};
"A"+(0,-8)*{\afcjl{16}8};
%-
"A"+(-8,-8)*{\afcjrm88};
"A"+(0,-8)*{\afvjcol8{Red}};
"A"+(8,-8)*{\afcjl88};
%
"A"+(0,-8)*{\afcjrm{16}8};
"A"+(8,-8)*{\afcjrcol88{Red}};
"A"+(16,-8)*{\afvj8};
%
"A"+(  0,-16)*{\affr{10}8};
"A"+(  0,-16)*{\copy\contrdown};
"A"+( 12,-16)*{\affr{10}8};
"A"+( 12,-16)*{\copy\contrdown};
%---
"A"+(12,-24)*{\afawu{}{}{}{}};
"A"+(0,-25)*{\afvjcol{10}{Red}};
% === BOX THREE ===
"A"+"D"="A";
%-
"A"+(  0,24)*{\afawd{}{}{}{}};
%-
"A"+(-12,16)*{\affr{10}8};
"A"+(-12,16)*{\copy\contrup};
"A"+(  0,16)*{\affr{10}8};
"A"+(  0,16)*{\copy\contrup};
%-
"A"+(-16,8)*{\afvjcol8{Red}};
"A"+(-8,8)*{\afcjr88};
"A"+(0,8)*{\afcjrm{16}8};
%
"A"+(-8,8)*{\afcjlcol88{Red}};
"A"+(0,8)*{\afvj8};
"A"+(8,8)*{\afcjrm88};
%
"A"+(0,8)*{\afcjlcol{16}8{Red}};
"A"+(8,8)*{\afcjl88};
%-
"A"+(-12,0)*{\affr{10}8};
"A"+(0,0)*{\affr{10}8};
"A"+(12,0)*{\affr{10}8};
"A"+(-10,2)*{\aflabelright{\phi_2}};
"A"+(2,2)*{\aflabelright{\phi_2}};
"A"+(14,2)*{\aflabelright{\phi_2}};
%-
"A"+(-16,-8)*{\afvjm8};
"A"+(-8,-8)*{\afcjl88};
"A"+(0,-8)*{\afcjl{16}8};
%-
"A"+(-8,-8)*{\afcjrm88};
"A"+(8,-8)*{\afcjl88};
"A"+(0,-8)*{\afvj8};
%
"A"+(0,-8)*{\afcjrm{16}8};
"A"+(8,-8)*{\afcjr88};
"A"+(16,-8)*{\afvj8};
%
"A"+(  0,-16)*{\affr{10}8};
"A"+(  0,-16)*{\copy\contrdown};
"A"+( 12,-16)*{\affr{10}8};
"A"+( 12,-16)*{\copy\contrdown};
%-
"A"+(  6,-24)*{\afaiuex{}{}{}{}{}{}31};
%---
"A"+(-20,-16)="A";
"A"+(-20,58)*{\afvjm{108}};
"A"+(-16,54)*{\afvjm{100}};
"A"+(-12,54)*{\afvjm{100}};
"A"+(-8,33)*{\afvjm{58}};
"A"+(-4,29)*{\afvjm{50}};
"A"+( 0,29)*{\afvjm{50}};
"A"+(-4,0)*{\affr{34}8};
"A"+(-4,0)*{\copy\contrdown};
"A"+(-4,-5.5)*{\afvjm3};
}
\]
\[
\atomicflow{
(0,34.5)*{\afvjm3};
(0,29)*{\affr{82}8};
(0,29)*{\copy\contrup};
%
(0,-29)*{\affr{82}8};
(0,-29)*{\copy\contrdown};
(0,-34.5)*{\afvjm3};
%-------------------
(30,0)="B";
%---------------
(0,0)-"B"="A";
"A"+(-10,14.5)*{\afvjm{21}};
"A"+(0,14.5)*{\afvjm{21}};
"A"+(10,14.5)*{\afvjm{21}};
%---
"A"+(-10,0)*{\affr88};
"A"+( -9,2)*{\aflabelright\psi};
"A"+(  0,0)*{\affr88};
"A"+(  1,2)*{\aflabelright\psi};
"A"+( 10,0)*{\affr88};
"A"+( 11,2)*{\aflabelright\psi};
%---
"A"+(-10,-14.5)*{\afvjm{21}};
"A"+(0,-14.5)*{\afvjm{21}};
"A"+(10,-14.5)*{\afvjm{21}};
%---------------
(0,0)="A";
%---
"A"+(-10,14.5)*{\afvjm{21}};
"A"+(0,14.5)*{\afvjm{21}};
"A"+(10,14.5)*{\afvjm{21}};
%---
"A"+(-10,0)*{\affr88};
"A"+( -9,2)*{\aflabelright\psi};
"A"+(  0,0)*{\affr88};
"A"+(  1,2)*{\aflabelright\psi};
"A"+( 10,0)*{\affr88};
"A"+( 11,2)*{\aflabelright\psi};
%---
"A"+(-10,-14.5)*{\afvjm{21}};
"A"+(0,-14.5)*{\afvjm{21}};
"A"+(10,-14.5)*{\afvjm{21}};
%---------------
"A"+"B"="A";
%---
"A"+(-10,14.5)*{\afvjm{21}};
"A"+(0,14.5)*{\afvjm{21}};
"A"+(10,14.5)*{\afvjm{21}};
%---
"A"+(-10,0)*{\affr88};
"A"+( -9,2)*{\aflabelright\psi};
"A"+(  0,0)*{\affr88};
"A"+(  1,2)*{\aflabelright\psi};
"A"+( 10,0)*{\affr88};
"A"+( 11,2)*{\aflabelright\psi};
%---
"A"+(-10,-14.5)*{\afvjm{21}};
"A"+(0,-14.5)*{\afvjm{21}};
"A"+(10,-14.5)*{\afvjm{21}};
}\quad.
\]
We marked some edges in red to point out the fundamental difference between the subflows containing $\phi_1$ and the subflows containing $\phi_2$
\end{example}
%------------
