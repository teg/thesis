Consider a flow $\phi'$ with polarity assignemnt $\pi$, such that $\phi$ is the subflow of $\phi'$ containing all the edges with a positive polarity assignment. We can observe that $\phi$ contains four types of paths: 1) paths from $\top$ to $\bot$; 2) paths from an interaction vertex to $\bot$; 3) paths from $\top$ to a cut vertex; and 4) paths from an interaction vertex to a cut vertex. We can turn $\phi'$ into simple form with respect to $\pi$ if we can make sure that no edge belongs to both a path of type 1) and a path of type 4). In the following reduction, we acheive this by making four copies of $\phi$ each of which only contains one of the above types of paths.

%--------------------------------
\newcommand{\frfb}{{\mathsf{sf}}}
\begin{definition}\label{definition:FourBoxes}
We define the reduction $\to_\frfb$ (where $\frfb$ stands for \emph{simple form}\index{simplifier!reduction}) as follows, for any two flows $\phi$ and $\psi$ that do not contain any interaction or cut vertices:
\[
\vcenter{\hbox{\includegraphics{Figures/redSimRedex}}}
\quad\to_\frfb\quad
\vcenter{\hbox{\includegraphics{Figures/redSimContractum}}}
\quad.
\]
\end{definition}
%---------------

%------------------------------------------------
\begin{remark}\label{remark:RestrictionFourBoxes}
The reduction $\to_\frfb$ would still be sound if we removed the restriction on the flows $\phi$ and $\psi$ in Definition~\vref{definition:FourBoxes}. However, such a reduction would no longer correspond to the intuition described above.
\end{remark}
%-----------

%--------------------------------------------
\begin{theorem}\label{theorem:SoundFourBoxes}
Reduction\/ $\to_\frfb$ is sound; moreover if $\Phi\to_\frfb\Psi$, then the size of $\Psi$ depends at most polynomially on the size of $\Phi$.
\end{theorem}

\begin{proof}
Let $\Phi$ be a derivation with flow $\phi'$, such that $\phi'\to_\frfb\psi'$. We show that there exists a derivation $\Psi$ with flow $\psi'$ and with the same premiss and conclusion as $\Phi$. In the following, we refer to the figure in Definition~\vref{definition:FourBoxes}.

Assume all the edges in $\phi$ are mapped to from occurrences of the atoms $a_1$, $\dots$, $a_n$, and let
\[
\vlder{\Phi'}{\{\aid,\aiu\}}
{
 \vlsbr[\beta\;.\;\vlinf{}{}{\fff}{\vlsmallbrackets\vls(a_n^\phi.\bar a_n^\psi)}\;.\;\cdots\;.\;\vlinf{}{}{\fff}{\vlsmallbrackets\vls(a_1^\phi.\bar a_1^\psi)}]
}
{
 \vlsbr(\vlinf{}{}{\vlsmallbrackets\vls[a_1^\phi.\bar a_1^\psi]}{\ttt}\;.\;\cdots\;.\;\vlinf{}{}{\vlsmallbrackets\vls[a_n^\phi.\bar a_n^\psi]}{\ttt}\;.\;\alpha)
}\quad,
\]
be the $\ai$-decomposed form of $\Phi$.

We show several intermediate derivations which will be used to build $\Psi$. To make it easier to verify the flow of $\Psi$, we will, through a slight misuse of notation, label the atom occurrences of the intermediate derivations to indicate what atomic flow each atom occurrence will map to, once the derivations are combined to create $\Psi$.

Consider the substitution
\[
\sigma=\{a_1^\phi/\vlsmallbrackets\vlsbr([a_1^{f_1(\phi)}.a_1^{f_2(\phi)}].[a_1^{f_3(\phi)}.a_1^{f_4(\phi)}]),\dots,a_n^\phi/\vlsmallbrackets\vlsbr([a_n^{f_1(\phi)}.a_n^{f_2(\phi)}].[a_n^{f_3(\phi)}.a_n^{f_4(\phi)}])\}\;.
\]
We can then obtain, by Proposition~\vref{proposition:DerivationSubstitution}, the derivation $\Phi'\sigma$ with flow
\[
\vcenter{\hbox{\includegraphics{Figures/thmSimplifier1}}}
\qquad.
\]
For every $1\le i\le n$, there exist derivations
\[
\vlinf{}{}
{
 \vls(
  [
   a_i^{f_1(\phi)}
  \;.\;
   \vlinf{}{}{a_i^{f_2(\phi)}}{\fff}
  ]
 \;.\;
  [
   a_i^{f_3(\phi)}
  \;.\;
   \vlinf{}{}{a_i^{f_4(\phi)}}{\fff}
  ]
 )
}
{a_i}
\qquad\mbox{and}\qquad
\vls
(
 \vlinf{}{}
 {a_i}
 {
  \vlsmallbrackets\vls[a_i^{f_1(\phi)}.a_i^{f_2(\phi)}]
 }
\;.\;
 [
  \vlinf{}{}{\ttt}{a_i^{f_3(\phi)}}
 \;.\;
  \vlinf{}{}{\ttt}{a_i^{f_4(\phi)}}
 ]
)
\quad,
\]
which allow us to build
\[
\vlder{\Psi_\top}{}{\alpha\sigma}{\alpha}
\qquad\mbox{and}\qquad
\vlder{\Psi_\bot}{}{\beta}{\beta\sigma}
\quad,
\]
with flows
\[
\vcenter{\hbox{\includegraphics{Figures/thmSimplifier2}}}
\qquad\mbox{and}\qquad
\vcenter{\hbox{\includegraphics{Figures/thmSimplifier3}}}
\qquad,
\]
respectively.
Furthermore, for every $1\le i\le n$, there exist derivations
\[
\Psi_{\ttt,i}\;=\;
\vlderivation
{
 \vlin{=}{}{\vls[([\vlinf{}{}{a_i^{f_1(\phi)}}{\fff}\;.\;a_i^{f_2(\phi)}]\;.\;[\vlinf{}{}{a_i^{f_3(\phi)}}{\fff}\;.\;a_i^{f_4(\phi)}])\;.\;\vlinf{}{}{\bar a_i^{g(\psi)}}{\vls[\bar a_i.\bar a_i]}]}
 {
  \vlin{\swi}{}{\vls[\vlinf{\swi}{}{\vls\vlsmallbrackets[(a_i^{f_2(\phi)}.a_i^{f_4(\phi)}).\bar a_i]}{\vls\vlsmallbrackets(a_i^{f_2(\phi)}.[a_i^{f_4(\phi)}.\bar a_i])}\;.\;\bar a_i]}
  {
   \vlhy{\vls(\vlinf{}{}{\vls[a_i^{f_2(\phi)}.\bar a_i]}{\ttt}\;.\;\vlinf{}{}{\vls[a_i^{f_4(\phi)}.\bar a_i]}{\ttt})}
  }
 }
}
\]
and
\[
\Psi_{\fff,i}\;=\;
\vlderivation
{
 \vlin{\swi}{}{\vls[\vlinf{}{}{\fff}{\vls(a_i^{f_3(\phi)}.\bar a_i)}\;.\;\vlinf{}{}{\fff}{\vls(a_i^{f_4(\phi)}.\bar a_i)}]}
 {
  \vlin{=}{}{\vls(\vlinf{\swi}{}{\vls[a_i^{f_3(\phi)}.(a_i^{f_4(\phi)}.\bar a_i)]}{\vls([a_i^{f_3(\phi)}.a_i^{f_4(\phi)}].\bar a_i)}\;.\;\bar a_i)}
  {
   \vlhy{\vls(([\vlinf{}{}{\ttt}{a_i^{f_1(\phi)}}\;.\;\vlinf{}{}{\ttt}{a_i^{f_2(\phi)}}]\;.\;\vlsmallbrackets[a_i^{f_3(\phi)}.a_i^{f_4(\phi)}])\;.\;\vlinf{}{}{\vls(\bar a_i.\bar a_i)}{\bar a_i^{g(\psi)}})}
  }
 }
}\quad,
\]
which allow us to build
\[
\Psi_\ttt\;=\;
\vlsbr
(
 \vlder{\Psi_{\ttt,1}}{}
 {
  \vlsmallbrackets\vlsbr[a^\phi_1.\bar a^\psi_1]\sigma
 }
 {
  \ttt
 }
\;\;.\;\;\cdots\;\;.\;\;
 \vlder{\Psi_{\ttt,n}}{}
 {
  \vlsmallbrackets\vlsbr[a^\phi_n.\bar a^\psi_n]\sigma
 }
 {
  \ttt
 }
)
\qquad\mbox{and}
\]
\[
\Psi_\fff\;=\;
\vlsbr
[
 \vlder{\Psi_{\fff,n}}{}
 {
  \fff
 }
 {
  \vlsmallbrackets\vlsbr(a^\phi_n.\bar a^\psi_n)\sigma
 }
\;\;.\;\;\cdots\;\;.\;\;
 \vlder{\Psi_{\fff,1}}{}
 {
  \fff
 }
 {
  \vlsmallbrackets\vlsbr(a^\phi_1.\bar a^\psi_1)\sigma
 }
]
\quad,
\]
with flows
\[
\vcenter{\hbox{\includegraphics{Figures/thmSimplifier4}}}
\qquad\mbox{and}\qquad
\vcenter{\hbox{\includegraphics{Figures/thmSimplifier5}}}
\qquad,
\]
respectively.
Combining these derivations we can build
\[
\Psi\;=\;
\vlderivation
{
 \vlde{\vls[\Psi_\bot.\Psi_\fff]}{}
 {
  \beta
 }
 {
  \vlde{\Phi'\sigma}{}
  {
   \vlsbr[\beta.(a^\phi_n.\bar a^\psi_n).\cdots.(a^\phi_1.\bar a^\psi_1)]\sigma
  }
  {
   \vlde{\vls(\Psi_\ttt.\Psi_\top)}{}
   {
    \vlsbr([a^\phi_1.\bar a^\phi_1].\cdots.[a^\phi_n.\bar a^\phi_n].\alpha)\sigma
   }
   {
    \vlhy
    {
     \alpha
    }
   }
  }
 }
}\quad,
\]
with the desired flow.

We know that the size of $\Phi'\sigma$ depends at most polynomially on the size of $\Phi$ by Theorem~\vref{theorem:aiDecomposedForm} and Proposition~\vref{proposition:DerivationSubstitution}, and it is straightforward to observe that the sizes of $\Psi_\ttt$, $\Psi_\top$, $\Psi_\fff$ and $\Psi_\bot$ depend at most linearly on the size of $\Phi$, so the size of $\Psi$ depends at most polynomially on the size of $\Phi$.
\end{proof}
%----------

%----------------------------------
\newcommand{\Simpl}{\mathsf{Si}}
\begin{definition}\label{definition:Simplifier}
The \emph{Simplifier}\index{Simplifier}, $\Simpl$, is an operator whose arguments are distinct and pairwise non-dual atoms $a_1$, $\dots$, $a_n$ and a derivation $\Phi$, with flow
\[
\vcenter{\hbox{\includegraphics{Figures/defSimplifier}}}
\quad,
\]
such that all the edges in $\phi$ are mapped to from occurrences of $a_1$, $\dots$, $a_n$ and no edges in $\psi$ are mapped to from occurrences of $a_1$, $\dots$, $a_n$.
We then define $\Simpl(\Phi,a_1,\dots,a_n)$ to be such that $\Phi\to_\frfb\Simpl(\Phi,a_1,\dots,a_n)$, where $\phi$ and $\psi$ are the flows, by the same names, shown in Definition~\vref{definition:FourBoxes}.
\end{definition}
%---------------

%------------------------------------------------------------
\begin{proposition}\label{proposition:Simplifier}
Given distinct and pairwise non-dual atoms $a_1$, $\dots$, $a_n$, and a derivation $\Phi$,
\begin{enumerate}
\item $\Simpl(\Phi,a_1,\dots,a_n)$ is in simple form with respect to $a_1$, $\dots$, $a_n$;
\item for any atom $b$, if\/ $\Phi$ is weakly streamlined with respect to $b$, then\/ $\Simpl(\Phi,a_1,\dots,a_n)$ is weakly streamlined with respect to $b$; and
\item the size of\/ $\Simpl(\Phi,a_1,\dots,a_n)$ depends at most polynomially on the size of\/ $\Phi$.
\end{enumerate}
\end{proposition}

\begin{proof}
In the following we refer to the figure in Definition~\vref{definition:FourBoxes}:
\begin{itemize}
\item by case (\ref{definition:FlowNormalForms:item:SimpleForm}) of Definition~\vref{definition:FlowNormalForms};
\item by studying the flows in Definition~\ref{definition:FourBoxes} we can observe that for every path from an interaction vertex to a cut vertex in the atomic flow of $\Simpl(\Phi,a_1,\dots,a_n)$ whose edges are mapped to from occurrences of $b$, there is a path from an interaction vertex to a cut vertex in the flow of $\Phi$ whose edges are mapped to from occurrences of $b$; and
\item by Theorem~\vref{theorem:SoundFourBoxes}.
\end{itemize}
\end{proof}
%----------