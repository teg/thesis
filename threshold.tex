\section{Quasi Polynomial Normalisation via Threshold Formulae}

%-------------------------------------------------------------------------------
\begin{proposition}\label{PropAuxNorm}
For any formula $\alpha$ and atom $a$, there exist derivations whose size is cubic in\/ $\size\alpha$ and that have shape
\[
\vlder{}{\{\awd,\acd,\swi\}}{\vls[a.\alpha\{a/\fff\}]}\alpha
\qquad\text{and}\qquad
\vlder{}{\{\awu,\acu,\swi\}}\alpha{\vls(a.\alpha\{a/\ttt\})}
\quad.
\]
\end{proposition}

%-------------------------------------------------------------------------------
\begin{proof}
If there are no occurrences of $a$ in $\alpha$, the desired derivations are
\[
\vlderivation                 {
\vlin{}{}{\vls[a   .\alpha]} {
\vlin= {}{\vls[\fff.\alpha]}{
\vlhy    \alpha             }}}
\qquad\text{and}\qquad
\vlderivation                 {
\vlin= {}\alpha              {
\vlin{}{}{\vls(\ttt.\alpha)}{
\vlhy    {\vls(a   .\alpha)}}}}
\quad.
\]
If there are $h>0$ occurrences of $a$ in $\alpha$, obtain, by repeatedly applying Proposition~\ref{PropSwitch}, the following derivations:
\[
\vlder{}
      {\{\swi\}}
      {\vlsbr[\vlinf{(h-1)\cdot\acd}
                    {}
                    a
                    {\vls[a.\vldots.a]}.\alpha\{a/\fff\}]}
      \alpha
\qquad\text{and}\qquad
\vlder{}
      {\{\swi\}}
      \alpha
      {\vlsbr(\vlinf{(h-1)\cdot\acu}
                    {}
                    {\vls(a.\vldots.a)}
                    a.\alpha\{a/\fff\})}
\quad.
\]
If $\size\alpha=n$, the size of the desired derivations is $\Ord{n^3}$ because we have to apply Proposition~\ref{PropSwitch} at most $\Ord n$ times.
\end{proof}

%-------------------------------------------------------------------------------
\begin{definition}\label{DefNorm}
For $n>0$, let $\Pi$ be a proof in simple form over $\avec1n$, such that it and its atomic flow have shape
\[
\hbox{\phantom{$\vls[\alpha.{}]$}}
\vlder\Psi
      {}
      {\vls[\llap{$\vls[\alpha.{}]$}
            \vlinf{}{}\fff{\vls(a_1.\bar a_1^{\phi_1})}.\cdots.
            \vlinf{}{}\fff{\vls(a_n.\bar a_n^{\phi_n})}]}
      {\vls(\vlinf{}{}{\vls[a_1.\bar a_1^{\phi_1}]}\ttt.\cdots.
            \vlinf{}{}{\vls[a_n.\bar a_n^{\phi_n}]}\ttt)}
\qquad\text{and}\qquad
\atomicflow{
(17.75 ,16  )*{\afaidex{}{}{}{}{}{}{43}8};
(14.25 ,14  )*{\afaidex{}{}{}{}{}{}{13}8};
( 9    ,12  )*{\cdots};
( 7    ,11  )*{\afvj2};
(28.5  ,11  )*{\afvj2};
(18.5  , 9  )*{\aflabelright{\phi_1}};
(29.5  , 9  )*{\aflabelright{\phi_n}};
( 1    , 6  )*{\copy\interdown};
( 4    , 6  )*{\copy\weakdown};
( 5.5  , 6  )*{\affr{13}8};
( 6    , 6  )*{\copy\weakup};
( 8    , 6  )*{\copy\contrdown};
(10    , 6  )*{\copy\contrup};
(16    , 6  )*{\copy\contrdown};
(17.5  , 6  )*{\affr78};
(18    , 6  )*{\copy\contrdown};
(23    , 6  )*{\cdots};
(27    , 6  )*{\copy\contrup};
(28.5  , 6  )*{\affr78};
(29    , 6  )*{\copy\contrup};
( 7    , 1  )*{\afvj2};
(28.5  , 1  )*{\afvj2};
( 9    , 0  )*{\cdots};
( 1    ,-1.5)*{\afvjm7};
(14.25 ,-2  )*{\afaiuex{}{}{}{}{}{}{13}8};
(17.75 ,-4  )*{\afaiuex{}{}{}{}{}{}{43}8}}
\quad,
\]
for some derivation $\Psi$. For $0\le i\le n+1$, let $\theta_i\equiv\th in\avec1n$. For $0\le k\le n$, we define the derivations $\vlder{\Phi_k}{\SKS\setminus\{\aiu\}}{\vls[\alpha.\theta_{k+1}]}{\theta_k}$ as
\[ %%%%% The \dimen's must be adjusted if fonts and layout parameters are changed
\dimen0=2560000sp
\kern\dimen0
\vlder{\Psi_k}
      {\SKS\setminus\{\aiu\}}
      {\kern-\dimen0\vlsbr[\alpha.
       \vlinf{\scriptstyle(n-1)\cdot\gcd}
             {}
             {\theta_{k+1}}
             {\vls[\vlder{}
                         {\{\awu,\acu,\swi\}}
                         {\theta_{k+1}}
                         {\vlsbr(a_1
                                .\;\;
                                \vlder{\Gth k1n\avec1n}
                                      {\{\awd,\awu\}}
                                      {\theta_{k+1}\{a_1/\ttt\}}
                                      {\theta_k    \{a_1/\fff\}})}
                  \;\;\;.\vldots.\;\;\;
                   \vlder{}
                         {\{\awu,\acu,\swi\}}
                         {\theta_{k+1}}
                         {\vlsbr(a_n
                                .\;\;
                                \vlder{\Gth knn\avec1n}
                                      {\{\awd,\awu\}}
                                      {\theta_{k+1}\{a_n/\ttt\}}
                                      {\theta_k    \{a_n/\fff\}})}
                 ]}]                                              }  
      {\vlinf{\llap{$\scriptstyle(n-1)\cdot\cou$}}
             {}
             {\vlsbr(\vlder{}
                           {\{\awd,\acd,\swi\}}
                           {\vls[a_1.\theta_k\{a_1/\fff\}]}
                           {\theta_k                      }
                    \;\;\;.\vldots.\;\;\;
                     \vlder{}
                           {\{\awd,\acd,\swi\}}
                           {\vls[a_n.\theta_k\{a_n/\fff\}]}
                           {\theta_k                      })}
             {\theta_k                                    }}
\quad,
\]
where $\Psi_k=\Psi\{\bar a_1^{\phi_1}/\theta_k\{a_1/\fff\},\dots,\bar a_n^{\phi_n}/\theta_k\{a_n/\fff\}\}$ and where we use Proposition~\ref{PropAuxNorm}. We define the \emph{cut-free form of\/ $\Pi$} as the following proof in $\SKS\setminus\{\aiu\}$:
\[ %%%%% The \dimen's must be adjusted if fonts and layout parameters are changed
\dimen0=910000sp
\dimen1=2970000sp
\dimen2=3870000sp
\vlinf{n\cdot\gcd}
      {\quad.}
      \alpha
      {\kern\dimen2\vlder{\Phi_0}
             {}
             {\kern-\dimen2
              \vlsbr[\alpha
                    .
                    \kern\dimen1
                    \vlder{\Phi_1}
                          {}
                          {\kern-\dimen1
                           \vlsbr[\alpha
                                 .
                                 \cdots.\kern\dimen0
                                 \begin{tabular}{@{}c@{}}
                                 $\theta_2$\\
                                 $\vdots$\\
                                 $\kern-\dimen0
                                  \vlsbr[\alpha
                                        .
                                        \vlder{\Phi_n}
                                              {}
                                              {\vlsbr[\alpha
                                                     .\theta_{n+1}]}
                                              {\theta_n}]$
                                 \end{tabular}]}
                          {\theta_1}]}
            {\theta_0}}
\]
(We recall that $\theta_0\equiv\ttt$ and $\theta_{n+1}\equiv\fff$.)
\end{definition}

\begin{theorem}\label{ThPreNorm}
Given any proof\/ $\Pi$ of $\alpha$ in\/ $\SKS$, we can construct a proof of $\alpha$ in\/ $\SKS\setminus\{\aiu\}$ in time quasipolynomial in the size of\/ $\Pi$.
\end{theorem}

%-------------------------------------------------------------------------------
\begin{proof}
By Theorem~\ref{ThSimpleForm} we can construct from $\Pi$, in polynomial time, a proof $\Pi'$ of $\alpha$ in simple form. We can then proceed with the construction of Definition~\ref{DefNorm}, to which we refer here. For $0\le k\le n$, constructing $\Phi_k$ requires quasipolynomial time because of Propositions~\ref{PropGenAtPol}, \ref{PropSubst} and \ref{PropAuxNorm} and Theorems~\ref{TheoQuas} and \ref{TheoThrDer}, and because obtaining $\Psi_k$ from $\Psi$ requires quasipolynomial time. Constructing the cut-free from of $\Pi'$ from $\Phi_0$, \dots, $\Phi_n$ is done in polynomial time.
\end{proof}

%-------------------------------------------------------------------------------
\begin{remark}
In Figure~\ref{FigNormFlow}, we show the atomic flow of the cut-free form obtained from a proof $\Pi$ in simple form. We refer to Definition~\ref{DefNorm}. Let the following be the flow of the simple core $\Psi$ of $\Pi$:
\[
\atomicflow{
(12  ,12)*{\afvjm4};
(20  ,12)*{\afvjm4};
(12  , 9)*{\aflabelright\phi};
(21.5, 9)*{\aflabelright\psi};
( 2  , 6)*{\copy\interdown};
( 5  , 6)*{\copy\weakdown};
( 7  , 6)*{\copy\weakup};
( 7  , 6)*{\affr{14}8};
( 9  , 6)*{\copy\contrdown};
(11  , 6)*{\copy\contrup};
(18.5, 6)*{\copy\contrdown};
(20  , 6)*{\affr78};
(20.5, 6)*{\copy\contrup};
( 2  , 0)*{\afvjum4\alpha{}};
(12  , 0)*{\afvjm4};
(20  , 0)*{\afvjm4};
}\quad,
\]
where $\psi$ is the union of flows $\phi_1$, \dots, $\phi_n$, and where we denote by $\alpha$ the edges corresponding to the atom occurrences appearing in the conclusion $\alpha$ of $\Pi$. We then have that, for $0<k<n$, the flow of $\Phi_k$ is $\phi'_k$, as in Figure~\ref{FigNormFlow}, where $\psi_k$ is the flow of the derivation $\Psi_k$. The flows of $\Phi_0$ and $\Phi_n$ are, respectively, $\phi'_0$ and $\phi'_n$.
\end{remark}

%-------------------------------------------------------------------------------
\begin{figure}
\[
\atomicflow{
(1,47)="A";
"A"+(16,15  )*{\aflabelright{\phi'_0}};
"A"+(12,14  )*{\afawdm{}{}{}{}};
"A"+(12, 9  )*{\aflabelright\phi};
"A"+( 2, 6  )*{\copy\interdown};
"A"+( 5, 6  )*{\copy\weakdown};
"A"+( 7, 6  )*{\copy\weakup};
"A"+( 7, 6  )*{\affr{14}8};
"A"+( 9, 6  )*{\copy\contrdown};
"A"+(11, 6  )*{\copy\contrup};
"A"+(9 , 4.5)*{\affr{19}{23}};
"A"+(16, 4  )*{\afawdm{}{}{}{}};
"A"+(12, 1  )*{\afvjm2};
"A"+(14,-3  )*{\affr86};
"A"+(14,-3  )*{\copy\contrdown};
"A"+(14,-8.5)*{\afvjum5{}{\theta_1}};
(15  ,35   )*{\vdots};
(15  ,28.5 )*{\afvjdm5{}{\theta_k}};
(22.5,26   )*{\aflabelright{\phi'_k}};
(15  ,23   )*{\affr46};
(15  ,23   )*{\copy\contrup};
(13  ,19   )*{\afcjrm22};
(17.5,15   )*{\afcjlm3{10}};
(12  ,15   )*{\affr46};
(12  ,15   )*{\copy\contrdown};
(12  ,11   )*{\afvjm2};
(22  ,10   )*{\aflabelright{\psi_k}};
(12  , 9   )*{\aflabelright\phi};
( 2  , 6   )*{\copy\interdown};
( 5  , 6   )*{\copy\weakdown};
( 7  , 6   )*{\copy\weakup};
( 7  , 6   )*{\affr{14}8};
( 9  , 6   )*{\copy\contrdown};
(11  , 6   )*{\copy\contrup};
(12  , 6   )*{\affr{25}{10}};
(12  , 6   )*{\affr{26}{42}};
(18  , 6   )*{\copy\contrdown};
(19  , 6   )*{\affr68};
(20  , 6   )*{\copy\contrup};
(12  , 1   )*{\afvjm2};
(19  , 1   )*{\afvjm2};
(12  ,-3   )*{\affr46};
(12  ,-3   )*{\copy\contrup};
(18  ,-3   )*{\copy\weakdown};
(19  ,-3   )*{\affr66};
(20  ,-3   )*{\copy\weakup};
(13  ,-7   )*{\afcjlm22};
(17.5,-7   )*{\afcjrm32};
(15  ,-11  )*{\affr46};
(15  ,-11  )*{\copy\contrdown};
(15  ,-16.5)*{\afvjum5{}{\theta_{k+1}}};
(0,-35)="B";
"B"+( 2  ,-15  )*{\afvjm{10}};
"B"+(12  ,-14  )*{\afawum{}{}{}{}};
"B"+(19  ,-14  )*{\afawum{}{}{}{}};
"B"+( 2  , -6  )*{\copy\interdown};
"B"+( 5  , -6  )*{\copy\weakdown};
"B"+( 7  , -6  )*{\copy\weakup};
"B"+( 7  , -6  )*{\affr{14}8};
"B"+( 9  , -6  )*{\copy\contrdown};
"B"+(11  , -6  )*{\copy\contrup};
"B"+(18  , -6  )*{\copy\contrdown};
"B"+(19  , -6  )*{\affr68};
"B"+(20  , -6  )*{\copy\contrup};
"B"+(11  , -4.5)*{\affr{23}{23}};
"B"+(12  , -3  )*{\aflabelright\phi};
"B"+(13  , -1  )*{\afcjrm22};
"B"+(17.5, -1  )*{\afcjlm32};
"B"+(15  ,  3  )*{\affr66};
"B"+(15  ,  3  )*{\copy\contrup};
"B"+(20  ,  6  )*{\aflabelright{\phi'_n}};
"B"+(15  ,  8.5)*{\afvjdm5{}{\theta_n}};
"B"+(15  ,  15 )*{\vdots};
(-1.5, 28  )*{\afcjrm9{42}};
( 0  ,-20  )*{\afcjrm4{44}};
(-6  ,-24  )*{\afvjm{62}};
(-2  ,-48.5)*{\afvjm{13}};
(-4  ,-53  )*{\cdots};
( 0  ,-53  )*{\cdots};
(-2  ,-58  )*{\affr{10}6};
(-2  ,-58  )*{\copy\contrdown};
(-2  ,-63  )*{\afvjum4\alpha{}};
}
\]
\caption{Atomic flow of a proof in cut-free form.}
\label{FigNormFlow}
\end{figure}