\chapter{Introduction}

The mathematical proofs of journal articles and text books are usually informal arguments expressed in a natural language. In \emph{proof theory} we consider proofs to be well-defined mathematical objects expressed in some formal language, which we call a \emph{formalism}.

Since proofs are mathematical objects, it is natural to ask the question:

\emph{When are two mathematical proofs the same?}

This question was considered for inclusion in Hilbert's famous 1900 lecture \cite{Thie:03:Hilberts:yu}, and remains open today.

Formal proofs, generally speaking, are dense and terse creations that completely obscure any ingenuity that might have existed in the mind of the mathematician. \emph{Semantics} is the discipline which studies the essence, or meaning of proofs. Finding a good semantics of proofs would solve Hilbert's problem.

\emph{abstract}

\emph{concrete}

This thesis was written in the context of a program whose aim it is to find the semantics of proofs. The following slogan, which I will now explain, summarises our approach:
\[
\mbox{Locality}\qquad\rightarrow\qquad\mbox{Geometry}\qquad\rightarrow\qquad\mbox{Semantics}
\]
Given a formalism that allows \emph{locality}, we can find \emph{geometric} structures in our proofs, which we hope will lead to the discovery of a semantics.

The presence of `inessential details' in a proof, we call \emph{bureaucracy}. As an example of the kind of bureaucracy we are interested in, consider a proof that contains two independent arguments, \emph{i.e.}, the order of the arguments does not matter. The representation of the proof is considered bureaucratic, if one has to commit to an arbitrary order.

Girard's \emph{proof net} are geometric representations of \emph{linear logic} proofs, modulo bureaucracy \cite{Gira:87:Linear-L:wm}.

\TODO{proof system}

Motivated by the success of linear logic, we wish to use the same approach for classical logic, and indeed there are several notions of proof nets for classical logic. So far, none of them have been shown to constitute a proof system, but the results are promising.

In order to remove bureaucracy, one first has to be able to observe it. In terms of the example given above, this means to observe when parts of a proof are independent. In linear logic this is possible due to the restriction to linear inference rules. However, not all logics can be expressed in terms of linear inference rules, which motivated the concept of \emph{locality}.

The amount of information needed to verify a local inference rule is bounded by a constant. Intuitively, this condition limits the interdependence of inference rules, and so makes it easier to observe when parts of a proof are independent.

There are two known classes of local inference rules: linear and \emph{atomic}. An atomic inference rule is a rule which is restricted to only apply to atoms.  An example of a non-local rule is a rule which requires the comparison of two unbounded formulae, for instance a contraction.

\emph{Deep inference} \cite{Gugl:06:A-System:kl} is a methodology which allows for more general formalisms than the traditional Gentzen style. In particular, deep-inference formalisms can express classical logic using only local inference rules \cite{BrunTiu:01:A-Local-:mz}, something which is impossible in the traditional formalisms \cite{Brun:03:Two-Rest:mn}.

\TODO{logical system}



\emph{geometry}

\emph{proof nets} 

\emph{linearity}

\emph{atomicity}

\emph{locality}

---

\emph{classical logic}

\emph{deep inference}  (forced: )

\emph{normalisation}

\emph{atomic flows} \cite{GuglGund:07:Normalis:lr}

