\chapter{Introduction}

This thesis introduces \emph{atomic flows}, a language for studying normalisation of derivations in classical propositional logic.

As for natural deduction and the sequent calculus \cite{Gent:69:Investig:xi}, we intend normalisation as eliminating cuts, or, more in general, `detours', from derivations. However, contrary to natural deduction and the sequent calculus, we consider derivations to be top-down symmetric and `detours' to be more than just the existence of cuts.

Atomic flows were motivated by wanting to generalise traditional normalisation, based on elimination of inference rules, to a new notion of normalisation based on elimination of dependencies between inference rules. In order to formalise the notion of `detours', atomic flows were defined by discarding from derivations all but the information about casual relations between creation and destruction of atoms (see Figure~\vref{figure:ExampleAtomicFlows}).

We claim that atomic flows give us a more general view of normalisation because they gives us new normal forms, of which the traditional normal forms are special cases; at the same time they show that we need less of the information available to us in order to obtain normalisation.

Due to the top-down symmetry, and the generalisation of `detours', atomic flows allow us to describe a range of new normal forms (see Definition~\vref{definition:FlowNormalForms}). In the special cases where `cut elimination' makes sense, the new normal forms coincide with the traditional ones, as expected (see \vref{proposition:FlowNormalFormsNoUpper}).

Furthermore, it turned out that the information contained in atomic flows is sufficient to design normalisation procedures. This is surprising, since atomic flows throw away all information about logical connectives and logical inference rules, exactly the information we expected to be most crucial in order to obtain normalisation.

In fact, the principal advantage of atomic flows is that it allows us to use the graphical language of atomic flows to describe the gist of our results. Once the language is mastered, the technical details of most of our results can easily be reconstructed based on the illustrations alone. Illustrations describing our normalisation procedures are given in Definition~\vref{definition:IsolatedSubflowRemoval}, Definition~\vref{definition:PathBreaker}, Definition~\vref{definition:MultipleIsolatedSubflowsRemoval} and Figure~\vref{figure:ReductionRules}.

The thesis is split in three parts. In the first part I introduce derivations in classical propositional logic and show some elementary results. In the second part I define atomic flows, show how flows are extracted from derivations and define normal forms of derivations in terms of their associated atomic flows. Finally, in the last part I define several normalisation procedures based on rewriting atomic flows.
