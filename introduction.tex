This thesis develpos a graphical formalisms called \emph{atomic flows} for the study and design of normalisation techniques for propositional classical logic in deep inference. Furthermore, atomic flows are employed to define normal forms of derivations and normalisation procedures are given in terms of atomic flows to obtain these normal forms.

\section{A Deep Inference Formalism}

% What is deep inference?
\emph{Deep inference} \cite{Gugl:06:A-System:kl} is a methodology that allows generalisations of Gentzen's \emph{natural deduction} and \emph{sequent calculus} \cite{Gent:69:Investig:xi}. The standard deep-inference formalism, the \emph{calculus of structures}, generalises the sequent calculus by allowing deduction at any place in a formula, rather than restricting it to the main connective. As a consequence, it is possible for all inference rules to be unary. In other words, proofs are represented as lists of formulae rather than as trees of sequents. For an introduction to deep inference see \cite{Gugl::Deep-Inf:uq}.

% What is the functorial calculus?
In this thesis, a new deep-inference formalism, named the \emph{functorial calculus}, is presented. While, in the sequent calculus, the juxtaposition of two proofs denotes that they are composed by a conjunction, in the functorial calculus, this horizontal composition is generalised to allow both disjunctions and conjunctions. In other words, proofs are represented as directed acyclic graphs of formulae rather than as trees of sequents. Examples of derivations in the calculus of structures, as well as the corresponding derivations in the functorial calculus are given in the two first rows of Figure~\vref{figure:ExampleAtomicFlows}.

\newcommand{\ai   }{{\mathsf{ai}}}
\newcommand{\aw   }{{\mathsf{aw}}}
\newcommand{\ac   }{{\mathsf{ac}}}
\newcommand{\aid  }{{\ai{\downarrow}}}
\newcommand{\awd  }{{\aw{\downarrow}}}
\newcommand{\acd  }{{\ac{\downarrow}}}
\newcommand{\aiu  }{{\ai{\uparrow}}}
\newcommand{\awu  }{{\aw{\uparrow}}}
\newcommand{\acu  }{{\ac{\uparrow}}}
\newcommand{\swi  }{\mathsf{s}}
\newcommand{\med  }{\mathsf{m}}
\newcommand{\asor }{{=_\mathsf{a}{\downarrow}}}
\newcommand{\asand}{{=_\mathsf{a}{\uparrow}}}
\newcommand{\coor }{{=_{\vee\mathsf{c}}}}
\newcommand{\coand}{{=_{\wedge\mathsf{c}}}}
\newcommand{\fffd }{{{=_{\fff}}{\downarrow}}}
\newcommand{\fffu }{{{=_{\fff}}{\uparrow}}}
\newcommand{\tttd }{{{=_{\ttt}}{\downarrow}}}
\newcommand{\tttu }{{{=_{\ttt}}{\uparrow}}}
\newcommand{\tttord }{{{=_{\ttt\vee}}{\downarrow}}}
\newcommand{\fffandd }{{{=_{\fff\wedge}}{\downarrow}}}
\newcommand{\tttoru }{{{=_{\ttt\vee}}{\uparrow}}}
\newcommand{\fffandu }{{{=_{\fff\wedge}}{\uparrow}}}

\newcommand{\fff}{\mathsf f}
\newcommand{\ttt}{\mathsf t}
\newcommand{\RD}[1]{{\color{Red}#1}}
\newcommand{\GR}[1]{{\color{Green}#1}}
\newcommand{\DO}[1]{{\color{DarkOrchid}#1}}
\newcommand{\PR}[1]{{\color{ProcessBlue}#1}}
\newcommand{\MG}[1]{{\color{Magenta}#1}}
\newcommand{\SG}[1]{{\color{SpringGreen}#1}}
\newcommand{\RS}[1]{{\color{RawSienna}#1}}
\newcommand{\YO}[1]{{\color{YellowOrange}#1}}
\newcommand{\PW}[1]{{\color{Periwinkle}#1}}
%-------------------------------------------------------------------------------
\begin{figure}
\[
%---------------------------------------
\begin{array}{@{}c@{}c@{}c@{}}
\vlderivation                                                {
\vlin=   {}{\ttt                                  }         {
\vlin\aiu{}{\vls[\fff.\ttt]                       }        {
\vlin=   {}{\vls[(\GR{a}.\RD{\bar a}).\ttt]       }       {
\vlin\swi{}{\vls[[(\RD{\bar a}.\GR{a}).\ttt].\ttt]}      {
\vlin=   {}{\vls[(\RD{\bar a}.[\GR{a}.\ttt]).\ttt]}     {
\vlin\swi{}{\vls[([\GR{a}.\ttt].\RD{\bar a}).\ttt]}    {
\vlin=   {}{\vls([\GR{a}.\ttt].[\RD{\bar a}.\ttt])}   {
\vlin\med{}{\vls([\GR{a}.\ttt].[\ttt.\RD{\bar a}])}  {
\vlin=   {}{\vls[(\GR{a}.\ttt).(\ttt.\RD{\bar a})]} {
\vlin\aid{}{\vls[\GR{a}.\RD{\bar a}]              }{
\vlhy      {\ttt                                  }}}}}}}}}}}}
\qquad&
%-------------------
\vlderivation                                                              {
\vlin\aiu{}
   {\vls(\DO{a}.\fff)                                            }        {
\vlin=   {}
   {\vls(\DO{a}.(\PR{a}.\MG{\bar a}))                            }       {
\vlin\acu{}
   {\vls((\DO{a}.\PR{a}).\MG{\bar a})                            }      {
\vlin=   {}
   {\vls(\SG{a}.\MG{\bar a})                                     }     {
\vlin\aiu{}
   {\vls([\fff.\SG{a}].\MG{\bar a})                              }    {
\vlin\acd{}
   {\vls([(\RD{a}.\RS{\bar a}).\SG{a}].\MG{\bar a})              }   {
\vlin\swi{}
   {\vls([(\RD{a}.[\GR{\bar a}.\YO{\bar a}]).\SG{a}].\MG{\bar a})}  {
\vlin=   {}
   {\vls((\RD{a}.[[\GR{\bar a}.\YO{\bar a}].\SG{a}]).\MG{\bar a})} {
\vlin\aid{}
   {\vls((\RD{a}.[\GR{\bar a}.[\YO{\bar a}.\SG{a}]]).\MG{\bar a})}{
\vlhy
   {\vls((\RD{a}.[\GR{\bar a}.\ttt]).\MG{\bar a})                }}}}}}}}}}}
\qquad&
%-------------------
\vlderivation                                                            {
\vlin=   {}{\vls(([\RS{a}.\YO{b}].\PW{a}).([\GR{a}.\DO{b}].\SG{a}))}    {
\vlin\med{}{\vls(([\RS{a}.\YO{b}].[\GR{a}.\DO{b}]).(\PW{a}.\SG{a}))}   {
\vlin\acu{}{\vls([(\RS{a}.\GR{a}).(\YO{b}.\DO{b})].(\PW{a}.\SG{a}))}  {
\vlin\acu{}{\vls([(\RS{a}.\GR{a}).(\YO{b}.\DO{b})].\MG{a})         } {
\vlin\acu{}{\vls([(\RS{a}.\GR{a}).\PR{b}].\MG{a})                  }{
\vlhy      {\vls([\RD{a}.\PR{b}].\MG{a})                           }}}}}}}
\\
\noalign{\bigskip}
%---------------------------------------
\vlderivation                                                      {
\vlin\swi{}{\vlsbr[\vlinf{\swi}
                         {}
                         {\vls[\vlinf{}
                                     {}
                                     {\fff}
                                     {\vls(\GR{a}.\RD{\bar a})}
                              \;.\;
                              \ttt]}
                         {\vls([\GR{a}.\ttt].\RD{\bar a})}
                  \;\;.\;\;
                   \ttt
                  ]                                            }  {
\vlin\med{}{\vls([\GR{a}.\ttt].[\ttt.\RD{\bar a}])             } {
\vlin{}  {}{\vls[\GR{a}.\RD{\bar a}]                           }{
\vlhy      {\ttt                                               }}}}}
\qquad&
%-------------------
\vlinf=
      {}
      {\vls(\DO{a}\;.\;\vlinf{}{}\fff{\vls(\PR{a}.\MG{\bar a})})}
      {\vlsbr(\vlinf\swi
                    {}
                    {\vls[\vlinf{}
                                {}
                                \fff
                                {\vls(\RD{a}
                                     \;.\;\vlinf{}
                                            {}
                                            {\RS{\bar a}}
                                            {\vls[\GR{\bar a}.\YO{\bar a}]}
                                     )}
                         \;\;.\;\;
                         \vlinf{}{}{\vls(\DO{a}.\PR{a})}{\SG{a}}
                         ]}
                    {\vls(\RD{a}
                         \;.\;[\GR{\bar a}
                          \;.\;\vlinf{}
                                 {}
                                 {\vls[\YO{\bar a}.\SG{a}]}
                                 \ttt
                          ]
                         )}
            \;\;\;\;.\;\;\;\;
            \MG{\bar a}
            )}
\qquad&
%-------------------
\vls(\vlinf\med
           {}
           {\vls([\RS{a}.\YO{b}].[\GR{a}.\DO{b}])}
           {\vls[\vlinf{}{}{\vls(\RS{a}.\GR{a})}{\RD{a}}
                \;.\;{\vlinf{}{}{\vls(\YO{b}.\DO{b})}{\PR{b}}}
                ]}
    \;\;.\;\;
     \vlinf{}{}{\vls(\PW{a}.\SG{a})}{\MG{a}}
    )
\\
\noalign{\bigskip}
%---------------------------------------
\vcenter{\hbox{\includegraphics{Figures/flowExtract1}}}
\qquad&
%-------------------
\vcenter{\hbox{\includegraphics{Figures/flowExtract2}}}
\qquad&
%-------------------
\vcenter{\hbox{\includegraphics{Figures/flowExtract3}}}
\end{array}
\]
\caption{Examples of derivations in the calculus of structures (top row), their translation into the functorial calculus (middle row), and the flows associated with the latter (bottom row).}
\label{figure:ExampleAtomicFlows}
\end{figure}
%-----------

% What is the relationship between CoS and FC?
The calculus of structures and the functorial calculus are closely related and translations between the two are given. The relationship between the two formalisms is explored further in \cite{GuglGundPari::A-Proof-:fk}, where a generalisation, called \emph{open deduction}, is presented. It is shown there that a functorial calculus proof corresponds to an equivalence class of calculus of structures proofs.

% What is propositional classical logic in deep inference?
The focus of this thesis is propositional classical logic. By exploiting the symmetry available in deep-inference, it is possible to represent propositional classical logic in a system where every inference rule is either \emph{atomic} or \emph{linear} \cite{BrunTiu:01:A-Local-:mz}. This fact is what makes atomic flows suitable for studying deep-inference proofs, and why a similar application of atomic flows to sequent calculus proofs would not work.

% Linear rules
An inference rule is \emph{linear} if, for every instance of the rule, there is a one-to-one correspondence between the atom occurrences in the premiss and the atom occurrences in the conclusion. Linear inference rules increases the flexibility of proofs, as other inference rule instances can in most cases trivially be permuted `through' the linear ones.

% Atomic rules
The \emph{atomic} inference rules are rules where only a given atom or its dual occur in every instance. By replacing a generic inference rule with several atomic ones, the flexibility of the proof is increased as the different atomic rules can be permuted independently from each other.

% Consequences of locality
The possibility, which is not present in the sequent calculus \cite{Brun:03:Two-Rest:mn}, of having only linear and atomic inference rules allows representations of proofs which are extremely `malleable'.

\section{Atomic Flows}

% What do we propose instead?
% 
%   Atomic Flows
% 
%     proof invariants
%     structural information only
%     causal dependencies
%     no correctness

Atomic flows were first introduced in \cite{GuglGund:07:Normalis:lr}.

In order to describe notions of normal forms in a syntax independent way, we introduce a proof invariant that we call \emph{atomic flows}. Atomic flows are certain kinds of directed acyclic graphs that capture the structural information of proofs. Intuitively, an atomic flow is obtained from a proof by retaining the causal dependencies between creation, duplication and destruction of atoms and discarding all information about logical connectives, units and linear inference rules.

We can think of atomic flows as composite diagrams that are freely generated from the following six elementary diagrams:

\[
\begin{array}{@{}c@{}c@{}c@{}}
\vcenter{\hbox{\includegraphics{Figures/vertAFAID}}}&
\qquad
\vcenter{\hbox{\includegraphics{Figures/vertAFAWD}}}&
\qquad
\vcenter{\hbox{\includegraphics{Figures/vertAFACD}}}\\
\noalign{\smallskip}
      \mbox{$\aid$ or \emph{interaction}}&
\qquad\mbox{$\awd$ or \emph{weakening}}&
\qquad\mbox{$\acd$ or \emph{contraction}}\\
\noalign{\bigskip}
\vcenter{\hbox{\includegraphics{Figures/vertAFAIU}}}&
\qquad
\vcenter{\hbox{\includegraphics{Figures/vertAFAWU}}}&
\qquad
\vcenter{\hbox{\includegraphics{Figures/vertAFACU}}}\\
\noalign{\smallskip}
      \mbox{$\aiu$ or \emph{cut}}&
\qquad\mbox{$\awu$ or \emph{coweakening}}&
\qquad\mbox{$\acu$ or \emph{cocontraction}}\\
\end{array}\quad.
\]

Atomic flows are associated to derivations as in the examples in Figure~\vref{figure:ExampleAtomicFlows}. It is especially important to note that equations, linear rules and the units $\ttt$ and $\fff$ are lost in the translation from derivations to atomic flows.

Atomic flows can be seen as either specialised Buss flow graphs \cite{Buss:91:The-Unde:uq,Carb:97:Interpol:fk}, or a variation of the kind of proof nets developed in \cite{Stra:05:From-Dee:yb,Stra:09:From-Dee:fr}. The only difference between atomic flows and these proof nets is that the proof nets implement (co)associativity of (co)contraction as well as dinaturality of interaction and cut, while atomic flows do not, and atomic flows do not retain the formula structures of the premiss and conclusion. Despite their similarities, the motivation and use of atomic flows differ from those of proof nets.

\section{Normal Forms}

In both Gentzen-style formalisms and deep-inference formalisms, the cut can be considered horizontal composition of two proofs. We make two observations: 1) deep-inference formalisms are symmetric in the vertical axis, whereas Gentzen-style formalisms are not; and 2) in order for the cut to be admissible from deep-inference derivations the symmetry must be broken, to correspond to the asymmetry of Gentzen-style formalisms. In particular, the cut is only admissible from derivations with no premiss.

These observations prompted us to look for a generalisation of cut elimination that work for all deep-inference derivations. Furthermore, since we are interested in designing new formalisms, we wanted normal forms based on geometric notions which would be as syntax independent as possible.

We defined normal forms based on the causal dependency between structural inference rule instances. Atomic flows contain (by design) exactly the information needed in order to define normal forms in this way, and a derivation is said to be on normal form if its associated atomic flow is.

We call our generalisation of cut elimination \emph{streamlining} and we describe it in terms of atomic flows. Intuitively, if identities and weakenings are considered to be the `creators' of atom occurrences, and cuts and coweakening are the `destroyers' of atom occurrences, then an atomic flow is streamlined if no atom is first created and then destroyed.

Several variations of streamlining are defined, and the outline of their intuition is as follows:

An atomic flows is \emph{weakly streamlined} if it contains no path from an interaction to a cut vertex. Obtaining this normal form turns out to be the most challenging, and several normalisation procedures that produce weakly streamlined derivations are defined. They all have in common that they are global in nature and that they are non-confluent.

A weakly streamlined derivation can be streamlined by applying a local, confluent, linear and strongly normalising normalisation procedure. A natural continuation of this procedure leads to the notion of \emph{super streamlining}, where in addition to being streamlined weakenings and coweakenings are not connected to any other atomic rules.

Finally, the notion of \emph{hyper streamlining} is obtained from super streamlining by permuting contractions under cocontractions, a procedure which is also local, confluent and strongly normalining, but comes with an exponential cost.

The normal forms are expressed in terms of atomic flows in Figure~\vref{figure:AtomicFlowsNormalForm}, on a notation which is described in more detail in the thesis.

%-------------------------------------------------------------------------------
\begin{figure}
\[
%---------------------------------------
\begin{array}{@{}c@{\qquad\qquad}c@{}}
%-------------------
\vcenter{\hbox{\includegraphics{Figures/normWeakForm}}}
\qquad&
%-------------------
\vcenter{\hbox{\includegraphics{Figures/normForm}}}
\\
\noalign{\bigskip}
\mbox{weakly streamlined}&\mbox{streamlined}
\\
\noalign{\bigskip\bigskip}
%-------------------
\vcenter{\hbox{\includegraphics{Figures/normSuperForm}}}
\qquad&
%-------------------
\vcenter{\hbox{\includegraphics{Figures/normHyperForm}}}
\\
\noalign{\bigskip}
\mbox{super streamlined}&\mbox{hyper streamlined}
\end{array}
\]
\caption{Normal forms of atomic flows.}
\label{figure:AtomicFlowsNormalForm}
\end{figure}

\section{Normalisation}

% Atomic flows for normalisation
Atomic flows were designed to describe normal form of proofs. However, it turns out that atomic flows are also a very convenient tool for designing and arguing about normalisation procedures. Two kinds of normalisation procedures are given in the thesis. All the procedures are first presented in terms of atomic flows, before they are lifted to derivations. Properties such as confluence, termination and complexity can be shown in terms of the atomic flows alone.

\subsection{Global procedures}

The \emph{global} procedures, which produce weakly streamlined derivations, work by making several copies of an entire atomic flow, `pruning' each copy and `stitching' them together. It appears that there is great flexibility in the design of the global procedures and there is a lot of room for future investigations, especially with respect to complexity. We show that the global procedures can have less than exponential cost. However, they are all inherently non-confluent.

\subsubsection{The Path Breaker}

These results were presented at LICS 2010 \cite{GuglGundStra::Breaking:uq}.

The Path Breaker is the conceptually simplest procedure to obtain weakly streamlined atomic flows. The intuition behind the procedure is that if a derivation contains a path from an axiom to a cut, it can be rewritten in such a way that the path is broken. The following diagram expresses the intuition in terms of atomic flows:

%--------------------------------
\newcommand{\frpb}{{\mathsf{pb}}}
\[
\vcenter{\hbox{\includegraphics{Figures/redPBRedex}}}
\quad\to_\frpb\quad
\vcenter{\hbox{\includegraphics{Figures/redPBContractum}}}
\quad.
\]
%---------------

In the above diagram on the left there might be a path from the evidenced interaction to the evidenced cut. It can easily be verified by studying the diagram on the right that that it contains no path from the axiom (red) to the cut (green). In the thesis it is show how this transformation can be lifted to derivations, and how when applied repeatedly it will produce a weakly streamlined derivation.

\subsubsection{Quasipolynomial Normalisation}

Parts of these results were presented at LPAR 2010 \cite{BrusGuglGundPari:09:A-Quasip:fk}, and the full results have been submitted to 
Annals of Pure and Applied Logic \cite{BrusGuglGundPari:11:A-Quasip:uq}.


\subsection{Local reductions}

These results were published in Logical Methods in Computer Science \cite{GuglGund:07:Normalis:lr}.

Whereas the global procedures consider the whole atomic flow, the \emph{local} procedures work on one pair of adjacent vertices. These procedures are confluent, but their cost is inherently exponential. The local reducion rules are shown in Figure~\vref{figure:ReductionRules}.

\newcommand{\rwdcd}{{{\mathsf w}{\downarrow}{\hbox{-}}{\mathsf c}{\downarrow}}}
\newcommand{\rwdiu}{{{\mathsf w}{\downarrow}{\hbox{-}}{\mathsf i}{\uparrow  }}}
\newcommand{\rwdwu}{{{\mathsf w}{\downarrow}{\hbox{-}}{\mathsf w}{\uparrow  }}}
\newcommand{\rwdcu}{{{\mathsf w}{\downarrow}{\hbox{-}}{\mathsf c}{\uparrow  }}}
\newcommand{\rcuwu}{{{\mathsf c}{\uparrow  }{\hbox{-}}{\mathsf w}{\uparrow  }}}
\newcommand{\rcdwu}{{{\mathsf c}{\downarrow}{\hbox{-}}{\mathsf w}{\uparrow  }}}
\newcommand{\rcdiu}{{{\mathsf c}{\downarrow}{\hbox{-}}{\mathsf i}{\uparrow  }}}
\newcommand{\rcdcu}{{{\mathsf c}{\downarrow}{\hbox{-}}{\mathsf c}{\uparrow  }}}
\newcommand{\ridwu}{{{\mathsf i}{\downarrow}{\hbox{-}}{\mathsf w}{\uparrow  }}}
\newcommand{\ridcu}{{{\mathsf i}{\downarrow}{\hbox{-}}{\mathsf c}{\uparrow  }}}
%---------------------------------------
\begin{figure}[tbp]
\[
\begin{array}{@{}c@{}c@{}}
%-------------------
\rwdcd\colon\quad
\vcenter{\hbox{\includegraphics{Figures/redAWDACD}}}
&\qquad
%-------------------
\rcuwu\colon\quad
\vcenter{\hbox{\includegraphics{Figures/redACUAWU}}}
\\\noalign{\bigskip}
%-------------------
\rwdiu\colon\quad
\vcenter{\hbox{\includegraphics{Figures/redAWDAIU}}}
&\qquad
%-------------------
\ridwu\colon\quad
\vcenter{\hbox{\includegraphics{Figures/redAIDAWU}}}
\\\noalign{\bigskip}
%-------------------
\multispan2{\hfil$
\rwdwu\colon\quad
\vcenter{\hbox{\includegraphics{Figures/redAWDAWU}}}
$\hfil}\\\noalign{\bigskip}
%-------------------
\rwdcu\colon\quad
\vcenter{\hbox{\includegraphics{Figures/redAWDACU}}}
&\qquad
%-------------------
\rcdwu\colon\quad
\vcenter{\hbox{\includegraphics{Figures/redACDAWU}}}
\\\noalign{\bigskip}
%-------------------
\rcdiu\colon\quad
\vcenter{\hbox{\includegraphics{Figures/redACDAIU}}}
&\qquad
%-------------------
\ridcu\colon\quad
\vcenter{\hbox{\includegraphics{Figures/redAIDACU}}}
\\\noalign{\bigskip}
%-------------------
\multispan2{\hfil$
\rcdcu\colon\quad
\vcenter{\hbox{\includegraphics{Figures/redACDACU}}}
$\hfil}\\
%%-------------------
\end{array}
\]
\caption{Atomic-flow reduction rules.}
\label{figure:ReductionRules}
\end{figure}%

\section{Conclusions}

% Are the properties as expected?
It is expected that propositional classical logic normalisation is inherently exponential and non-confluent, and in fact we observe both these phenomena. However, they are separated into two distinct phases, which can be studied independently.

% What are the benefits of arguing in terms of atomic flows?
The main contribution of this thesis is the use of atomic flows for arguing about normalisation. While it is true that all the results could be reformulated in terms of derivations, this would only serve to obfuscate what is going on.

It should be noted that all the important properties of normalisation can be proven in terms of atomic flows alone. In particular results about complexity, termination, confluence and correctness can be proven without reference to derivations. The challenge in designing normalisation procedures is finding the correct atomic flow transformation, verifying that a transformation can be lifted to derivations is always straight forward.

\end{document}