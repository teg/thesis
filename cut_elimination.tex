\chapter{Cut Elimination}

\TODO{Redo this so that it is in line with how $\Core$ is defined.}

\TODO{Integrate this section with the rest of the thesis. In particular do it in terms of weak cut-elimination, which will make it much more elegant.}

% TODO: big remark or section on philosophy
% TODO: remark on symmetry/confluence

\newcommand{\Assignments}{\mathcal A}
\newcommand{\Sym}{\mathsf{Sym}}

\begin{definition}\label{DefSymmetricProof}
Given distinct and pairwise non-dual atoms, $a_1,\dots,a_n$, define
\begin{itemize}
\item the set $\Assignments_k=\{\{b_1,\dots,b_k\}|b_i\in\{a_i,\bar a_i\}\}$ for $1\leq k\leq n$ and
\item a \emph{symmetric proof of }$\bigvee_{\{b_1,\dots,b_k\}\in\Assignments_k}\vlsbr(b_1.\cdots.b_n)$, denoted $\Sym(a_1,\dots,a_k)$, by induction on $k$:

The base case is
\[
\Sym(a_1)=\vlinf{}{}{\vls[a_1.\bar a_1]}{\ttt}\quad,
\]
and the inductive case is
\[
\newbox\DerCap
\setbox\DerCap=
\hbox{$
\vlderivation
{
 \vlde{}{\{\acu,\med\}}{\bigwedge_{i=1}^{2^{k-1}}[b_k.\bar b_k]}
 {
  \vlin{}{}{\vls[b_k.\bar b_k]}
  {
   \vlhy{\ttt}
  }
 }
}$
}
\newbox\DerCap
\setbox\DerCap=
\hbox{$
\vlderivation
{
 \vlde{}{\{\acu,\med\}}{\bigwedge_{i=1}^{2^{k-1}}\vls[b_k.\bar b_k]}
 {
  \vlin{}{}{\vls[b_k.\bar b_k]}
  {
   \vlhy{\ttt}
  }
 }
}$
}
\Sym(a_1,\dots,a_k)\quad=\quad
\vlderivation
{
 \vlin{=}{}{\bigvee_{\{b_1,\dots,b_k\}\in \Assignments_k}\vlsbr(b_1.\cdots.b_k)}
 {
  \vlde{}{\{\swi\}}{\bigvee_{\{b_1,\dots,b_{k-1}\}\in \Assignments_{n-1}}\vlsbr[(b_1.\cdots.b_{k-1}.b_n).(b_1.\cdots.b_{k-1}.\bar b_k)]}
  {
  \vlpr{\Sym(a_1,\dots,a_{k-1})}{\{\aid,\acu,\swi,\med\}}{\vlsbr(\box\DerCap.\bigvee_{\{b_1,\dots,b_{k-1}\}\in \Assignments_{k-1}}(\vlinf{}{}{\vls(b_1.b_1)}{b_1}.\cdots.\vlinf{}{}{\vls(b_{k-1}.b_{k-1})}{b_{k-1}}))}
  }
 }
}\quad,
\]
for $1 < k \leq n$.
\end{itemize}
\end{definition}

\begin{proposition}
The atomic flow associated with $\Sym(a_1,\dots,a_n)$ is
\[
\atomicflow
{
(-13, 8)*{\afaid{}{}{}{}{}{}};
(-18, 0)*{\affr{8}{8}};
(-18, 0)*{\copy\contrup};
(-16, 2)*{a_1};
( -8, 0)*{\affr{8}{8}};
( -8, 0)*{\copy\contrup};
( -6, 2)*{\bar a_1};
(-18,-6)*{\afvjm{4}};
( -8,-6)*{\afvjm{4}};
%------------
(0,0)*{\cdots};
%------------
(13, 8)*{\afaid{}{}{}{}{}{}};
( 8, 0)*{\affr{8}{8}};
( 8, 0)*{\copy\contrup};
(10, 2)*{a_n};
(18, 0)*{\affr{8}{8}};
(18, 0)*{\copy\contrup};
(20, 2)*{\bar a_n};
( 8,-6)*{\afvjm{4}};
(18,-6)*{\afvjm{4}};
}\quad.
\]
\end{proposition}

\begin{definition}
Given a proof, $\vlproof{\Phi}{}{\beta}$, where the distinct and non-dual atoms $a_1,\dots,a_n$ and their duals are all the atoms that occur, \emph{a symmetric cut-free proof obtained from $\Phi$} is:
\[
\vlderivation
{
 \vlin{(2^n-1)\times\cod}{}{\beta}
 {
  \vlpr{\Sym(a_1,\dots,a_n)}{\{\aid,\acu,\swi,\med\}}{\bigvee_{\{b_1,\dots,b_n\}\in\Assignments_n}\left(\vlder{\Exp(\Phi,b_1,\dots,b_n)}{\SKS\setminus\{\aid,\aiu\}}{\beta}{\vls(b_1.\cdots.b_n)}\right)}
 }
}\quad.
\]
\end{definition}


\begin{proposition}\label{ProUniqueCutFreeFlow}
Given a proof $\Phi$ with atomic flow
\[
A=
\atomicflow
{
(-18, 0)*{\affr{8}{8}};
(-16, 2)*{a_1};
(-21,-6)*{\afvjm4};
%
(-13,8)*{\afaidm{}{}{}{}{}{}};
(-13,-8)*{\afaium{}{}{}{}{}{}};
%
( -8, 0)*{\affr{8}{8}};
( -6, 2)*{\bar a_1};
( -5,-6)*{\afvjm4};
%------------
(0,0)*{\cdots};
%------------
( 8, 0)*{\affr{8}{8}};
(10, 2)*{a_n};
( 5,-6)*{\afvjm4};
%
(13,8)*{\afaidm{}{}{}{}{}{}};
(13,-8)*{\afaium{}{}{}{}{}{}};
%
(18, 0)*{\affr{8}{8}};
(20, 2)*{\bar a_n};
(21,-6)*{\afvjm4};
}\quad,
\]
the atomic flow associated with a symmetric cut-free proof obtained from $\Phi$ is
\[
\atomicflow
{
(-27,18)*{\afaid{}{}{}{}{}{}};
%----
(27,18)*{\afaid{}{}{}{}{}{}};
%--------
(-38,10)*{\affr{20}8};
(-38,10)*{\copy\contrup};
(-16,10)*{\affr{20}8};
(-16,10)*{\copy\contrup};
%----
(16,10)*{\affr{20}8};
(16,10)*{\copy\contrup};
(38,10)*{\affr{20}8};
(38,10)*{\copy\contrup};
%--------
(-44,5)*{\afvjm2};
(-32,5)*{\afvjm2};
(-22,5)*{\afvjm2};
(-10,5)*{\afvjm2};
%----
(10,5)*{\afvjm2};
(22,5)*{\afvjm2};
(33,5)*{\afvjm2};
(44,5)*{\afvjm2};
%--------
(-44,0)*{\affr88};
(-42,2)*{a_1};
(-38,0)*{\vldots};
(-32,0)*{\affr88};
(-30,2)*{a_1};
(-22,0)*{\affr88};
(-20,2)*{\bar a_1};
(-16,0)*{\vldots};
(-10,0)*{\affr88};
(-8,2)*{\bar a_1};
%----
(0,0)*{\vldots};
%----
(10,0)*{\affr88};
(12,2)*{a_n};
(16,0)*{\vldots};
(22,0)*{\affr88};
(24,2)*{a_n};
(32,0)*{\affr88};
(34,2)*{\bar a_n};
(38,0)*{\vldots};
(44,0)*{\affr88};
(46,2)*{\bar a_n};
%--------
(-50,-2)*{\afawdm{}{}{}{}};
(-44,-5)*{\afvjm2};
(-32,-5)*{\afvjm2};
(-22,-5)*{\afvjm2};
(-10,-5)*{\afvjm2};
(-4,-2)*{\afawdm{}{}{}{}};
%----
(4,-2)*{\afawdm{}{}{}{}};
(10,-5)*{\afvjm2};
(22,-5)*{\afvjm2};
(32,-5)*{\afvjm2};
(44,-5)*{\afvjm2};
(50,-2)*{\afawdm{}{}{}{}};
%--------
(-40,-10)*{\affr{24}8};
(-40,-10)*{\copy\contrdown};
(-14,-10)*{\affr{24}8};
(-14,-10)*{\copy\contrdown};
%----
(14,-10)*{\affr{24}8};
(14,-10)*{\copy\contrdown};
(40,-10)*{\affr{24}8};
(40,-10)*{\copy\contrdown};
%--------
(-40,-16)*{\afvjm4};
(-14,-16)*{\afvjm4};
%----
(14,-16)*{\afvjm4};
(40,-16)*{\afvjm4};
}\quad,
\]
where the subflow of the former atomic flow labelled $a_i$ (resp., $\bar a_i$) is isomorphic to the subflow of the latter atomic flow labelled $a_i$ (resp., $\bar a_i$) for every $1\leq i\leq n$ and the subflows labelled with contraction vertices are unique modulo associativity of contraction by Remark~\ref{RemUniquGenContr}.
\end{proposition}
