\TODO{This reduction differs from the other ones, since the atomic flows in the contractum is not yet defined.}

\newcommand{\frmis}{{\mathsf{mis}}}
%---------------------------------------
\begin{definition}\label{DefMulIsoSubRem}
We define the reduction $\to_\frmis$ (where $\frmis$ stands for \emph{multiple isolated subflow}) as follows, for any atomic flows $\phi$ with upper edges $\boldsymbol\epsilon$ and lower edges $\boldsymbol\iota$, and $\psi_1$, $\dots$, $\psi_n$:
\[
\atomicflow{
(-12,  8)*{\afvjdm8{}{\boldsymbol\epsilon}};
(  4, 10)*{\afaidex{}{}{}{}{}{}{20}{4}};
( -6,  5)*{\afvj2};
( -4,  6)*{\cdots};
( 14,  5)*{\afvj2};
(  1,  8)*{\afaidex{}{}{}{}{}{}{6}{4}};
( -7,  0)*{\affr{12}{8}};
( -4,  2)*{\aflabelright{\phi}};
(  4,  0)*{\affr{6}{8}};
(  4,  2)*{\aflabelright{\psi_1}};
(  9,  0)*{\cdots};
( 14,  0)*{\affr{6}{8}};
( 14,  2)*{\aflabelright{\psi_n}};
(  1, -8)*{\afaiuex{}{}{}{}{}{}{6}{4}};
( -6, -5)*{\afvj2};
( -4, -6)*{\cdots};
( 14, -5)*{\afvj2};
(  4,-10)*{\afaiuex{}{}{}{}{}{}{20}{4}};
(-12, -8)*{\afvjum8{}{\boldsymbol\iota}};
}
\quad\to_\frmis\quad
\atomicflow{
(-12, 54)*{\afvjdm4{}{\boldsymbol\epsilon}};
(-12, 48)*{\copy\contrup};
(-12, 48)*{\affr{16}{8}};
( -5, 43)*{\afvjm2};
( -7, 33)*{\afcjlm{4}{22}};
(-12, 42)*{\cdots};
(-10,  2)*{\afcjlm{10}{32}};
(-15, 31)*{\afvjm{26}};
(-12,-13)*{\afcjlm{14}{42}};
(-19, 26)*{\afvjm{36}};
%--
( 5, 46)*{\afawdm{}{}{}{}};
( 0, 38)*{\affr{12}{8}};
( 3, 40)*{\aflabelright{\phi}};
( 5, 33)*{\afvjm2};
( 9, 28)*{\affr{10}{8}};
(11, 30)*{\aflabelright{\gamma_0}};
( 5, 23)*{\afvjm2};
(11, 23)*{\afvjm2};
( 0, 18)*{\affr{12}{8}};
( 3, 20)*{\aflabelright{\phi}};
( 5, 13)*{\afvjm2};
(11, 13)*{\afvjm2};
(11, 18)*{\affr{6}{8}};
(11, 20)*{\aflabelright{\delta_1}};
( 9,  8)*{\affr{10}{8}};
(11, 10)*{\aflabelright{\gamma_1}};
%-
( 9,  1)*{\vdots};
%-
( 9, -8)*{\affr{10}{8}};
(10, -6)*{\aflabelright{\gamma_{n-1}}};
( 5,-13)*{\afvjm2};
(11,-13)*{\afvjm2};
( 0,-18)*{\affr{12}{8}};
( 3,-16)*{\aflabelright{\phi}};
(11,-18)*{\affr{6}{8}};
(11,-16)*{\aflabelright{\delta_n}};
( 5,-23)*{\afvjm2};
(11,-23)*{\afvjm2};
( 9,-28)*{\affr{10}{8}};
(11,-26)*{\aflabelright{\gamma_n}};
( 5,-33)*{\afvjm2};
( 0,-38)*{\affr{12}{8}};
( 3,-36)*{\aflabelright{\phi}};
( 5,-46)*{\afawum{}{}{}{}};
%--
(-12, 13)*{\afcjrm{14}{42}};
(-19,-26)*{\afvjm{36}};
(-10, -2)*{\afcjrm{10}{32}};
(-15,-31)*{\afvjm{26}};
(-12,-42)*{\cdots};
( -7,-33)*{\afcjrm{4}{22}};
( -5,-43)*{\afvjm2};
(-12,-48)*{\affr{16}{8}};
(-12,-48)*{\copy\contrdown};
(-12,-54)*{\afvjum4{}{\boldsymbol\iota}};
}\quad,
\]
for  some flows $\gamma_0$, $\dots$, $\gamma_n$, $\delta_1$, $\dots$, $\delta_n$, which do not contain interaction or cut vertices, where, for each $1\le i\le n$, the subflow $\psi_i$ is connected and does not contain any interaction or cut vertices.
\end{definition}

\TODO{Redo this theorem:}

\begin{theorem}
Given a set of formulae $\gamma_{i,k}$ such that $\dots$, let $\gamma_k$ be the atomic flow of $\Gamma_k$ and $\delta_k$ be the atomic flow consisting of, for every $1\le i\le n$, $|\gamma_{i,k}|$ copies of $\psi_i$, then $\to_\frmis$ is sound.
\end{theorem}


\TODO{Change this introduction}

\newcommand{\Gammasf}{\mathsf\Gamma}
We present here the main construction of this paper, \emph{i.e.}, a class of derivations $\Gammasf$ that only depend on a given set of atoms and that allow us to normalise any proof containing those atoms. The complexity of the $\Gammasf$ derivations dominates the complexity of the normal proof, and is due to the complexity of certain `threshold formulae', on which the $\Gammasf$ derivations are based. The $\Gammasf$ derivations are constructed in Definition~\ref{DefThrDer}; this directly leads to Theorem~\ref{TheoThrDer}, which states a crucial property of the $\Gammasf$ derivations and which is the main result of this section.

\subsubsection{Threshold Formulae}

\TODO{Define $\lfloor x\rfloor$.}

\TODO{Define $a^\phi$.}

%In the following, $\lfloor x\rfloor$ denotes the maximum integer $n$ such that $n\le x$.

There are several ways of encoding threshold functions into formulae, and the problem is to find, among them, an encoding that allows us to obtain Theorem~\ref{TheoThrDer}. Efficiently obtaining the property stated in Theorem~\ref{TheoThrDer} crucially depends also on the proof system we adopt.

\TODO{Lemma on substituting into isolated subflows}

\begin{remark}
Consider the derivation
\[
\vlder{\Phi}{}
{
 \vls[\beta\;.\;\vlinf{}{}{\fff}{\vls(a_1.\bar a_1^{\psi_1})}\;.\;\vlinf{}{}{\fff}{\vls(a_n.\bar a_n^{\psi_n})}]
}
{
 \vls(\vlinf{}{}{\vls[a_1.\bar a_1^{\psi_1}]}{\ttt}\;.\;\vlinf{}{}{\vls[a_n.\bar a_n^{\psi_n}]}{\ttt}\;.\;\alpha)
}\quad,
\]
such that, for $1\le i\le n$, $\phi_i$ is an isolated subflow. For every $k\ge0$, consider some class of formulae $\gamma_{k,1}$, $\dots$, $\gamma_{k,n}$, then, by Lemma~\ref{TODO}, we can construct the derivation
\[
\Phi_k\quad=\quad
\vlder{\Phi\{\bar a_1^{\psi_1}/\gamma_{k,1},\dots,\bar a_n^{\psi_n}/\gamma_{k,n}\}}{}
{
 \vls
 [
  \beta.(a_1.\gamma_{k,1}).\cdots.(a_n.\gamma_{k,n})
 ]
}
{
 \vls
 (
  [a_1.\gamma_{k,1}].\cdots.[a_n.\gamma_{k,n}].\alpha
 )
}.
\]

If
\begin{itemize}
 \item $\gamma_{0,1}=\cdots=\gamma_{0,n}=\ttt$,
 \item there is an $l\ge0$ such that $\gamma_{l,1}=\cdots=\gamma_{l,n}=\fff$ and,
 \item for every $1\le k\le l$, there exists a derivation
\[
\Gammasf_k\quad=\quad
\vlder{}{\SKS\setminus\{\aid,\aiu\}}
{
 \vls([a_1.\gamma_{k,1}].\cdots.[a_n.\gamma_{k,n}])
}
{
 \vls[(a_1.\gamma_{k-1,1}).\cdots.(a_n.\gamma_{k-1,n})]
}\quad,
\]
\end{itemize}
then we can compose $\Phi_0$, $\dots$, $\Phi_l$, $\Gammasf_1$, $\Gammasf_l$ to obtain
\[
\vlder{}{}
{
 \vls[\beta\;.\;(\vlinf{\awu}{}{\ttt}{a_1}\;.\;\fff)\;.\;\cdots\;.\;(\vlinf{\awu}{}{\ttt}{a_n}\;.\;\fff)]
}
{
 \vls([\vlinf{\awd}{}{a_1}{\fff}\;.\;\ttt]\;.\;\cdots\;.\;[\vlinf{\awd}{}{a_n}{\fff}\;.\;\ttt]\;.\;\alpha)
}
\]
with the desired atomic flow.
\end{remark}

\begin{remark}
Threshold formulae realise boolean threshold functions, which are defined as boolean functions that are true if and only if at least $k$ of $n$ inputs are true (see \cite{Wege:87:The-Comp:vn} for a thorough reference on threshold functions). 

If we let $\gamma_{k,i}$ be a threshold formula which is true if and only if at least $k$ of the atoms $a_1$, $\dots$, $a_{i-1}$, $a_{i+1}$, $\dots$, $a_n$ are true, then we can trivially observe the following
\[
\gamma_{0,1}=\cdots=\gamma_{0,n}=\ttt
\quad\mbox{and}\quad
\gamma_{n,1}=\cdots=\gamma_{n,n}=\fff
\]

Furthermore, the two formulae
\[
\vls[(a_1.\gamma_{k-1,1}).\cdots.(a_n.\gamma_{k-1,n})]
\quad\mbox{and}\quad
\vls([a_1.\gamma_{k,1}].\cdots.[a_n.\gamma_{k,n}])
\]
are equivalent since they are true if and only if at least $k$ of the atoms $a_1$, $\dots$, $a_n$ are true. Hence, since system $\SKS$ is implicationally complete, the derivation
\[
\vlder{}{\SKS}
{
 \vls([a_1.\gamma_{k,1}].\cdots.[a_n.\gamma_{k,n}])
}
{
 \vls[(a_1.\gamma_{k-1,1}).\cdots.(a_n.\gamma_{k-1,n})]
}
\]
exists.

The challenge lies in finding a class of threshold formulae such that the above derivation is as small as possible and does not contain any interaction or cut instances.
\end{remark}

The following class of threshold formulae, which we found to work for system $\SKS$, is a simplification of the one adopted in \cite{AtseGalePudl:02:Monotone:yu}.

\renewcommand{\th}[2]{\mathop{\thetaup_{#1}^{#2}}}
%-------------------------------------------------------------------------------
\begin{definition}
Consider $n>0$, distinct atoms $a_1$, \dots, $a_n$, and let $p=\lfloor n/2\rfloor$ and $q=n-p$; for $k\ge0$, we define the \emph{threshold formulae\/} $\th kn\avec1n$ as follows:
\begin{itemize}
%---------------------------------------
\item for any $n>0$ let $\th0n\avec1n\equiv\ttt$;
%---------------------------------------
\item for any $n>0$ and $k>n$ let $\th kn\avec1n\equiv\fff$;
%---------------------------------------
\item $\th11(a_1)\equiv a_1$;
%---------------------------------------
\item for any $n>1$ and $0<k\le n$ let
$\th kn\avec1n\equiv\bigvee_{\begin{subarray}{l}i+j=k      \\ 
                                                0\le i\le p\\ 
                                                0\le j\le q
                             \end{subarray}}
\vlsbr(\th ip\avec1p.\th jq\avec{p+1}n)$.
%---------------------------------------
\end{itemize}
\end{definition}

See, in Figure~\ref{FigThrEx}, some examples of threshold formulae.

\TODO{Check if this still holds, if it does, find a new explanation:}

The only reason why we require atoms to be distinct in threshold formulae is to avoid certain technical problems with substitutions in the definition of cut-free form, later on. However, there is no substantial difficulty in relaxing this definition to any set of atoms.

%-------------------------------------------------------------------------------
\begin{figure}
\vlsmallbrackets
\begin{eqnarray*}
%---------------------------------------
\th02(a,b)&\equiv&\ttt
\quad,\\
\noalign{\medskip}
%---------------------------------------
\th12(a,b)&\equiv&\vls[({\vlnos\th11(a)}.{\vlnos\th01(b)}).
                       ({\vlnos\th01(a)}.{\vlnos\th11(b)})]
           \equiv     [(a.\ttt).(\ttt.b)]\\
          &=     &\vls [a      .      b ]
\quad,\\
\noalign{\medskip}
%---------------------------------------
\th22(a,b)&\equiv&\vls({\vlnos\th11(a)}.{\vlnos\th11(b)})\\
          &\equiv&\vls(a.b)
\quad,\\
\noalign{\medskip}
%---------------------------------------
\th03(a,b,c)&\equiv&\ttt
\quad,\\
\noalign{\medskip}
%---------------------------------------
\th13(a,b,c)&\equiv&\vls[({\vlnos\th11(a)}.{\vlnos\th02(b,c)}).
                         ({\vlnos\th01(a)}.{\vlnos\th12(b,c)})]
             \equiv     [(a.\ttt).(\ttt.[(b.\ttt).(\ttt.c)])]\\
            &=     &\vls[a.b.c]
\quad,\\
\noalign{\medskip}
%---------------------------------------
\th23(a,b,c)&\equiv&\vls[({\vlnos\th11(a)}.{\vlnos\th12(b,c)}).
                    ({\vlnos\th01(a)}.{\vlnos\th22(b,c)})]\\
            &=     &\vls[(a.[b.c]).(b.c)]
\quad,\\
\noalign{\medskip}
%---------------------------------------
\th33(a,b,c)&\equiv&\vls({\vlnos\th11(a)}.{\vlnos\th22(b,c)})
             \equiv     [(a.(b.c))]\\
            &=     &\vls(a.b.c)
\quad,\\
\noalign{\medskip}
%---------------------------------------
\th05(a,b,c,d,e)&\equiv&\ttt
\quad,\\
\noalign{\medskip}
%---------------------------------------
\th15(a,b,c,d,e)&\equiv&\vls[({\vlnos\th12(a,b)}.{\vlnos\th03(c,d,e)}).
                             ({\vlnos\th02(a,b)}.{\vlnos\th13(c,d,e)})]\\
                &=     &\vls[a.b.c.d.e]
\quad,\\
\noalign{\medskip}
%---------------------------------------
\th25(a,b,c,d,e)&\equiv&\vls[({\vlnos\th22(a,b)}.{\vlnos\th03(c,d,e)}).
                             ({\vlnos\th12(a,b)}.{\vlnos\th13(c,d,e)}).
                             ({\vlnos\th02(a,b)}.{\vlnos\th23(c,d,e)})]\\
                &=     &\vls[(a.b                                    ).
                             ([a.b]             .[c.d.e]             ).
                                                 (c.[d.e]).(d.e)      ]
\quad,\\
\noalign{\medskip}
%---------------------------------------
\th35(a,b,c,d,e)&\equiv&\vls[({\vlnos\th22(a,b)}.{\vlnos\th13(c,d,e)}).
                             ({\vlnos\th12(a,b)}.{\vlnos\th23(c,d,e)}).
                             ({\vlnos\th02(a,b)}.{\vlnos\th33(c,d,e)})]\\
                &=     &\vls[(a.b               .[c.d.e]             ).
                             ([a.b]             .[(c.[d.e]).(d.e)]   ).
                                                 (c.d.e)              ]
\quad,\\
\noalign{\medskip}
%---------------------------------------
\th45(a,b,c,d,e)&\equiv&\vls[({\vlnos\th22(a,b)}.{\vlnos\th23(c,d,e)}).
                             ({\vlnos\th12(a,b)}.{\vlnos\th33(c,d,e)})]\\
                &=     &\vls[(a.b               .[(c.[d.e]).(d.e)]   ).
                             ([a.b]             .c.d.e               )]
\quad,\\
\noalign{\medskip}
%---------------------------------------
\th55(a,b,c,d,e)&\equiv&\vls({\vlnos\th22(a,b)}.{\vlnos\th33(c,d,e)})\\
                &=     &\vls(a.b.c.d.e)
\quad,\\
\noalign{\medskip}
%---------------------------------------
\th65(a,b,c,d,e)&\equiv&\fff
\quad.
\end{eqnarray*}
\caption{Examples of threshold formulae.}
\label{FigThrEx}
\end{figure}

The formulae for threshold functions adopted in \cite{AtseGalePudl:02:Monotone:yu} correspond, for each choice of $k$ and $n$, to $\bigvee_{i\ge k}\th in\avec1n$. We presume that \cite{AtseGalePudl:02:Monotone:yu} employs these more complicated formulae because the formalism adopted there, the sequent calculus, is less flexible than deep inference, requiring more information in threshold formulae in order to construct suitable derivations.

%-------------------------------------------------------------------------------
\begin{remark}
For $n>0$, we have $\th1n\avec1n=\vls[a_1.\vldots.a_n]$ and $\th nn\avec1n=\vls(a_1.\vldots.a_n)$.
\end{remark}

The size of the threshold formulae dominates the cost of the normalisation procedure, so, we evaluate their size. We leave as an exercise the proof of the following proposition.

%-------------------------------------------------------------------------------
\begin{proposition}\label{PropQuasAux}
For any $n>0$ and $k\ge0$, $\size{\th kn\avec1n}\le\size{\th{\lfloor n/2\rfloor+1}n\avec1n}$.
\end{proposition}

%-------------------------------------------------------------------------------
\begin{lemma}\label{LemmaQuas}
The size of\/ $\th{\lfloor n/2\rfloor+1}n\avec1n$ is $n^{\Ord{\log n}}$.
\end{lemma}

%-------------------------------------------------------------------------------
\begin{proof}
Observe that $\size{\th kn\avec1n}\le\size{\th k{n+1}\avec1{n+1}}$. Let $p=\lfloor n/2\rfloor$ and $q=n-p$ and consider:
\begin{equation}\label{PropQuasIneq}
\begin{split}
\size{\th{p+1}n\avec1n}
&=\textstyle\sum_{\begin{subarray}{l}i+j=p+1    \\
                                     0\le i\le p\\
                                     0\le j\le q
                  \end{subarray}}
  \left(\size{\th ip\avec1p}+
        \size{\th jq\avec{p+1}n}\right)             \\
&\le\textstyle\sum_{\begin{subarray}{l}i+j=p+1\\
                                       0\le i,j\le q
                    \end{subarray}}
  \left(\size{\th iq\avec1q}+
        \size{\th jq\avec1q}\right)                 \\
&\le2(q+1)
  \size{\th{\lfloor q/2\rfloor+1}q\avec1q}\quad,
\end{split}
\end{equation}
where we use Proposition~\ref{PropQuasAux}. We show that, for $h=2/(\log3-\log2)$ and for any $n>0$, we have $\size{\th{\lfloor n/2\rfloor+1}n\avec1n}\le n^{h\log n}$. We reason by induction on $n$; the case $n=1$ trivially holds. By the inequality~\eqref{PropQuasIneq}, and for $n>1$, we have
\begin{equation*}
\begin{split}
\size{\th{\lfloor n/2\rfloor+1}n\avec1n}
&\le2(n-\lfloor n/2\rfloor+1)
     (n-\lfloor n/2\rfloor)^{h\log(n-\lfloor n/2\rfloor)}       \\
&\le n^2n^{h\log(2n/3)}=n^{h\log n-h(\log3-\log2)+2}=n^{h\log n}
\quad.
\end{split}
\end{equation*}
\end{proof}

%-------------------------------------------------------------------------------
\begin{theorem}\label{TheoQuas}
For any $k\ge0$ the size of\/ $\th kn\avec1n$ is $n^{\Ord{\log n}}$.
\end{theorem}

%-------------------------------------------------------------------------------
\begin{proof}
It immediately follows from Proposition~\ref{PropQuasAux} and Lemma~\ref{LemmaQuas}.
\end{proof}

\subsubsection{Glue derivations}

\TODO{Define generic weakening.}

%-------------------------------------------------------------------------------
\begin{remark}
Given $n>1$, let $p=\lfloor n/2\rfloor$ and $q=n-p$. For $0\le k\le q$ and $1\le l\le p$, the following derivation is well defined:
\[
\vlinf{\gwu}
      {}
      {\fff}
      {\vls({\vlnos(\th pp\avec1p)}\{a_l/\fff\}.\th kq\avec{p+1}n)}
=
\vls(
\vlinf{\gwu}
      {}
      {\vls(\ttt)}
      {\vls(a_1.\cdots.a_{l-1}.a_{l+1}.\cdots.a_p.\th kq\avec{p+1}n)}
.\fff)
\quad.
\]
Analogously, for $0\le k\le p$ and $p+1\le l\le n$, we can define the following derivation:
\[
\vlinf{\gwu}
      {}
      {\fff}
      {\vls(\th kp\avec1p.{\vlnos(\th qq\avec{p+1}n)}\{a_l/\fff\})}
=
\vls(
\vlinf{\gwu}
      {}
      {\vls(\ttt)}
      {\vls(\th kp\avec1p.a_{p+1}.\cdots.a_{l-1}.a_{l+1}.\cdots.a_n)}
.\fff)
\quad.
\]
Both classes of derivations are used in Definition~\ref{DefThrDer}.
\end{remark}

\newcommand{\Uth}[3]{\mathop{\mathsf\Upsilon_{#1,#2}^{#3}}}
\newcommand{\Dth}[3]{\mathop{\mathsf\Delta_{#1,#2}^{#3}}}
\newcommand{\Gth}[3]{\mathop{\Gammasf_{#1,#2}^{#3}}}
%-------------------------------------------------------------------------------
\begin{definition}\label{DefThrDer}
Consider $n>0$, distinct atoms $a_1$, \dots, $a_n$, and let $p=\lfloor n/2\rfloor$ and $q=n-p$.
\begin{itemize}
%---------------------------------------
%---------------------------------------
\item
For $n>1$ and $1\le l\le n$, we define the derivations $\Uth kln\avec1n$ and $\Dth kln\avec1n$ as follows:
\[
\Uth kln\avec1n=\begin{cases}
\vlinf{\gwu}
      {}
      {\fff}
      {\vls({\vlnos(\th pp\avec1p)}\{a_l/\fff\}.\th{k-p}q\avec{p+1}n)}
             &\text{if $p\le k\le n$ and $l\le p$}\\
\noalign{\medskip}
\vlinf{\gwu}
      {}
      {\fff}
      {\vls(\th{k-q}p\avec1p.{\vlnos(\th qq\avec{p+1}n)}\{a_l/\fff\})}
             &\text{if $q\le k\le n$ and $p<l$}\\
\noalign{\medskip}
\fff         &\text{otherwise}
              \end{cases}
\]
and
\[
\Dth kln\avec1n=\begin{cases}
\vlinf{\gwd}
      {}
      {\th kq\avec{p+1}n}
      {\fff}
             &\text{if $0<k\le q$ and $l\le p$}\\
\noalign{\medskip}
\vlinf{\gwd}
      {}
      {\th kp\avec1p}
      {\fff}
             &\text{if $0<k\le p$ and $p<l$}\\
\noalign{\medskip}
\fff         &\text{otherwise}
              \end{cases}\quad.
\]
%---------------------------------------
%---------------------------------------
\item
For $k\ge0$ and $1\le l\le n$, we define the derivations $\vlsmash{\Gth kln\avec1n}$, recursively on $n$, as follows:
\begin{itemize}
%---------------------------------------
\item $\Gth 011(a_1)=\ttt$;
%---------------------------------------
\item for $k>0$, $\Gth k11(a_1)=\fff$;
%---------------------------------------
\item for $k>n$, $\Gth kln\avec1n=\fff$;
%---------------------------------------
\item for $n>1$ and $k\le n$, let
\[
\Gth kln\avec1n=\begin{cases}
%---------------------------------------
\vls[
\bigvee_{\begin{subarray}{l}i+j=k      \\ 
                            0\le i<p   \\ 
                            0\le j\le q
         \end{subarray}}(
\Gth ilp\avec1p.
\th jq\avec{p+1}n).
\Uth kln\avec1n.\Dth{k+1}ln\avec1n]
&\text{if $l\le p$}\\
\noalign{\medskip}
%---------------------------------------
\vls[
\bigvee_{\begin{subarray}{l}i+j=k      \\
                            0\le i\le p\\ 
                            0\le j<q
         \end{subarray}}(
\th ip\avec1p.
\Gth j{l-p}q\avec{p+1}n).
\Uth kln\avec1n.\Dth{k+1}ln\avec1n]
&\text{if $p<l$}
\end{cases}
\quad.
\]
%---------------------------------------
\end{itemize}
%---------------------------------------
%---------------------------------------
\end{itemize}
\end{definition}


%-------------------------------------------------------------------------------
\begin{example}
See, in Figure~\ref{FigPThEx}, some examples of derivations $\vlsmash{\Gth kln\avec1n}$. Note that, for clarity, we removed all instances of the trivial derivations $\Uth112\avec12=\Uth122\avec12=\Uth113\avec13=\vldownsmash{\vlinf\gwu{}\fff\fff}$. We can do so because these derivation instances appear as disjuncts.
\end{example}

%-------------------------------------------------------------------------------
\begin{figure}
\begin{eqnarray*}
%---------------------------------------
\Gth 015\avecletter&=&
\vls [\ttt.\vlderivation{
\vlin{}{}{b}{
\vlhy{\vls \fff}
}}
.\vlderivation{
\vlin{}{}{\vls [c.d.e]}{
\vlhy{\vls \fff}
}}
]\quad,\\
\noalign{\smallskip}
%---------------------------------------
\Gth 115\avecletter&=&
\vls [b.([\ttt.\vlderivation{
\vlin{}{}{b}{
\vlhy{\vls \fff}
}}
].[c.d.e]).\vlderivation{
\vlin{}{}{\vls [(c.[d.e]).(d.e)]}{
\vlhy{\vls \fff}
}}
]\quad,\\
\noalign{\smallskip}
%---------------------------------------
\Gth 215\avecletter&=&
\vls [(b.[c.d.e]).([\ttt.\vlderivation{
\vlin{}{}{b}{
\vlhy{\vls \fff}
}}
].[(c.[d.e]).(d.e)]).\vlderivation{
\vlin{}{}{\vls \fff}{
\vlhy{\vls (\fff.b)}
}}
.\vlderivation{
\vlin{}{}{\vls (c.d.e)}{
\vlhy{\vls \fff}
}}
]\quad,\\
\noalign{\smallskip}
%---------------------------------------
\Gth 315\avecletter&=&
\vls [(b.[(c.[d.e]).(d.e)]).([\ttt.\vlderivation{
\vlin{}{}{b}{
\vlhy{\vls \fff}
}}
].c.d.e).\vlderivation{
\vlin{}{}{\vls \fff}{
\vlhy{\vls (\fff.b.[c.d.e])}
}}
]\quad,\\
\noalign{\smallskip}
%---------------------------------------
\Gth 415\avecletter&=&
\vls [(b.c.d.e).\vlderivation{
\vlin{}{}{\vls \fff}{
\vlhy{\vls (\fff.b.[(c.[d.e]).(d.e)])}
}}
]\quad,\\
\noalign{\smallskip}
%---------------------------------------
\Gth 515\avecletter&=&
\vlderivation{
\vlin{}{}{\vls \fff}{
\vlhy{\vls (\fff.b.c.d.e)}
}}
\quad,\\
\noalign{\smallskip}
%---------------------------------------
\Gth 035\avecletter&=&
\vls [\ttt.\vlderivation{
\vlin{}{}{\vls [d.e]}{
\vlhy{\vls \fff}
}}
.\vlderivation{
\vlin{}{}{\vls [a.b]}{
\vlhy{\vls \fff}
}}
]\quad,\\
\noalign{\smallskip}
%---------------------------------------
\Gth 135\avecletter&=&
\vls [([a.b].[\ttt.\vlderivation{
\vlin{}{}{\vls [d.e]}{
\vlhy{\vls \fff}
}}
]).d.e.\vlderivation{
\vlin{}{}{\vls (d.e)}{
\vlhy{\vls \fff}
}}
.\vlderivation{
\vlin{}{}{\vls (a.b)}{
\vlhy{\vls \fff}
}}
]\quad,\\
\noalign{\smallskip}
%---------------------------------------
\Gth 235\avecletter&=&
\vls [(a.b.[\ttt.\vlderivation{
\vlin{}{}{\vls [d.e]}{
\vlhy{\vls \fff}
}}
]).([a.b].[d.e.\vlderivation{
\vlin{}{}{\vls (d.e)}{
\vlhy{\vls \fff}
}}
]).(d.e).\vlderivation{
\vlin{}{}{\vls \fff}{
\vlhy{\vls (\fff.[d.e])}
}}
]\quad,\\
\noalign{\smallskip}
%---------------------------------------
\Gth 335\avecletter&=&
\vls [(a.b.[d.e.\vlderivation{
\vlin{}{}{\vls (d.e)}{
\vlhy{\vls \fff}
}}
]).([a.b].[(d.e).\vlderivation{
\vlin{}{}{\vls \fff}{
\vlhy{\vls (\fff.[d.e])}
}}
]).\vlderivation{
\vlin{}{}{\vls \fff}{
\vlhy{\vls (\fff.d.e)}
}}
]\quad,\\
\noalign{\smallskip}
%---------------------------------------
\Gth 435\avecletter&=&
\vls [(a.b.[(d.e).\vlderivation{
\vlin{}{}{\vls \fff}{
\vlhy{\vls (\fff.[d.e])}
}}
]).\vlderivation{
\vlin{}{}{\vls \fff}{
\vlhy{\vls ([a.b].\fff.d.e)}
}}
]\quad,\\
\noalign{\smallskip}
%---------------------------------------
\Gth 535\avecletter&=&
\vlderivation{
\vlin{}{}{\vls \fff}{
\vlhy{\vls (a.b.\fff.d.e)}
}}\quad,\\
\noalign{\smallskip}
%---------------------------------------
\Gth 055\avecletter&=&
\vls [\ttt.\vlderivation{
\vlin{}{}{d}{
\vlhy{\vls \fff}
}}
.\vlderivation{
\vlin{}{}{c}{
\vlhy{\vls \fff}
}}
.\vlderivation{
\vlin{}{}{\vls [a.b]}{
\vlhy{\vls \fff}
}}
]\quad,\\
\noalign{\smallskip}
%---------------------------------------
\Gth 155\avecletter&=&
\vls [([a.b].[\ttt.\vlderivation{
\vlin{}{}{d}{
\vlhy{\vls \fff}
}}
.\vlderivation{
\vlin{}{}{c}{
\vlhy{\vls \fff}
}}
]).(c.[\ttt.\vlderivation{
\vlin{}{}{d}{
\vlhy{\vls \fff}
}}
]).d.\vlderivation{
\vlin{}{}{\vls (a.b)}{
\vlhy{\vls \fff}
}}
]\quad,\\
\noalign{\smallskip}
%---------------------------------------
\Gth 255\avecletter&=&
\vls [(a.b.[\ttt.\vlderivation{
\vlin{}{}{d}{
\vlhy{\vls \fff}
}}
.\vlderivation{
\vlin{}{}{c}{
\vlhy{\vls \fff}
}}
]).([a.b].[(c.[\ttt.\vlderivation{
\vlin{}{}{d}{
\vlhy{\vls \fff}
}}
]).d]).(c.d).\vlderivation{
\vlin{}{}{\vls \fff}{
\vlhy{\vls (d.\fff)}
}}
]\quad,\\
\noalign{\smallskip}
%---------------------------------------
\Gth 355\avecletter&=&
\vls [(a.b.[(c.[\ttt.\vlderivation{
\vlin{}{}{d}{
\vlhy{\vls \fff}
}}
]).d]).([a.b].[(c.d).\vlderivation{
\vlin{}{}{\vls \fff}{
\vlhy{\vls (d.\fff)}
}}
]).\vlderivation{
\vlin{}{}{\vls \fff}{
\vlhy{\vls (c.d.\fff)}
}}
]\quad,\\
\noalign{\smallskip}
%---------------------------------------
\Gth 455\avecletter&=&
\vls [(a.b.[(c.d).\vlderivation{
\vlin{}{}{\vls \fff}{
\vlhy{\vls (d.\fff)}
}}
]).\vlderivation{
\vlin{}{}{\vls \fff}{
\vlhy{\vls ([a.b].c.d.\fff)}
}}
]\quad,\\
\noalign{\smallskip}
%---------------------------------------
\Gth 555\avecletter&=&
\vlderivation{
\vlin{}{}{\vls \fff}{
\vlhy{\vls (a.b.c.d.\fff)}
}}\quad.
\end{eqnarray*}
\caption{Examples of $\Gth kl5\avecletter$, where $\avecletter=(a,b,c,d,e)$.}
\label{FigPThEx}
\end{figure}


%-------------------------------------------------------------------------------
\begin{theorem}\label{TheoThrDer}
For any $n>0$, $k\ge0$ and\/ $1\le l\le n$, the derivation\/ $\vlsmash{\Gth kln\avec1n}$ has shape
\[
\vlder{}{\{\awd,\awu\}}{(\th{k+1}n\avec1n)\{a_l/\ttt\}}
                       {(\th kn\avec1n)\{a_l/\fff\}}
\quad,
\]
and\/ $\size{\Gth kln\avec1n}$ is $n^{\Ord{\log n}}$.
\end{theorem}

%-------------------------------------------------------------------------------
\begin{proof}
The shape of $\Gth kln\avec1n$ can be verified by inspecting Definition~\ref{DefThrDer}. For example, this is the case when $n>1$ and $l\le p\le k<q$, where $p=\lfloor n/2\rfloor$ and $q=n-p$:
\vlstore{\noalign{\medskip}
\vls[
\textstyle\bigvee_{\begin{subarray}{l}i+j=k      \\
                                      0\le i<p   \\
                                      0\le j\le q
                   \end{subarray}}(
\vlder{\Gth ilp\avec1p}
      {}
      {(\th{i+1}p\avec1p)\{a_l/\ttt\}}
      {(\th ip\avec1p)\{a_l/\fff\}}
.
\th jq\avec{p+1}n)
.
\vlinf{\gwu}
      {}
      {\fff}
      {\vls({\vlnos(\th pp\avec1p)}\{a_l/\fff\}.\th{k-p}q\avec{p+1}n)}
.
\vlinf{\gwd}
      {}
      {\th{k+1}q\avec{p+1}n}
      {\fff}
]}
\begin{multline*}
\vlder{\Gth kln\avec1n}
      {}
      {(\th{k+1}n\avec1n)\{a_l/\ttt\}}
      {(\th kn\avec1n)\{a_l/\fff\}}
={}\\
\vlread
\quad.
\end{multline*}
(Remember that
\[
\th kn\avec1n\equiv\bigvee_{\begin{subarray}{l}
                            i+j=k\\ 
                            0\le i\le p\\ 
                            0\le j\le q
                            \end{subarray}}
                   \vlsbr(\th ip\avec1p.\th jq\avec{p+1}n)
\]
and $\th0p\avec1p\equiv\ttt$.) General (co)weak\-en\-ing rule instances can be replaced by atomic ones because of Proposition~\ref{PropGenAtPol}. The size bound on $\Gth kln\avec1n$ follows from Proposition~\ref{PropGenAtPol} and Theorem~\ref{TheoQuas}.
\end{proof}

\TODO{Define operator for the following definition}

\begin{definition}\label{DefThrDer2}
Consider $n>0$, distinct atoms $a_1$, \dots, $a_n$. For $k\ge0$, we define the derivations $\vlsmash{\Gth k{}n\avec1n}$ as follows:
\[
\Gth k{}n\avec1n\quad=\quad
\vlderivation
{
 \vlin{n\cdot\cou}{}
 {
  \vls
  (
   \vlder{}{}
   {
    \vls[a_1.(\th {k+1}n\avec1n)\{a_1/\fff\}]
   }
   {
    \th {k+1}n\avec1n
   }
  \;\;.\;\;
   \vlder{}{}
   {
    \vls[a_n.(\th {k+1}n\avec1n)\{a_n/\fff\}]
   }
   {
    \th {k+1}n\avec1n
   }
  )
 }
 {
  \vlin{n\cdot\cod}{}
  {
   \th {k+1}n\avec1n
  }
  {
   \vlhy
   {
    \vls
    [
     \vlder{}{}
     {
      \th {k+1}n\avec1n
     }
     {
      \vlsbr
      (
       a_1
      .
       \vlder{\Gth kin\avec1n}{}
       {
        (\th {k+1}n\avec1n)\{a_1/\ttt\}
       }
       {
        (\th kn\avec1n)\{a_1/\fff\}
       }
      )
     }
    \;.\;
     \vlder{}{}
     {
      \th {k+1}n\avec1n
     }
     {
      \vlsbr
      (
       a_n
      .
       \vlder{\Gth knn\avec1n}{}
       {
        (\th {k+1}n\avec1n)\{a_n/\ttt\}
       }
       {
        (\th kn\avec1n)\{a_n/\fff\}
       }
      )
     }
    ]
   }
  }
 }
}
\]
\end{definition}

%-------------------------------------------------------------------------------
\begin{theorem}\label{TheoThrDer}
For any $n>0$ and $k\ge0$, the derivation\/ $\vlsmash{\Gth k{}n\avec1n}$ has shape
\[
\vlder{}{\SKS\setminus\{\aid,\aiu\}}
{
 \vls([a_1.(\th{k+1}n\avec1n)\{a_1/\fff\}].\cdots.[a_n.(\th{k+1}n\avec1n)\{a_n/\fff\}])
}
{
 \vls[(a_1.(\th{k}n\avec1n)\{a_1/\fff\}).\cdots.(a_n.(\th{k}n\avec1n)\{a_n/\fff\})]
}
\quad,
\]
and\/ $\size{\Gth k{}n\avec1n}$ is $n^{\Ord{\log n}}$.
\end{theorem}

\TODO{Check correctness and integrate into the rest a bit better. Also find the correct place for it (before/after the definition of the class of threshold formulae we use (as is, it expects to be after)).}

\TODO{Add a few words to the effect: `We have now seen that threshold formulae do the job, but do we HAVE to use threshold formulae?'}

%------------------------------------------------------
\begin{theorem}
Given atoms $a_1$, $\dots$, $a_n$ and, for every $0\le k\le n$, formulae $\gamma_{k,1}$, $\dots$, $\gamma_{k,n}$, such that
\[
\gamma_{0,1}=\cdots=\gamma_{0,n}=\ttt
\quad\mbox{and}\quad
\gamma_{l,1}=\cdots=\gamma_{l,n}=\fff
\quad,
\]
for some $l>0$, and such that, for every $1\le k\le l$, there exists a derivation
\[
\vlder{}{\SKS\setminus\{\aid,\aiu\}}
{
 \vls([a_1.\gamma_{k,1}].\cdots.[a_n.\gamma_{k,n}])
}
{
 \vls[(a_1.\gamma_{k-1,1}).\cdots.(a_n.\gamma_{k-1,n})]
}\quad.
\]
Then, for every $1\le i\le n$ and every $0\le k\le l$, the formula $\gamma_{k,i}$
\begin{itemize}
 \item is true if at least $k$ of the atoms $a_1$, $\dots$, $a_{i-1}$, $a_{i+1}$, $\dots$, $a_n$ are true, and
 \item is false if at least $l-k$ of the atoms $a_1$, $\dots$, $a_{i-1}$, $a_{i+1}$, $\dots$, $a_n$ are false.
\end{itemize}
\end{theorem}

%------------------------------------------------------
\begin{proof}
Let $\theta_k\equiv\theta_k^n\avec 1n$, be the threshold formulae defined in Definition~\ref{TODO}.

We first observe that, for every $0\le k\le n$ and for every $1\le i\le n$
\begin{itemize}
 \item $\theta_k\{a_i/\fff\}$ is true if at least $k$ of the atoms $a_1$, $\dots$, $a_{i-1}$, $a_{i+1}$, $\dots$, $a_n$ are true, and
 \item $\theta_{n-l+k+1}\{a_i/\ttt\}$ is false if at least $l-k$ of the atoms $a_1$, $\dots$, $a_{i-1}$, $a_{i+1}$, $\dots$, $a_n$ are false.
\end{itemize}
Hence, we can prove the claim by showing that derivations $\Phi_{k,i}$ and $\Psi_{k,i}$ exist such that:
\[
\vlderivation
{
 \vlde{\Psi_{k,i}}{}
 {
  \theta_{n-l+k+1}\{a_i/\ttt\}
 }
 {
  \vlde{\Phi_{k,i}}{}
  {
   \gamma_{k,i}
  }
  {
   \vlhy
   {
    \theta_{k,i}\{a_i/\fff\}
   }
  }
 }
}\quad.
\]

%---
We prove by induction on $k$ how to build $\Phi_{k,i}$. The base case is trivial:
\[
 \vlinf{=}{}
 {
  \gamma_{0,i}
 }
 {
  \theta_0\{a_i/\fff\}
 }\quad.
\]
We prove the inductive case by fixing a $k>0$. We need the following derivation:
\[
\vlder{}{\SKS\setminus\{\aid,\aiu\}}
{
\vls[(a_1.\theta_k\{a_1/\fff\}).\cdots.(a_n.\theta_k\{a_n/\fff\})]
}
{
 \theta_{k+1}
}\quad.
\]
We know such a derivation exists in system $\SKS$ since the premiss and conclusion are logically equivalent. Furthermore, since the premiss and conclusion do not contain any pair of dual atom occurrences we know that a derivation exists in the system $\SKS\setminus\{\aid,\aiu\}$, by Proposition~\ref{TODO}. We assume the following derivations exist
\[
\vlder{\Phi_{k,1}}{\SKS\setminus\{\aid,\aiu\}}
{
 \gamma_{k,1}
}
{
 \theta_k\{a_1/\fff\}
}
\quad,\cdots,\quad
\vlder{\Phi_{k,n}}{\SKS\setminus\{\aid,\aiu\}}
{
 \gamma_{k,n}
}
{
 \theta_k\{a_n/\fff\}
}\quad.
\]
Then we can construct, for every $1\le i\le n$:
\[
\Phi'_{k+1,i}\quad=\quad
\vlderivation
{
 \vlde{}{\{\awu\}}
 {
  \vls[a_i.\gamma_{k+1,i}]
 }
 {
  \vlde{\Gamma_k}{\SKS\setminus\{\aid,\aiu\}}
  {
   \vls([a_1.\gamma_{k+1,1}].\cdots.[a_n.\gamma_{k+1,n}])
  }
  {
   \vlde{}{\SKS\setminus\{\aid,\aiu\}}
   {
    \vls
    [
     (
      a_1
     .
      \vlder{\Phi_{k,1}}{\SKS\setminus\{\aid,\aiu\}}
      {
       \gamma_{k,1}
      }
      {
       \theta_k\{a_1/\fff\}
      }
     )
    .\cdots.
     (
      a_n
     .
      \vlder{\Phi_{k,n}}{\SKS\setminus\{\aid,\aiu\}}
      {
       \gamma_{k,n}
      }
      {
       \theta_n\{a_1/\fff\}
      }
     )
    ]
   }
   {
    \vlde{}{\{\awd\}}
    {
     \theta_{k+1}
    }
    {
     \vlhy{\theta_{k+1}\{a_i/\fff\}}
    }
   }
  }
 }
}\quad.
\]
Since $\Phi'_{k+1,i}$ does not contain any $\aid$ instances and the premiss of $\Phi'_{k+1,i}$ does not contain any occurrences of $a_i$, we can apply weakening reductions to obtain:
\[
\vlder{\Phi_{k+1,i}}{\SKS\setminus\{\aid,\aiu\}}
{
 \vlsbr[\vlinf{}{}{a_i}{\fff}.\gamma_{k+1,i}]
}
{
 \theta_{k+1}\{a_i/\fff\}
}\quad,
\]
where $\Phi_{k+1,i}$ is the required derivation.

%---
We prove by induction on $k$ how to build $\Psi_{k,i}$. The base case is trivial:
\[
 \vlinf{=}{}
 {
  \theta_{n+1}\{a_i/\ttt\}
 }
 {
  \gamma_{l,i}
 }
\]
We prove the inductive case by fixing a $k<l$. We need the following derivation:
\[
\vlder{}{\SKS\setminus\{\aid,\aiu\}}
{
\vls[(a_1.\theta_{n-l+k}\{a_1/\fff\}).\cdots.(a_n.\theta_{n-l+k}\{a_n/\fff\})]
}
{
 \theta_{n-l+k+1}
}\quad,
\]
We know such a derivation exists in system $\SKS$ since the premiss and conclusion are logically equivalent. Furthermore, since the premiss and conclusion do not contain any pair of dual atom occurrences we know that a derivation exists in the system $\SKS\setminus\{\aid,\aiu\}$, by Proposition~\ref{TODO}. We assume the following derivations exist

\[
\vlder{\Psi_{k,1}}{\SKS\setminus\{\aid,\aiu\}}
{
 \theta_{n-l+k+1}\{a_1/\ttt\}
}
{
 \gamma_{k,1}
}
\quad,\cdots,\quad
\vlder{\Psi_{k,n}}{\SKS\setminus\{\aid,\aiu\}}
{
 \theta_{n-l+k+1}\{a_n/\ttt\}
}
{
 \gamma_{k,n}
}\quad.
\]
Then we can construct, for every $1\le i\le n$:
\[
\Psi'_{k-1,i}\quad=\quad
\vlderivation
{
 \vlde{}{\{\awu\}}
 {
  \theta_{n-l+k}\{a_i/\ttt\}
 }
 {
  \vlde{}{}
  {
   \theta_{n-l+k}
  }
  {
   \vlde{\Gamma_k}{\SKS\setminus\{\aid,\aiu\}}
   {
    \vls
    (
     [
      a_1
     .
      \vlder{\Psi_{k,1}}{\SKS\setminus\{\aid,\aiu\}}
      {
       \theta_{n-l+k+1}\{a_1/\ttt\}
      }
      {
       \gamma_{k,1}
      }
     ]
    .\cdots.
     [
      a_n
     .
      \vlder{\Psi_{k,n}}{\SKS\setminus\{\aid,\aiu\}}
      {
       \theta_{n-l+k+1}\{a_n/\ttt\}
      }
      {
       \gamma_{k,n}
      }
     ]
    )
   }
   {
    \vlde{}{\{\awd\}}
    {
     \vls[(a_1.\gamma_{k-1,1}).\cdots.(a_n.\gamma_{k-1,n})]
    }
    {
     \vlhy{\vls(a_i.\gamma_{k-1,i})}
    }
   }
  }
 }
}\quad.
\]
Since $\Psi'_{k-1,i}$ does not contain any $\aiu$ instances and the conclusion of $\Psi'_{k-1,i}$ does not contain any occurrences of $a_i$, we can apply weakening reductions to obtain:
\[
\vlder{\Psi_{k-1,i}}{\SKS\setminus\{\aid,\aiu\}}
{
 \theta_{n-l+k}\{a_i/\ttt\}
}
{
 \vlsbr(\vlinf{}{}{\ttt}{a_i}.\gamma_{k-1,i})
}\quad,
\]
where $\Psi_{k-1,i}$ is the required derivation.
\end{proof}

\TODO{Fix the ambiguous subscripts in the following remark.}

%------------------------------------------------
\begin{remark}
In the above theorem $l\ge n$ and if $l=n$, then
\[
\theta_{k,i}\{a_i/\fff\}\leftrightarrow\gamma_{k,i}\leftrightarrow\theta_{k+1,i}\{a_i/\ttt\}
\quad.
\]
\end{remark}

%================================================================================
\subsubsection{Quasipolynomial Reduction}

\TODO{Either rephrase this in terms of an atomic flow transformation and a soundness proof or move it away from here.}

\TODO{Define for derivations instead of proofs. Use simple form of derivations with respect to a polarity assignment. Use the decomposed normal form, yet to be given a name, for derivations.}

\TODO{Come up with a better name than cut-free form (as it is not an accurate description if we talk about derivations.).}

\newcommand{\frqp}{{\mathsf{qp}}}

\TODO{Use the symbol $\frqp$ to denote the following reduction.}

%-------------------------------------------------------------------------------
\begin{definition}\label{DefNorm}
Consider the derivation
\[
\vlder{\Phi}{}
{
 \vls[\beta\;.\;\vlinf{}{}{\fff}{\vls(a_1.\bar a_1^{\psi_1})}\;.\;\vlinf{}{}{\fff}{\vls(a_n.\bar a_n^{\psi_n})}]
}
{
 \vls(\vlinf{}{}{\vls[a_1.\bar a_1^{\psi_1}]}{\ttt}\;.\;\vlinf{}{}{\vls[a_n.\bar a_n^{\psi_n}]}{\ttt}\;.\;\alpha)
}\quad,
\]
with atomic flow $\phi$, such that, for $1\le i\le n$, $\psi_i$ is an isolated subflow of $\phi$ that does not contain identity or cut instances. For $0\le i\le n+1$, let $\theta_i\equiv\vlsmallbrackets\vls([a_1.(\th in\avec1n)\{a_1/\fff\}].\cdots.[a_n.(\th in\avec1n)\{a_n/\fff\}])$. We define $\Phi_0$ and $\Phi_n$ as
\[
\vlder{\Phi\{\bar a_1^{\psi_1}/\ttt,\dots,\bar a_n^{\psi_n}/\ttt\}}{}
{
 \vlsbr
 [
  \beta
 \;\;.\;\;
  \vlder{\Gth k{}n\avec1n}{}
  {
   \theta_1
  }
  {
   \vls[a_1.\cdots.a_n]
  }
 ]
}
{
 \vls([\vlinf{}{}{a_1}{\fff}\;.\;\ttt]\;.\;[\vlinf{}{}{a_n}{\fff}\;.\;\ttt]\;.\;\alpha)
}
\quad\mbox{and}\quad
\vlder{\Phi\{\bar a_1^{\psi_1}/\fff,\dots,\bar a_n^{\psi_n}/\fff\}}{}
{
 \vls[\beta\;.\;(\vlinf{}{}{\ttt}{a_1}\;.\;\fff)\;.\;\cdots\;.\;(\vlinf{}{}{\ttt}{a_1}\;.\;\fff)]
}
{
 \vls(\theta_n.\alpha)
}\quad,
\]
respectively. For $1\le k\le n-1$, we define the derivations $\Phi_k$:
\[
\vlder{\Phi\{\bar a_1^{\psi_1}/(\th kn\avec1n)\{a_1/\fff\},\dots,\bar a_n^{\psi_n}/(\th kn\avec1n)\{a_n/\fff\}\}}{}
{
 \vlsbr
 [
  \beta
 \;\;.\;\;
  \vlder{\Gth k{}n\avec1n}{}
  {
   \theta_{k+1}
  }
  {
   \vls[(a_1.(\th{k}n\avec1n)\{a_1/\fff\}).\cdots.(a_n.(\th{k}n\avec1n)\{a_n/\fff\})]
  }
 ]
}
{
 \vls(\theta_k.\alpha)
}
\]
 where we use Proposition~\ref{PropAuxNorm}.

\TODO{Change the rest of this definition}
 
We define the \emph{cut-free form of\/ $\Pi$} as the following proof in $\SKS\setminus\{\aiu\}$:
\[
\vlderivation
{
 \vlin{n\cdot\cou}{}
 {
  \beta
 }
 {
  \vlin{\swi}{}
  {
   \vls
   [
    \beta\;\;.\;\;\cdots\;\;.\beta
   \;\;.\;\;
    \vlder{\Phi_n}{}
    {
     \beta
    }
    {
     \vls(\theta_n.\alpha)
    }
   ]
  }
  {
   \vlin{\swi}{}
   {
    \vls
    (
     [
      \beta\;\;.\;\;\cdots\;\;.\beta
     \;\;.\;\;
      \vlder{\Phi_{n-1}}{}
      {
       \vls[\beta.\theta_n]
      }
      {
       \vls(\theta_{n-1}.\alpha)
      }
     ]
    \;\;.\;\;
     \alpha
    )
   }
   {
    \vlin{\swi}{}
    {
     \vdots
    }
    {
     \vlin{\swi}{}
     {
      \vls
      (
       [
        \beta
       \;\;.\;\;
        \vlder{\Phi_1}{}
        {
         \vls[\beta.\theta_2]
        }
        {
         \vls(\theta_1.\alpha)
        }
       ]
      \;\;.\;\;
       \alpha\;\;.\;\;\cdots\;\;.\alpha
      )
     }
     {
      \vlin{n\cdot\cod}{}
      {
       \vls
       (
        \vlder{\Phi_0}{}
        {
         \vls[\beta.\theta_1]
        }
        {
         \alpha
        }
       \;\;.\;\;
        \alpha\;\;.\;\;\cdots\;\;.\alpha
       )
      }
      {
       \vlhy{\alpha}
      }
     }
    }
   }
  }
 }
}
\]
\end{definition}
