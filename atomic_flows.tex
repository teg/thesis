%================================================
\section{Atomic Flows}\label{section:AtomicFlows}

Atomic flows, which have been introduced in \cite{GuglGund:07:Normalis:lr}, are, essentially, specialised Buss flow graphs \cite{Buss:91:The-Unde:uq}. They are particular directed graphs associated with $\SKS$ derivations: every derivation yields one atomic flow obtained by tracing the atom occurrences in the derivation. Infinitely many derivations correspond to each atomic flow; this suggests that much of the information in a derivation is lost in its associated atomic flow; in particular, there is no information about instances of logical rules, only structural rules play a role. As shown in \cite{GuglGund:07:Normalis:lr}, it turns out that atomic flows contain sufficient structure to control normalisation procedures, providing in particular induction measures that can be used to ensure termination. Such normalisation procedures require exponential time on the size of the derivation to be normalised. In the present work, we improve the complexity of proof normalisation to quasipolynomial time, but an essential role is played by the complex logical relations of threshold formulae, which are external and independent from the given proof. This means that atomic flows are not sufficient to define the normalisation procedure; however, they still are a very convenient tool for defining and understanding several of its aspects.

We can single out three features of atomic flows that, in general, and not just in this work, help in designing normalisation procedures:

\begin{enumerate}
%---------------------------------------
\item\label{item:UseTop} Atomic flows conveniently express the topological structure of atom occurrences in a proof. This is especially useful for defining the `simple form' of proofs, in Definition~\ref{DefSimpleForm}.
%---------------------------------------
\item\label{item:UseSubst} Atomic flows provide for an efficient way to control substitutions for atom occurrences in derivations. This is especially useful for defining the `cut-free form' of proofs, in Definition~\ref{DefNorm}.
%---------------------------------------
\item\label{item:UseNorm} We can define graph rewriting systems over atomic flows that control normalisation procedures on derivations. This could be used for obtaining the `analytic form' of proofs, as we do in Theorem~\ref{ThNormAn}. 
\end{enumerate}
Our aim now is to quickly and informally provide the necessary notions about atomic flows, especially concerning aspects \eqref{ItemUseTop} and \eqref{ItemUseSubst} above. Although the feature \eqref{ItemUseNorm} of atomic flows did help us in obtaining proofs in analytic form, we estimate that formally introducing the necessary machinery is unjustified in this paper. In fact, given our limited needs here, we can operate directly on derivations, without the intermediate support of atomic flows. Nonetheless, being aware of the underlying atomic-flow methods is useful for the reader who wishes to further investigate this matter. So, we informally provide, in Section~\ref{SectNorm}, enough material to make the connection with the atomic-flow techniques that are fully developed in \cite{GuglGund:07:Normalis:lr}.

We obtain one atomic flow from each derivation by tracing all its atom occurrences and by keeping track of their creation and destruction (in identity/cut and weak\-en\-ing/co\-weak\-en\-ing instances), their duplication (in contraction/cocontraction instances) and their duality (in identity/cut instances). Technically, atomic flows are directed graphs of a special kind, but it is more intuitive to consider them as diagrams generated by composing \emph{elementary atomic flows} that belong to one of seven kinds.

The first kind of elementary atomic flow is the \emph{edge}
\[
\atomicflow
{
(0,0)*{\afvj4};
}
\]
which corresponds to one or more occurrences of the same atom in a given derivation, all of which are not active in any structural rule instance, \emph{i.e.}, they are not the atom occurrences that instantiate a structural rule.

The other six kinds of elementary diagrams are associated with the six structural inference rules, as shown in Figure~\vref{figure:AtomicFlowVertices}, and they are called \emph{vertices}; each vertex has some incident edges. At the left of each arrow, we see an instance of a structural rule, where the atom occurrences are labelled by small numerals; at the right of the arrow, we see the vertex corresponding to the rule instance, whose incident edges are labelled in accord with the atom occurrences they correspond to. We qualify each vertex according to the rule it corresponds to; for example, in a given atomic flow, we might talk about a \emph{contraction vertex}, or a \emph{cut vertex}, and so on. Instead of small numerals, sometimes we use $\epsilon$ or $\iota$ or colour to label edges (as well as atom occurrences), but we do not always use labels.

\newcommand{\one  }{{\mathchoice{\scriptstyle      \mathbf1}
                                {\scriptstyle      \mathbf1}
                                {\scriptstyle      \mathbf1}
                                {\scriptscriptstyle\mathbf1}}}
\newcommand{\two  }{{\mathchoice{\scriptstyle      \mathbf2}
                                {\scriptstyle      \mathbf2}
                                {\scriptstyle      \mathbf2}
                                {\scriptscriptstyle\mathbf2}}}
\newcommand{\three}{{\mathchoice{\scriptstyle      \mathbf3}
                                {\scriptstyle      \mathbf3}
                                {\scriptstyle      \mathbf3}
                                {\scriptscriptstyle\mathbf3}}}
%-------------------------------------------------------------------------------
\begin{figure}
\[
\begin{array}{@{}c@{}c@{}c@{}c@{}c@{}c@{}}
\vlinf\aid{}{\vls[a^\one.\bar a^\two]}\ttt&\quad{\to}\quad
\afaid{}{}{}{}{}{}&\qquad\qquad
\vlinf\awd{}{a^\one}\fff&\quad{\to}\quad
\afawd{}{}{}{}&\qquad\qquad
\vlinf\acd{}{a^\three}{\vls[a^\one.a^\two]}&\quad{\to}\quad
\afacd{}{}{}{}{}{}
\\
\noalign{\bigskip}
\vlinf\aiu{}\fff{\vls(a^\one.\bar a^\two)}&\quad{\to}\quad
\afaiu{}{}{}{}{}{}&\qquad\qquad
\vlinf\awu{}\ttt{a^\one}&\quad{\to}\quad
\afawu{}{}{}{}&\qquad\qquad
\vlinf\acu{}{\vls(a^\one.a^\two)}{a^\three}&\quad{\to}\quad
\afacu{}{}{}{}{}{}\\
\end{array}
\]
\caption{Vertices of atomic flows.}
\label{figure:AtomicFlowVertices}
\end{figure}

All edges are directed, but we do not explicitly show the orientation. Instead, we consider it as implicitly given by the way we draw them, namely, edges are oriented along the vertical direction. So, the vertices corresponding to dual rules, in Figure~\ref{figure:AtomicFlowVertices}, are distinct, for example, an identity vertex and a cut vertex are different because the orientation of their edges is different. On the other hand, the horizontal direction plays no role in distinguishing atomic flows; this corresponds to commutativity of logical relations.

\newcommand{\ppl}{{\mathchoice{\scriptstyle+}
                              {\scriptstyle+}
                              {\scriptstyle+}
                              {\scriptscriptstyle+}}}
\newcommand{\pmi}{{\mathchoice{\scriptstyle-}
                              {\scriptstyle-}
                              {\scriptstyle-}
                              {\scriptscriptstyle-}}}
We can define (\emph{atomic}) \emph{flows} as the smallest set of diagrams containing elementary atomic flows, and closed under the composition operation consisting in identifying zero or more edges such that no cycle is created. In addition, for a diagram to be an atomic flow, it must be possible to assign it a polarity, according to the following definition. A \emph{polarity assignment} is a mapping of each edge to an element of $\{\pmi,\ppl\}$, such that the two edges of each identity or cut vertex map to different values and the three edges of each contraction or cocontraction vertex map to the same value. We denote atomic flows by $\phi$ and $\psi$.

Let us see some examples. The flow
\begin{equation}\label{ExFlow}
\atomicflow{
(17,12)*{\afaid{}{}{}{}{}{}};
(-3, 8)*{\afvjd8{}{}};
(-1, 8)*{\afvjd8{}{}};
( 1, 8)*{\afvjd8{}{}};
( 4, 8)*{\afvjd8{}{}};
(10, 8)*{\afacu{}{}{}{}{}{}};
(17, 4)*{\afaiu{}{}{}{}{}{}};
( 6, 2)*{\afaiunw{}{}}}
\end{equation}
is obtained by juxtaposing (\emph{i.e.}, composing by identifying zero edges):
\begin{itemize}
\item three edges, 
\item a flow obtained by composing a cut vertex with a cocontraction vertex, and
\item a flow obtained by composing an identity vertex with a cut vertex.
\end{itemize}
Note that there are no cycles in the flow, and that we can find 32 different polarity assignments, \emph{i.e.}, two for each of the five connected components of the flow (this is a general rule).

\newcommand{\four}{{\mathchoice{\scriptstyle      \mathbf4}
                                {\scriptstyle      \mathbf4}
                                {\scriptstyle      \mathbf4}
                                {\scriptscriptstyle\mathbf4}}}
\newcommand{\five }{{\mathchoice{\scriptstyle\mathbf5}
                                {\scriptstyle\mathbf5}
                                {\scriptstyle\mathbf5}
                                {\scriptscriptstyle\mathbf5}}}


Let us see another example. These are three different representations of the same flow:
\[
\atomicflow{
(10,8)*{\afacu\four{}{}{}{}\two};
( 0,8)*{\afvjd8\one{}};
( 4,8)*{\afvjd8{}\five};
( 6,2)*{\afaiunw{}{}};
( 6,0)*{\afaiuex{}{}{}\three{}{}31}}
\quad,\qquad
\aflower{\atomicflow{
( 0  ,6)*{\afvjd{8}\one\ppl};
( 6  ,6)*{\afacu\three{}{}\four\two\pmi};
(12  ,6)*{\afvjd{8}\ppl\five};
(10  ,0)*{\afaiunw{}{}};
( 2  ,0)*{\afaiunw{}{}};
(-1.5,0)*{\invisiblemark};
(13.5,0)*{\invisiblemark}}}
\qquad\hbox{and}\qquad
\atomicflow{
( 8  ,10)*{\afacu{}\three{}\four\two\ppl};
( 0  , 8)*{\afvjd{12}\one\pmi};
( 4  ,10)*{\afvjd{8}\five\pmi};
( 5  , 4)*{\afex24};
(10  , 4)*{\afvj4};
( 2  , 0)*{\afaiunw{}{}};
( 8  , 0)*{\afaiunw{}{}};
(-1.5, 0)*{\invisiblemark};
(11.5, 0)*{\invisiblemark}}
\quad,
\]
where we label edges to show their correspondence. In the two rightmost flows, we indicate the two different polarity assignments that are possible.

The following two diagrams are not atomic flows:
\[
\atomicflow{
(4,11.7)*{\afacu{}{}{}{}{}{}};
(0, 3.7)*{\afacd{}{}{}{}{}{}};
(8, 7.7)*{\afvj{16}};
(4,16);(8,16)**[|<\atflowthickone>]\crv{(4,18)&(6,20)&(8,18)};
(0,0);(8,0)**[|<\atflowthickone>]\crv{(0,-2)&(4,-4)&(8,-2)}
}
\qquad\mbox{and}\qquad
\atomicflow{
(0,4)*{\afaidnw{}{}};
(0,0)*{\afacd{}{}{}{}{}{}}
}\quad.
\]
The left one is not a flow because it contains a cycle, and the right one because there is no possible polarity assignment.

\newbox\contrup\setbox\contrup=\hbox{$
   \divide\atflowunit by6\multiply\atflowunit by3\afsetunits
   \atomicflow{(0,0)*{\afacu{}{}{}{}{}{}}}$}
\newbox\contrdown\setbox\contrdown=\hbox{$
   \divide\atflowunit by6\multiply\atflowunit by3\afsetunits
   \atomicflow{(0,0)*{\afacd{}{}{}{}{}{}}}$}
\newbox\interdown\setbox\interdown=\hbox{$
   \divide\atflowunit by6\multiply\atflowunit by3\afsetunits
   \atomicflow{(0,0)*{\afaid{}{}{}{}{}{}}}$}
\newbox\interup\setbox\interup=\hbox{$
   \divide\atflowunit by6\multiply\atflowunit by3\afsetunits
   \atomicflow{(0,0)*{\afaiu{}{}{}{}{}{}}}$}
\newbox\weakdown\setbox\weakdown=\hbox{$
   \divide\atflowunit by6\multiply\atflowunit by3\afsetunits
   \atomicflow{(0,0)*{\afawd{}{}{}{}{}{}}}$}
\newbox\weakup\setbox\weakup=\hbox{$
   \divide\atflowunit by6\multiply\atflowunit by3\afsetunits
   \atomicflow{(0,0)*{\afawu{}{}{}{}{}{}}}$}

%---------------------------------------------
\begin{notation}\label{notation:LabelsOnBoxes}
Let $\phi$ be an atomic flow with upper edges $\boldsymbol\epsilon=\epsilon_1,\dots,\epsilon_n$ and lower edges $\boldsymbol\iota=\iota_1,\dots,\iota_m$, we then represent it as
\[
\atomicflow{
(-3, 6)*{\afvju4{\epsilon_1}{}};
( 0, 6)*{\cdots};
( 3, 6)*{\afvju4{}{\epsilon_n}};
( 0, 0)*{\affr88};
( 1, 2)*{\aflabelright\phi};
(-3,-6)*{\afvju4{\iota_1}{}};
( 0,-6)*{\cdots};
( 3,-6)*{\afvju4{}{\iota_m}};
}
\qquad\mbox{or}\qquad
\atomicflow{
( 0,6)*{\afvjum4{\boldsymbol\epsilon}{}};
( 0,0)*{\affr88};
( 1,2)*{\aflabelright\phi};
( 0,-6)*{\afvjdm4{\boldsymbol\iota}{}};
}\quad.
\]
We sometimes use flow labels to indicate what kind of vertices an atomic flow might contain. \emph{E.g.}, the following flows
\[
\atomicflow{
( 0, 6)*{\afvjm4};
(-2, 0)*{\copy\contrup};
( 0, 0)*{\affr{10}8};
( 2, 0)*{\copy\contrdown};
( 0,-6)*{\afvjm4};
}\qquad\mbox{and}\qquad
\atomicflow{
(0, 6)*{\afvjm4};
(0, 0)*{\affr88};
(0, 0)*{\copy\contrdown};
(0,-6)*{\afvjm4};
}\quad,
\]
do not contain $\aid$, $\aiu$, $\awd$, $\awu$ vertices, and the flow to the right does not contain $\acu$ vertices.
\end{notation}
%-------------

%===========================================================
\subsection{Flow Isomorphism}\label{section:FlowIsomorphism}

\newcommand{\up}{{\mathit up}}
\newcommand{\lo}{{\mathit lo}}

\TODO{Define: $\up$ and $\lo$.}

%---------------------------------------------------
\begin{definition}\label{definition:FlowIsomorphism}
An (\emph{atomic}) \emph{flow isomorphism}\index{flow!isomorphism} between the atomic flows $\phi_1=(V_1,E_1,\eta_1,\up_1,\lo_1)$ and $\phi_2=(V_2,E_2,\eta_2,\up_2,\lo_2)$ is a pair of functions $(f_V,f_E)$, such that
\begin{itemize}
\item $f_V$ is a bijection from $V_1$ to $V_2$; and 
\item $f_E$ is a bijection from $E_1$ to $E_2$,
\end{itemize}
such that, for every $\epsilon$ in $E_1$,
\begin{itemize}
\item for every $\nu$ in $V_1$, $\up_1(\epsilon)=\nu$ (resp., $\lo_1(\epsilon)=\nu$) if and only if $\up_2(f_E(\epsilon))=f_V(\nu)$ (resp., $\lo_2(f_E(\epsilon))=f_V(\nu)$); and
\item $\up_1(\epsilon)=\top$ (resp., $\lo_1(\epsilon)=\bot$) if and only if $\up_2(f_E(\epsilon))=\top$ (resp., $\lo_2(f_E(\epsilon))=\bot$).
\end{itemize}
\end{definition}
%---------------

%---------------
\begin{notation}
We extend the double-line notation to collections of isomorphic flows. For example, for $n\ge0$; $\boldsymbol{\epsilon}=\epsilon_1,\dots,\epsilon_n$; $\boldsymbol{\epsilon'}=\epsilon'_1,\dots,\epsilon'_n$; and $\boldsymbol{\epsilon''}=\epsilon''_1,\dots,\epsilon''_n$, the following diagrams represent the same flow:
\[
\atomicflow
{
(-9,0)*{\invisiblemark};
(-6,0)*{\afvjum8{\boldsymbol{\epsilon}}{}};
(-4,4)*{\afaidnw{}{}};
( 0,0)*{\afacdm{}{}{}{}{\boldsymbol{\epsilon'}}{}};
( 4,4)*{\afaidnw{}{}};
( 6,0)*{\afvjum8{}{\boldsymbol{\epsilon''}}};
( 9,0)*{\invisiblemark};
}
\quad\mbox{and}\quad
\atomicflow
{
(-9,0)*{\invisiblemark};
(-6,0)*{\afvju8{\epsilon_1}{}};
(-4,4)*{\afaidnw{}{}};
( 0,0)*{\afacd{}{}{}{}{\epsilon'_1}{}};
( 4,4)*{\afaidnw{}{}};
( 6,0)*{\afvju8{}{\epsilon''_1}};
( 9,0)*{\invisiblemark};
}
\quad\cdots\quad
\atomicflow
{
(-9,0)*{\invisiblemark};
(-6,0)*{\afvju8{\epsilon_n}{}};
(-4,4)*{\afaidnw{}{}};
( 0,0)*{\afacd{}{}{}{}{\epsilon'_n}{}};
( 4,4)*{\afaidnw{}{}};
( 6,0)*{\afvju8{}{\epsilon''_n}};
( 9,0)*{\invisiblemark};
}\quad.
\]
\end{notation}
%---------------


%-----------------------------------------------------
\begin{notation}\label{notation:LabelsIsomorphicFlows}
Given an atomic flow
\[
\atomicflow
{
(0, 6)*{\afvjdm4{\boldsymbol\epsilon}{}};
(0, 0)*{\affr88};
(4, 2)*{\aflabelleft{\phi}};
(0,-6)*{\afvjum4{\boldsymbol\iota}{}};
}
\quad,
\]
and a flow $\psi$ which is isomorphic to $\phi$, whenever we write
\[
\psi\;=\;
\atomicflow
{
(0, 6)*{\afvjdm4{f(\boldsymbol\epsilon)}{}};
( 0, 0)*{\affr88};
(4, 2)*{\aflabelleft{f(\phi)}};
(0,-6)*{\afvjum4{f(\boldsymbol\iota)}{}};
}
\quad,
\]
we mean that $f$ is a given flow isomorphism between $\phi$ and $\psi$.
\end{notation}
%-------------

%----------------------------------------------
\begin{convention}\label{convention:EqualFlows}
We consider atomic flows to be equal modulo flow isomorphisms and modulo the equivalence relation generated by the following graph rewriting system:
\[
\atomicflow
{
(-2,2.5)*{\afacdnw{\one}{}{\two}{}};
(0,-2)*{\afacd{}{}{}{}{\four}{}};
(2,4.35)*{\afvjd{4.6}{\three}{}}
}\quad\rightarrow\quad
\atomicflow
{
(2,2.5)*{\afacdnw{\two}{}{\three}{}};
(0,-2)*{\afacd{}{}{}{}{\four}{}};
(-2,4.35)*{\afvjd{4.6}{\one}{}}
}\quad,\qquad
\atomicflow
{
(-2,-4.5)*{\afacunw{\two}{}{\three}{}};
(0,2)*{\afacu{}{}{}{}{\one}{}};
(2,-4.35)*{\afvju{4.6}{\four}{}}
}\quad\rightarrow\quad
\atomicflow
{
(2,-4.5)*{\afacunw{\three}{}{\four}{}};
(0,2)*{\afacu{}{}{}{}{\one}{}};
(-2,-4.35)*{\afvju{4.6}{\two}{}}
}\quad.
\]
\end{convention}
%---------------

%---------------
\begin{notation}\label{notation:LabelsFlowPolarity}
Given an atomic flow $\phi$ and a polarity assignment $\pi$ for $\phi$, whenever we write
\[
\atomicflow{
( 0,6)*{\afvjum4{}{}};
( 0,0)*{\affr88};
(-4,2)*{\aflabelright\ppl};
( 4,2)*{\aflabelleft\phi};
( 0,-6)*{\afvjdm4{}{}};
}\qquad\mbox{or}\qquad
\atomicflow{
( 0,6)*{\afvjum4{}{}};
( 0,0)*{\affr88};
(-4,2)*{\aflabelright\pmi};
( 4,2)*{\aflabelleft\phi};
( 0,-6)*{\afvjdm4{}{}};
}\quad,
\]
respectively, we mean that all the edges in $\phi$ have polarity assignment $\ppl$ or $\pmi$, respectively. If we label an atomic flow with a polarity assignment it can not contain any interaction or cut vertices duo to property~\ref{definition:AtomicFlow:item:PolarityAssignment} of Definition~\vref{definition:AtomicFlow}.
\end{notation}
%-------------

%-----------------
\begin{definition}\label{definition:DualPolarity}
Given an atomic flow $\phi$ and a polarity assignment $\pi$ for $\phi$, the polarity assignment $\bar\pi$ for $\phi$ is defined to be, for every $\epsilon$ in $\phi$:
\[
\bar\pi(\epsilon)=
\begin{cases}
\pmi & \mbox{if $\pi(\epsilon)=\ppl$,}
\\
\ppl & \mbox{otherwise.}
\end{cases}
\]
\end{definition}
%---------------

%=======================================================
\subsection{Paths}\label{subsection:Paths}

We now define the notion of `path' in atomic flows. Paths are sequences of adjacent edges that only `go down' or only `go up'.

%---------------------------------------
\begin{definition}\label{definition:FlowPaths}
Given an atomic flow $(V,E,\eta,\up,\lo)$ and $\epsilon_1,\dots,\epsilon_h\in E$ such that, for $1\le i<h$, we have $\lo(\epsilon_i)=\up(\epsilon_{i+1})$, $\up(\epsilon_1)=\nu$ and $\lo(\epsilon_h)=\nu'$, we say that $\epsilon_1,\dots,\epsilon_h$ is a \emph{path from $\nu$ to $\nu'$}\index{path} and that $\epsilon_h,\dots,\epsilon_1$ is a \emph{path from $\nu'$ to $\nu$}; both paths have \emph{length} $h$\index{path!length}.
\end{definition}

%---------------------------------------
\begin{example}\label{example:Paths}
The atomic flow on the left has the paths on the right:
\[
\begin{array}{@{}c@{}c@{}}
\atomicflow{
( 2  ,10)*{\afaidnw{}{}};
( 0  , 6)*{\afvju8\one{}};
( 6  , 6)*{\afacd\two{}\three{}\four{}};
( 8  ,10)*{\afawdnw{}{}};
(10  , 6)*{\afvjd8{}\five};
( 8  , 0)*{\afaiunw{}{}};
(-1.5, 0)*{\invisiblemark};
(11.5, 0)*{\invisiblemark}}
\qquad&
\begin{array}{@{}lllll@{}}
\one                 &                &                  &           &       \\
                     &\two            &\three            &           &       \\
                     &\two,\four      &\three,\four      &\four      &       \\
                     &                &                  &           &\five  \\
\end{array}\quad.
\\
\end{array}
\]
In addition, the flow has the paths obtained from the shown ones by inverting the order of edges, for example $\four,\three$ is a path. The paths from the interaction vertex are $\one$ and $\two$ and $\two,\four$; the paths to the contraction vertex are $\two$ and $\three$ and $\four$.
\end{example}

%=========================================
\subsection{Subflows}\label{subsection:Subflows}

\TODO{Give examples.}

%-------------------------------------------
\begin{definition}\label{definition:Subflow}
We say that the flow $\phi_1=(V_1,E_1,\eta_1,\up_1,\lo_1)$ is a \emph{subflow}\index{subflow} of\newline $\phi_2=(V_2,E_2,\eta_2,\up_2,\lo_2)$, if
\begin{itemize}
\item $V_1\subset V_2$;
\item $E_1\subset E_2$;
\item $\eta_1=\eta_2|_{V_1}$;
\item for every $\epsilon$ in $E_1$
\[
\up_1(\epsilon)=
\begin{cases}
\up_2(\epsilon) & \mbox{if $\up_2(\epsilon)\in V_1$,}
\\
\top & \mbox{otherwise.}
\end{cases}
\qquad\mbox{and}\qquad
\lo_1(\epsilon)=
\begin{cases}
\lo_2(\epsilon) & \mbox{if $\lo_2(\epsilon)\in V_1$,}
\\
\bot & \mbox{otherwise.}
\end{cases}\quad;
\]
\item if $\nu$ is a vertex in $\phi_1$ then $U_\nu$ and $L_\nu$ are subsets of $E_1$; and
\item
if $\nu_1$ and $\nu_2$ are vertices in $\phi_1$, and there is a vertex $\nu'$ in $\phi_2$, such that there are paths from $\nu_1$ to $\nu'$ and from $\nu'$ to $\nu_2$ in $\phi_2$, then $\nu'$ is a vertex in $\phi_1$.
\end{itemize}
\end{definition}
%---------------

%--------------------------------------------------
\begin{definition}\label{defintion:IsolatedSubflow}
Given two flows $\phi$ and $\psi$, such that $\phi$ is a subflow of $\psi$, we say that \emph{$\phi$ is an isolated subflow of $\psi$}\index{subflow!isolated} if there is no path in $\phi$ from a vertex in $\psi$ to $\top$ or $\bot$.
\end{definition}
%---------------

%------------------------------------------------------
\begin{definition}\label{definition:ConnectedComponent}
Given two flows $\phi$ and $\psi$, such that $\phi$ is a subflow of $\psi$, we say that \emph{$\phi$ is a connected component of $\psi$}\index{connected component} if, for any two polarity assignments $\pi$ and $\pi'$ for $\psi$ and for any two edges $\epsilon$ and $\epsilon'$ in $\phi$, $\pi(\epsilon)=\pi(\epsilon')$ if and only if $\pi'(\epsilon)=\pi'(\epsilon')$.
\end{definition}
%---------------