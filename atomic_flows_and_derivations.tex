%------------------------------------------------------------------------
\chapter{Atomic Flows and Derivations}\label{chapter:FlowsAndDerivations}

\section{Extracting Flows from Derivations}\label{section:ExtractingFlowsFromDerivations}

We now define the mapping from derivations to flows. As we said, the idea is that structural rule instances map to the respective atomic-flow vertices, and the edges trace the atom occurrences between rule instances. We first state a fact, whose proof is immediate.

\Tom{Rehprased slightly and added ``up to isomorphism''.}

\Tom{Added sketch of proof.}

%------------------------------------------------
\begin{proposition}\label{proposition:FlowUnique}
Given a derivation\/ $\Phi$ in the system $\SKS$, there is a unique (up to isomorphism) flow $\phi$ such that:
\begin{enumerate}
%
\item there is a surjective map between the set of atom occurrences of\/ $\Phi$ and the set of edges of $\phi$;
%
\item there is a bijective map between the set of structural inference rule instances of\/ $\Phi$ and the set of vertices of $\phi$, such that, for each inference rule instance $\rho$ that maps to a vertex $\nu$, the label of $\nu$ are given below, for each possible case of the inference rules:
\[
\begin{array}{@{}ccc@{}ccc@{}}
\vlinf{\aid}{}{\vls[a^\one.{\bar a^\two}]}{\ttt}&\mbox{to\/}&
\vcenter{\hbox{\includegraphics{Figures/vertAFAID}}}
\quad,&\qquad
\vlinf{\aiu}{}{\fff}{\vls(a^\one.{\bar a^\two})}&\mbox{to\/}&
\vcenter{\hbox{\includegraphics{Figures/vertAFAIU}}}
\quad,\\
\noalign{\medskip}
\vlinf{\awd}{}{a^\one}{\fff}                    &\mbox{to\/}&
\vcenter{\hbox{\includegraphics{Figures/vertAFAWD}}}
\quad,&\qquad
\vlinf{\awu}{}{\ttt}{a^\one}                    &\mbox{to\/}&
\vcenter{\hbox{\includegraphics{Figures/vertAFAWU}}}
\quad,\\
\noalign{\medskip}
\vlinf{\acd}{}{a^\three}{\vls[a^\one.a^\two]}   &\mbox{to\/}&
\vcenter{\hbox{\includegraphics{Figures/vertAFACD}}}
\quad,&\qquad
\vlinf{\acu}{}{\vls(a^\two.a^\three)}{a^\one}   &\mbox{to\/}&
\vcenter{\hbox{\includegraphics{Figures/vertAFACU}}}
\quad,\\
\end{array}
\]
and the map between the atom occurrences in the premiss (resp., conclusion) of $\rho$ and the upper (resp., lower) edges of $\nu$ is indicated by small numerals; and
%
\item for each inference rule instance of\/ $\Phi$ of kind
\[\hss
\begin{array}{@{}r@{}l@{}}
\vlinf{\swi}{}{\vls[(\alpha.\beta).\gamma]}
              {\vls(\alpha.[\beta.\gamma])}           \quad,&\qquad
\vlinf{\med}{}{\vls([\alpha.\gamma].[\beta.\delta])}
              {\vls[(\alpha.\beta).(\gamma.\delta)]}  \quad,      \\
\noalign{\smallskip}
\vlinf{\coor}{}{\vls[\beta.\alpha]}{\vls[\alpha.\beta]}\quad,\qquad
\vlinf{\coand}{}{\vls(\beta.\alpha)}{\vls(\alpha.\beta)}\quad,&\qquad
\vlinf{\asor}{}{\vls[[\alpha.\beta].\gamma]}
         {\vls[\alpha.[\beta.\gamma]]}                \quad,\qquad
\vlinf{\asand}{}{\vls(\alpha.(\beta.\gamma))}
         {\vls((\alpha.\beta).\gamma)}                \quad,      \\
\noalign{\smallskip}
\vlinf{\fffd}{}{\vls[\alpha.\fff]}{\alpha}           \quad,\qquad
\vlinf{\tttd}{}{\vls(\alpha.\ttt)}{\alpha}           \quad,&\qquad
\vlinf{\fffu}{}{\alpha}{\vls(\ttt.\alpha)}        \qquad\hbox{and\/}\qquad
\vlinf{\tttu}{}{\alpha}{\vls[\fff.\alpha]}
\end{array}
\]
all the atom occurrences in $\alpha$, $\beta$, $\gamma$ and $\delta$ in the premiss are respectively mapped to the same edges of $\phi$ as the atom occurrences in $\alpha$, $\beta$, $\gamma$ and $\delta$ in the conclusion.
\end{enumerate}
\end{proposition}

\begin{proof}
We sketch the proof: We proceede by induction on $\Phi$. If $\Phi$ is a formula or a horizontal composition of derivations, the result is immediate. Otherwise, if $\Phi$ is a vertical composition of two derivations by an inference rule, the result follows by induction and a case analysis of each inference rule of $\SKS$.
\end{proof}

\Tom{I decided not to do this for lack of time (unless you insist of course):}

\TODO{Macro for ``mapped from occurrences of''.}

\Tom{Moved definition of $a^\phi$ and $a^\epsilon$ away from here.}

%---------------------------------------
\begin{definition}\label{definition:AssociatedFlow}
Given a derivation $\Phi$, we say that the flow $\phi$ constructed in the proof of  Proposition~\vref{proposition:FlowUnique} is the (\emph{atomic}) \emph{flow associated with}\index{flow!associated with derivation} the derivation $\Phi$.
\end{definition}
%---------------

%--------------
\begin{example}
Figure~\vref{figure:ExampleAtomicFlows} has three examples of derivations and their associated flows, where colors are used to indicate the mapping from atom occurrences to edges.
\end{example}
%------------

\newcommand{\RD}[1]{{\color{Red}#1}}
\newcommand{\GR}[1]{{\color{Green}#1}}
\newcommand{\DO}[1]{{\color{DarkOrchid}#1}}
\newcommand{\PR}[1]{{\color{ProcessBlue}#1}}
\newcommand{\MG}[1]{{\color{Magenta}#1}}
\newcommand{\SG}[1]{{\color{SpringGreen}#1}}
\newcommand{\RS}[1]{{\color{RawSienna}#1}}
\newcommand{\YO}[1]{{\color{YellowOrange}#1}}
\newcommand{\PW}[1]{{\color{Periwinkle}#1}}
%-------------------------------------------------------------------------------
\begin{figure}
\[
%---------------------------------------
\begin{array}{@{}c@{}c@{}c@{}}
\vlderivation                                                {
\vlin=   {}{\ttt                                  }         {
\vlin\aiu{}{\vls[\fff.\ttt]                       }        {
\vlin=   {}{\vls[(\GR{a}.\RD{\bar a}).\ttt]       }       {
\vlin\swi{}{\vls[[(\RD{\bar a}.\GR{a}).\ttt].\ttt]}      {
\vlin=   {}{\vls[(\RD{\bar a}.[\GR{a}.\ttt]).\ttt]}     {
\vlin\swi{}{\vls[([\GR{a}.\ttt].\RD{\bar a}).\ttt]}    {
\vlin=   {}{\vls([\GR{a}.\ttt].[\RD{\bar a}.\ttt])}   {
\vlin\med{}{\vls([\GR{a}.\ttt].[\ttt.\RD{\bar a}])}  {
\vlin=   {}{\vls[(\GR{a}.\ttt).(\ttt.\RD{\bar a})]} {
\vlin\aid{}{\vls[\GR{a}.\RD{\bar a}]              }{
\vlhy      {\ttt                                  }}}}}}}}}}}}
\qquad&
%-------------------
\vlderivation                                                              {
\vlin\aiu{}
   {\vls(\DO{a}.\fff)                                            }        {
\vlin=   {}
   {\vls(\DO{a}.(\PR{a}.\MG{\bar a}))                            }       {
\vlin\acu{}
   {\vls((\DO{a}.\PR{a}).\MG{\bar a})                            }      {
\vlin=   {}
   {\vls(\SG{a}.\MG{\bar a})                                     }     {
\vlin\aiu{}
   {\vls([\fff.\SG{a}].\MG{\bar a})                              }    {
\vlin\acd{}
   {\vls([(\RD{a}.\RS{\bar a}).\SG{a}].\MG{\bar a})              }   {
\vlin\swi{}
   {\vls([(\RD{a}.[\GR{\bar a}.\YO{\bar a}]).\SG{a}].\MG{\bar a})}  {
\vlin=   {}
   {\vls((\RD{a}.[[\GR{\bar a}.\YO{\bar a}].\SG{a}]).\MG{\bar a})} {
\vlin\aid{}
   {\vls((\RD{a}.[\GR{\bar a}.[\YO{\bar a}.\SG{a}]]).\MG{\bar a})}{
\vlhy
   {\vls((\RD{a}.[\GR{\bar a}.\ttt]).\MG{\bar a})                }}}}}}}}}}}
\qquad&
%-------------------
\vlderivation                                                            {
\vlin=   {}{\vls(([\RS{a}.\YO{b}].\PW{a}).([\GR{a}.\DO{b}].\SG{a}))}    {
\vlin\med{}{\vls(([\RS{a}.\YO{b}].[\GR{a}.\DO{b}]).(\PW{a}.\SG{a}))}   {
\vlin\acu{}{\vls([(\RS{a}.\GR{a}).(\YO{b}.\DO{b})].(\PW{a}.\SG{a}))}  {
\vlin\acu{}{\vls([(\RS{a}.\GR{a}).(\YO{b}.\DO{b})].\MG{a})         } {
\vlin\acu{}{\vls([(\RS{a}.\GR{a}).\PR{b}].\MG{a})                  }{
\vlhy      {\vls([\RD{a}.\PR{b}].\MG{a})                           }}}}}}}
\\
\noalign{\bigskip}
%---------------------------------------
\vlderivation                                                      {
\vlin\swi{}{\vlsbr[\vlinf{\swi}
                         {}
                         {\vls[\vlinf{}
                                     {}
                                     {\fff}
                                     {\vls(\GR{a}.\RD{\bar a})}
                              \;.\;
                              \ttt]}
                         {\vls([\GR{a}.\ttt].\RD{\bar a})}
                  \;\;.\;\;
                   \ttt
                  ]                                            }  {
\vlin\med{}{\vls([\GR{a}.\ttt].[\ttt.\RD{\bar a}])             } {
\vlin{}  {}{\vls[\GR{a}.\RD{\bar a}]                           }{
\vlhy      {\ttt                                               }}}}}
\qquad&
%-------------------
\vlinf=
      {}
      {\vls(\DO{a}\;.\;\vlinf{}{}\fff{\vls(\PR{a}.\MG{\bar a})})}
      {\vlsbr(\vlinf\swi
                    {}
                    {\vls[\vlinf{}
                                {}
                                \fff
                                {\vls(\RD{a}
                                     \;.\;\vlinf{}
                                            {}
                                            {\RS{\bar a}}
                                            {\vls[\GR{\bar a}.\YO{\bar a}]}
                                     )}
                         \;\;.\;\;
                         \vlinf{}{}{\vls(\DO{a}.\PR{a})}{\SG{a}}
                         ]}
                    {\vls(\RD{a}
                         \;.\;[\GR{\bar a}
                          \;.\;\vlinf{}
                                 {}
                                 {\vls[\YO{\bar a}.\SG{a}]}
                                 \ttt
                          ]
                         )}
            \;\;\;\;.\;\;\;\;
            \MG{\bar a}
            )}
\qquad&
%-------------------
\vls(\vlinf\med
           {}
           {\vls([\RS{a}.\YO{b}].[\GR{a}.\DO{b}])}
           {\vls[\vlinf{}{}{\vls(\RS{a}.\GR{a})}{\RD{a}}
                \;.\;{\vlinf{}{}{\vls(\YO{b}.\DO{b})}{\PR{b}}}
                ]}
    \;\;.\;\;
     \vlinf{}{}{\vls(\PW{a}.\SG{a})}{\MG{a}}
    )
\\
\noalign{\bigskip}
%---------------------------------------
\vcenter{\hbox{\includegraphics{Figures/flowExtract1}}}
\qquad&
%-------------------
\vcenter{\hbox{\includegraphics{Figures/flowExtract2}}}
\qquad&
%-------------------
\vcenter{\hbox{\includegraphics{Figures/flowExtract3}}}
\end{array}
\]
\caption{Examples of derivations in the calculus of structures (top row), their translation into the functorial calculus (middle row), and the flows associated with the latter (bottom row).}
\label{figure:ExampleAtomicFlows}
\end{figure}
%-----------

%--------------------------------------------------
\begin{definition}\label{definiton:FlowRestriction}
Given a derivation $\Phi$ with flow $\phi$, and an atom $a$, the \emph{restriction of $\phi$ to $a$}\index{flow!restriction to atom} is the largest subflow $\phi_a$ of $\phi$, such that every edge of $\phi_a$ is mapped to from $a$ or $\bar a$.
\end{definition}
%---------------

%--------------
\begin{example}
Consider the rightmost derivation and its associated flow in Figure~\vref{figure:ExampleAtomicFlows}. The restriction of this flow to $a$ is:
\[
\vcenter{\hbox{\includegraphics{Figures/flowExtract4}}}
\]
\end{example}
%------------

\Tom{Gave the motivation for the following theorem.}

We now show that the definition of atomic flows is `minimal', in the sense that, if we restrict the definition, Proposition~\vref{proposition:FlowUnique} is no longer true.

%-------------------------------------------------
\begin{theorem}\label{theorem:SurjectiveDerToFlow}
Every atomic flow is associated with some derivation.
\end{theorem}

\begin{proof}
First, we show that, for any atom $a$ and formula contexts $\xi\vlhole$ and $\zeta\vlhole$, there exists a derivation
\[
\vlder{}{\{\swi,\med\}}
{
 \vls[(\xi\{\{a\}.\zeta\{\fff\}).\ttt]
}
{
 \vls[(\xi\{\{\ttt\}.\zeta\{a\}).\ttt]
}\quad,
\]
in other words we can `move' the atom $a$ from the context $\xi\vlhole$ to the context $\zeta\vlhole$ by using a derivation whose flow contains no vertices:
\[
\vlderivation
{
 \vlin{=}{}{\vls[(\xi\{a\}.\zeta\{\fff\}).\ttt]}
 {
  \vlin{=}{}
  {
   \vlsbr[\vlinf{\swi}{}{\vls[(\zeta\{\fff\}.\xi\{a\}).\ttt]}{\vls(\zeta\{\fff\}.[\xi\{a\}.\ttt])}\;.\;\ttt]
  }
  {
   \vlhy
   {
    \vlsbr[
    \vlderivation
    {
     \vlin{\swi}{}{\vls[([\xi\{a\}.\ttt].\zeta\{\fff\}).\ttt]}
     {
      \vlin{=}{}{\vls([\xi\{a\}.\ttt].[\zeta\{\fff\}.\ttt])}
      {
       \vlin{\med}{}{\vls([\xi\{a\}.\ttt].[\ttt.\zeta\{\fff\}])}
       {
        \vlin{=}{}{\vls[(\xi\{a\}.\ttt).(\ttt.\zeta\{\fff\})]}
        {
         \vlin{\supers}{}
         {
          \vls[\xi\{a\}.\zeta\{\fff\}]
         }
         {
          \vlhy
          {
           \vls(\xi\{\ttt\}.\zeta\{a\})
          }
         }
        }
       }
      }
     }
    }
    \;\;\;\;\;.\;\;\;\;\;\ttt]
   }
  }
 }
}
\quad.
\]
This construction can be used repeatedly to build the derivation $\Psi$, for $h\ge0$:
\[
\vlderivation                                                             {
\vlde{\Psi}{\{\swi,\med\}}
     {\vls[(\xi\{a_1 \}\cdots\{a_h \}.\zeta\{\fff\}\cdots\{\fff\}).\ttt]}{
\vlhy{\vls[(\xi\{\ttt\}\cdots\{\ttt\}.\zeta\{a_1 \}\cdots\{a_h \}).\ttt]}}}
\quad.
\]
We can now prove the theorem by induction on the number of vertices of a given flow $\phi$. The cases where $\phi$ only has zero or one vertex are trivial. Let us then suppose that $\phi$ has more than one vertex; then $\phi$ can be considered as composed of two flows $\phi_1$ and $\phi_2$, each with fewer vertices than $\phi$, as follows:
\[
\vcenter{\hbox{\includegraphics{Figures/surjectFlow}}}\quad,
\]
where $h\ge0$ (this can possibly be done in many different ways). By the inductive hypothesis, there exist derivations $\vlder{\Phi_1}{}{\zeta\{a_1^{\epsilon_1}\}\cdots\{a_h^{\epsilon_h}\}}{\gamma}$ and $\vlder{\Phi_2}{}{\delta}{\xi\{a_1^{\epsilon_1}\}\cdots\{a_h^{\epsilon_h}\}}$ whose flows are, respectively, $\phi_1$ and $\phi_2$. Using these, we can build
\[
\vlder{\Psi}{}
{
 \vlsbr[(
 \vlder{\Phi_2}{}{\delta}{\xi\{a_1^{\epsilon_1}\}\cdots\{a_h^{\epsilon_h}\}}
 \;\;.\;\;
 \zeta\{\fff\}\cdots\{\fff\})
 \;\;.\;\;
 \ttt]
}
{
 \vlsbr[(\xi\{\ttt\}\cdots\{\ttt\}
 \;\;.\;\;
 \vlder{\Phi_1}{}{\zeta\{a_1^{\epsilon_1}\}\cdots\{a_h^{\epsilon_h}\}}{\gamma}
 )
 \;\;.\;\;
 \ttt]
}
\quad,
\]
whose flow is $\phi$.
\end{proof}
%----------

\Tom{Added notation which used to be defined in a previous definition:}

%---------------
\begin{notation}
Given a derivation $\Phi$, an atom occurrence $a$ in $\Phi$ and the flow $\phi$ of $\Phi$, then, whenever we write $a^\epsilon$ or $a^\psi$, we mean that there is a subflow $\psi$ of $\phi$ containing the edge $\epsilon$, such that $a$ is mapped to $\epsilon$.
\end{notation}
%-------------

\Tom{Added intuition (from paper on quasipolynomial normalisation).}

We will now see how this notation might be useful when selectively substituting for atom occurrences. For example, let us suppose that we are given the following associated derivation and flow:
\[
\Phi=
\vlsbr[\vlderivation                                              {
       \vlin{    }{}\fff                                         {
       \vlin{\med}{}{\vls(\vlinf{}{}a{\vls[a.a]}
                         .\vlinf={}{\bar a}{\vls[\fff.\bar a]})}{
       \vlhy        {\vls[(a.\fff).(a
                                   .\vlinf{}{}{\bar a}\fff)]   }}}}
      \vlx.\vlx\bar a]
\qquad\text{and}\qquad
\vcenter{\hbox{\includegraphics{Figures/substEx}}}
\quad.
\]
We can then distinguish between the three occurrences of $\bar a$ that are mapped to edge $\one$ and the one that is not, as in
\[
\Phi=
\vlsbr[\vlderivation                                                        {
       \vlin{    }{}\fff                                                   {
       \vlin{\med}{}{\vls(\vlinf{}{}a{\vls[a.a]}
                         .\vlinf={}{\bar a^\one}{\vls[\fff.\bar a^\one]})}{
       \vlhy        {\vls[(a.\fff).(a
                                   .\vlinf{}{}{\bar a^\one}\fff)]        }}}}
      \vlx.\vlx\bar a]
\quad;
\]
we can also substitute for these occurrences, for example by $\{\bar a^\one/\fff\}$; such a situation occurs in the proof of Theorem~\vref{theorem:SoundIsolatedSubflowRemoval}. Note that simply substituting $\fff$ for $\bar a^\one$ would invalidate this derivation because it would break the cut and weakening instances; however, the proof of Theorem~\ref{theorem:SoundIsolatedSubflowRemoval} specifies how to fix the broken cut instance and Proposition~\vref{proposition:DerivationSubstitution} specifies how to fix the broken weakening.

We generalise this labelling mechanism to boxes. For example, we can use a different representation of the flow of $\Phi$ to individuate two classes $a^\phi$ and $\bar a^\phi$ of atom occurrences, as follows:
\[
\Phi=
\vlsbr[\vlderivation                                                        {
       \vlin{    }{}\fff                                                   {
       \vlin{\med}{}{\vls(\vlinf{}{}{a^\phi}{\vls[a.a]}
                         .\vlinf={}{\bar a^\phi}{\vls[\fff.\bar a^\phi]})}{
       \vlhy        {\vls[(a.\fff).(a
                                   .\vlinf{}{}{\bar a^\phi}\fff)]        }}}}
      \vlx.\vlx\bar a^\phi]
\qquad\text{and}\qquad
\vcenter{\hbox{\includegraphics{Figures/substExBox}}}
\quad.
\]
This notation is used in Proposition~\vref{proposition:DerivationSubstitution}, where we define how we can, in certain cases, substitute formulae in place of atom occurrences. This technique is used in Theorem~\vref{theorem:SoundFourBoxes}, Theorem~\vref{theorem:SoundIsolatedSubflowRemoval} and Theorem~\vref{theorem:SoundMultipleIsolatedSubflowsRemoval}.

\Tom{Moved the following Proposition, Notation and Remark from the sectoin on Global Reductions.}

%------------------------------------------------------------
\begin{proposition}\label{proposition:DerivationSubstitution}
Given a derivation\/ $\vlder\Phi\SKS{\beta}\alpha$, let its associated flow have shape
\[
\vcenter{\hbox{\includegraphics{Figures/propSubst1}}}\quad,
\]
such that $\phi$ is a connected component whose edges are each associated with occurrences of the atom $a$; then, for any formula $\gamma$, there exists a derivation
\[
\vlder\Psi\SKS{\beta \{a^\phi/\gamma\}}
              {\alpha\{a^\phi/\gamma\}}
\]
whose associated flow is
\[
\vcenter{\hbox{\includegraphics{Figures/propSubst2}}}
\]
where $n$ is the number of atom occurrences in $\gamma$; moreover, the size of\/ $\Psi$ depends linearly on the size of\/ $\Phi$ and quadratically on the size of $\gamma$.
\end{proposition}

\begin{proof}
We can proceed by structural induction on $\Phi$. For every formula in $\Phi$ we substitute $a^\phi$ with $\gamma$. Since all the edges in $\phi$ are mapped to from $a$ (and not $\bar a$), we know that all the vertices in $\phi$ are mapped to from instances of $\acd$, $\acu$, $\awd$ and $\awu$. We substitute every instance of $\acd$, $\acu$, $\awd$ and $\awu$ where $a^\phi$ appears, by $\cod$, $\cou$, $\wed$, $\weu$, respectively, with $\gamma$ in the place of $a^\phi$. The result then follows by Lemma~\vref{lemma:GenericWeakening} and Lemma~\vref{lemma:GenericContraction}.
\end{proof}
%----------

%---------------
\begin{notation}
The derivation $\Psi$ obtained in the proof of Propostion~\vref{proposition:DerivationSubstitution} is denoted $\Phi\{a^\phi/\gamma\}$.
\end{notation}
%-------------

%-------------
\begin{remark}
The notion of substitution can be extended to allow $\phi$ to contain interaction and cut vertices, but we shall not need that in this thesis.
\end{remark}
%-----------

%=================================================================
\section{A Normal Form of Derivation}\label{section:DerNormalForm}

In this section we introduce the \emph{$\ai$-decomposed form} of a derivation. The reason for introducing this normal form is that we will often find it convenient to assume that identity instances appear at the top and cut instances appear at the bottom of a derivation. The important features of this normal form is that a derivation can be transformed into $\ai$-decomposed form without changing its atomic flow, and without significantly changing its size.

\begin{definition}\label{definition:aiDecomposedForm}
Given two derivations
\[
\vlder{\Phi}{}{\beta}{\alpha}
\quad\mbox{and}\quad
\Psi\;=\;
\vlder{}{\SKS\setminus\{\aid,\aiu\}}
{
 \vlsbr[\beta\;.\;\vlinf{}{}{\fff}{\vls(b_m.\bar b_m)}\;.\;\cdots\;.\;\vlinf{}{}{\fff}{\vls(b_1.\bar b_1)}]
}
{
 \vlsbr(\vlinf{}{}{\vls[a_1.\bar a_1]}{\ttt}\;.\;\cdots\;.\;\vlinf{}{}{\vls[a_n.\bar a_n]}{\ttt}\;.\;\alpha)
}\quad,
\]
for some atoms $a_1,\dots,a_n,b_1,\dots,b_m$, such that $\Phi$ and $\Psi$ have isomorphic flows, we say that $\Psi$ is an \emph{$\ai$-decomposed form of\/ $\Phi$}\index{$\ai$-decomposed form}.
\end{definition}

\begin{convention}\label{convention:AlternativeAiDecomposedForm}
Given a derivation $\Phi$ and an $\ai$-decomposed form of\/ $\Phi$:
\[
\vlder{}{\SKS\setminus\{\aid,\aiu\}}
{
 \vlsbr[\beta\;.\;\vlinf{}{}{\fff}{\vls(d_l.\bar d_l)}\;.\;\cdots\;.\;\vlinf{}{}{\fff}{\vls(d_1.\bar d_1)}\;.\;\vlinf{}{}{\fff}{\vls(b_m.\bar b_m)}\;.\;\cdots\;.\;\vlinf{}{}{\fff}{\vls(b_1.\bar b_1)}]
}
{
 \vlsbr(\vlinf{}{}{\vls[a_1.\bar a_1]}{\ttt}\;.\;\cdots\;.\;\vlinf{}{}{\vls[a_n.\bar a_n]}{\ttt}\;.\;\vlinf{}{}{\vls[c_1.\bar c_1]}{\ttt}\;.\;\cdots\;.\;\vlinf{}{}{\vls[c_k.\bar c_k]}{\ttt}\;.\;\alpha)
}\quad,
\]
we sometimes want to single out only some of the interaction or cut instances. We therefore also call the following, partially sequentialised, derivation an $\ai$-decomposed form of $\Phi$:
\[
\vlderivation
{
 \vlin{=}{}
 {
  \vlsbr[\beta\;.\;\vlinf{}{}{\fff}{\vls(b_m.\bar b_m)}\;.\;\cdots\;.\;\vlinf{}{}{\fff}{\vls(b_1.\bar b_1)}]
 }
 {
  \vlde{}{\SKS\setminus\{\aid,\aiu\}}
  {
   \vlsbr[\beta\;.\;\vlinf{}{}{\fff}{\vls(d_l.\bar d_l)}\;.\;\cdots\;.\;\vlinf{}{}{\fff}{\vls(d_1.\bar d_1)}\;.\;(b_m.\bar b_m)\;.\;\cdots\;.\;(b_1.\bar b_1)]
  }
  {
   \vlin{=}{}
   {
    \vlsbr([a_1.\bar a_1]\;.\;\cdots\;.\;[a_n.\bar a_n]\;.\;\vlinf{}{}{\vls[c_1.\bar c_1]}{\ttt}\;.\;\cdots\;.\;\vlinf{}{}{\vls[c_k.\bar c_k]}{\ttt}\;.\;\alpha)
   }
   {
    \vlhy
    {
     \vlsbr(\vlinf{}{}{\vls[a_1.\bar a_1]}{\ttt}\;.\;\cdots\;.\;\vlinf{}{}{\vls[a_n.\bar a_n]}{\ttt}\;.\;\alpha)
    }
   }
  }
 }
}\quad.
\]
\end{convention}
%-----------

%----------------------------------------------
\begin{theorem}\label{theorem:aiDecomposedForm}
Given a derivation $\Phi$, an $\ai$-decomposed form of\/ $\Phi$ whose size depends at most cubically on the size of $\Phi$ can be constructed.
\end{theorem}
\begin{proof}
Using Lemma~\vref{lemma:SuperSwitch} apply, from top-to-bottom and left-to-right, the following transformations to each of the identity and cut instances in $\Phi$:
\[
\vlderivation
{
 \vlde{\Psi'}{}{\beta}
 {
  \vlde{\Psi}{}{\xi\left\{\vlinf{}{}{\vls[a.{\bar a}]}{\ttt}\right\}}
  {
   \vlhy{\alpha}
  }
 }
}\quad\rightarrow\quad
\vlderivation
{
 \vlde{\Psi'}{}{\beta}
 {
  \vlin{\ssu}{}{\xi\vlsbr[a.{\bar a}]}
  {
   \vlhy{\vlsbr(\vlinf{}{}{\vls[a.{\bar a}]}{\ttt}\;\;.\;\;\vlder{\Psi}{}{\xi\{\ttt\}}{\alpha})}
  }
 }
}\qquad\mbox{and}\qquad
\vlderivation
{
 \vlde{\Psi'}{}{\beta}
 {
  \vlde{\Psi}{}{\xi\left\{\vlinf{}{}{\fff}{\vls(a.{\bar a})}\right\}}
  {
   \vlhy{\alpha}
  }
 }
}\quad\rightarrow\quad
\vlderivation
{
 \vlin{\ssd}{}{\vlsbr[\vlder{\Psi'}{}{\beta}{\xi\{\fff\}}\;\;.\;\;\vlinf{}{}{\fff}{\vls(a.{\bar a})}]}
 {
  \vlde{\Psi}{}{\xi\vlsbr(a.{\bar a})}
  {
   \vlhy{\alpha}
  }
 }
}\quad
\]
to obtain an $\ai$-decomposed form of $\Phi$. The size of the $\ai$-decomposed form obtained in this way depends at most cubically on the size of $\Phi$, since, by Lemma~\vref{lemma:SuperSwitch}, each of the transformations increase the size of the derivation at most quadratically and the number of transformations is bound by the size of $\Phi$.
\end{proof}

\begin{remark}\label{remakr:aiDecomposedFormUnique}
The only reason to insist on performing the transformations in the proof of Theorem~\vref{theorem:aiDecomposedForm} in a certain order is to ensure that the resulting derivation is unique. The uniqueness is useful in the following definition.
\end{remark}

\begin{definition}\label{definition:TheAiDecomposedForm}
Given a derivation $\Phi$, the $\ai$-decomposed form of $\Phi$ obtained as described in the proof of Theorem~\vref{theorem:aiDecomposedForm} is called \emph{the} (\emph{canonical}) \emph{$\ai$-decomposed form of\/ $\Phi$}\index{$\ai$-decomposed form!canonical}.
\end{definition}